%        File: Lamentaciones.tex
%     Created: Mon Nov 11 08:00 PM 2019 C
% Last Change: Mon Nov 11 08:00 PM 2019 C
%
%\documentclass[12pt]{article}
%\usepackage{enumerate}
%\usepackage[spanish]{babel}
%\usepackage[margin=1.0in]{geometry}
%\begin{document}
\begin{section}{Lamentaciones}
	\begin{itemize}
		\item Título\\
			Le fue dado debido a que fue escrito como un quebrantamiento del corazón, sentimientos de la persona que lo escribió. El nombre viene de la Septuaginta donde se le nombra en esta forma en el griego, le dan el nombre de ``Eijá'' que se traduce ``como``, siguiendo la tradición de nombrar los libros de acuerdo a las palabras con las que empezaban.\\
			El profeta no solamente expresa dolor sino que mantiene su fe firme en Dios sabiendo que era preciso esperar. Podemos ver los gritos de un intenso dolor sobre lo ocurrido, coinciden en 5 poemas llenos de anguistia y horror. Jeremías es quien narra en ambos libros la tragedia y el dolor en la historia de Israel, el incio ocurrrió cuando Sedequías se rebela en contra de Nabucodonosor, provocando el ataque a Jersualén en $2^{o}$ Reyes 24. Mientras las fuerzas de Nabucodonosor tenían sitiada la ciudad empezó a escasear la comida y se estaban muriendo de hambre dentro de los muros. Cuando se abrió paso de los muros, huyeron el rey y los hombres de guerra y al alcanzarlo Nabucodonosor hizo que viera cómo degollaron a sus hijos y lo llevaron cautivo a Babilonia después de haberle sacado los ojos.\\
			El ejército de Nabucodonosor destruyó la mayoría de la ciudad, quemaron el templo y se llevaron cautivo al pueblo.\\
			Los poemas parecen haber sido compuestos durante y después de estos hechos.
		\item Autor y fecha\\
			En el texto hebreo del Antiguo Testamento no hay algo que lo relacione con Jeremías pero en la Septuaginta hay una nota que dice que Jeremías se sentó a llorar y entonó esta lamentación por Jerusalén. Esta nota no está en los originales pero fue agregada y pasó a la versión Vulgata por lo que el libro fue conocido como las lamentaciones de Jeremías. La tradición dice que hay una gruta a la cual se le llama ''gruta de jeremías`` donde se dice que ahí lloró el profeta y escribió estos cantos de tristeza. Es el mismo donde se escribió que Jesús fue crucificado, el Gólgota. La autoría del libro ha sido puesta en duda dada sus diferencias con el libro de Jeremías pero se sabe que compuso lamentos por $2^{o}$ Crónicas 35:25. Dichas diferencias se pueden deber a que las circunstancias eran distintas. \\
			La fecha más reciente posible de escritura es el año 586 a.C. que fue cuando se celebró la reconstrucción del templo. Estos poemas son desgarradores cuando contrastan con tantas promesas y bendiciones y muestras del poderío que Dios le dio a su pueblo en sus momentos de sufrimiento pero el pueblo escogido por Dios había perdido todo y estaba en una situación desesperada. En los poemas también se describe el ministerio de Jeremías como profeta según estas nuevas circunstancias del pueblo. Dios consoló a Jeremías de esta pena y lo ayudó a someterse al juicio divino.\\
			Después del año 586 a.C. comenzaron a celebrarse junto a las ruinas del templo ceremonias conmemorativas de esta catástorfe que había sufrido el pueblo, habían oraciones y ayuno. Se mantenían vivos el recuerdo de la tragedia y la esperanza de la restauración prometida y anunciada por los profetas.
			\newpage
			En estos poemas se presenta a Jerusalén como una mujer que quedó viuda pero, sobre todo, cómo el pueblo confiesa que todo era consecuencia de sus pecados, por ello es que contiene lamentos y expresiones de profunda confianza en Dios. Se habla de cantos de alabanza y reflexión sobre la disciplina que Dios le había puesto. El pueblo entendió que lo que había dicho Jeremías sí era Palabra de Dios.
		\item Tema\\
			Las consecuencias del pueblo por su pecado. La única opción era buscar la restauración y tener comunión con Dios. En el libro de Jeremías, él fue conocido como el profeta llorón por su gran pasión, este mismo dolor por el pecado del pueblo y su rechazo a Dios fue expresado de la misma manera por Jesús cuando estuvo en la Tierra y lloraba por nuestros pecados. Este libro consiste en 5 poemas que lamentan un acontecimiento terrible para el pueblo judío, esta catástrofe dividió la historia judía antigua en el antes y el después de la derrota. Estas lamentaciones les recordaban a los sobrevivientes la tragedia que era motivo de su pecado y que se pudo haber evitado si le hubieran hecho caso a las advertencias de Jeremías. Debían de tener una esperanza de que iba a mantener el nuevo pacto que Jeremías les había dicho que iba a tener para siempre con Su peublo.	
		\item Propósito\\
			Mostrar que aún después de la disciplina y el pecado puede haber una restauración con Dios. El pecado y la rebelión del pueblo fueron las causas de que se huiera derramado la ira de Dios, dando así paso al arrepentimiento que es consecuencia de la disciplina. Dios es un Dios de esperanza, sin importar el tamaño de nuestro pecado podemos encontrar su perdón y compasión. La promesa de restauración incluía el retorno a la tierra prometida y que Israel tendría una posición de gloria ante el resto del mundo. EL libro conluye con gracia aunque el retorno de la prosperidad de Israel no vendría en ese tiempo sino que llegará cuando Dios lo determine. Ante las promesas de salvación se destaca la promesa de que iba a restablecer la relación que tenía con el pueblo después de que dicha relación se había roto debido a su infidelidad.
	\end{itemize}
	\begin{subsection}{Bosquejo}
		Tambén puede clasificarse como un libro poético ya que 4 de 5 capítulos están escritos de forma acróstica de acuerdo al alfabeto hebreo. Lamentaciones 5 tiene 22 versículos pero no está escrito de forma acróstica. Dicha forma de escritura era para facilitar su aprendizaje, el bosquejo está marcado por cada capítulo.
		\begin{subsubsection}{La ruina de Jerusalén (Lamentaciones 1)}
			Lamentaciones 1:1 es la primera de las lamentaciones, compara la ciudad con una viuda que no tiene quien la consuele. Ya no había fiestas tradicionales en Jerusalén, quedaron algunos sacerdotes pero estaban solos. Se personifica la ciudad de Jerusalén como si ella hubiera pecado, la describe como ciudad vergonzosa y despreciable.
			\newpage
			La gente pasaba y no se compadecía de cómo había quedado la ciudad, en Lamentaciones 1:18 Jeremías reconoce que Dios es justo y que todo había sido su culpa.\\
		\end{subsubsection}
		\begin{subsubsection}{La ira de Dios (Lamentaciones 2)}
			En Lamentaciones 2:2-3 el lamento de Jeremías es sobre la ira que Dios derramó sobre su pueblo y lo dejó desprotegido en manos de sus enemigos. En aquel conflicto Dios estaba del lado del enemigo de Israel debido al pecado de Judá. Dios mismo mandó destruir el templo al que tenían como un amuleto y permitió que ya no hubiera alimento ni para los niños, siempre cumplió su palabra. En Jeremías 2:20 dice que el hambre había producido canibalismo en los habitantes mientras que Jeremías 2:21 menciona que hubo mortandad total, tal y como lo menciona $2^{o}$ Crónicas 36:17.
		\end{subsubsection}
		\begin{subsubsection}{Petición por misericordia (Lamentaciones 3)}
			EL lamento de Jeremías se puede resumir en Lamentaciones 3:19-23. Esos momentos por los que pasaba Jeremías no iban a ser fáciles de olvidar y el profeta recuerda que por la misericordia de Jehová no fueron consumidos todos. En Lamentaciones 3:1-18 muestra su condición, Jeremías oraba y Dios no le contestaba. El fundamento firme de la fe de Jeremías era que confiaba en que Dios cumpliría sus promesas de acuerdo a sus atributos como lo dice en Lamentaciones 3:24.\\
			Jeremías vuelve a recordar que todo por lo que estaban pasando era culpa de ellos y a partir de Lamentaciones 3:46 hace mención de sus enemigos, le ruega a Dios que aplique su justicia en contra de sus enemigos, cosa que vemos cumplida en Isaías.
		\end{subsubsection}
		\begin{subsubsection}{El repaso del sitio (Lamentaciones 4)}
			Jeremías hace un repaso de la condición en la que se encontraban los pocos habitantes que quedaban en la ciudad. Compara como los animales pueden cuidar a sus crías pero las mujeres no podían cuidar de sus hijos. En Lamentaciones 4:6 dice que Dios consideró el pecado de su pueblo aún más grande que el de Sodoma. Vuelve a recodrar en Lamentaciones 4:10 el canibalismo que estaba ocurriedo por el hambre que estaban padeciendo. Las autoridades habían sido las culpables directas de que el pueblo se hubiera contaminado con lo que ellos estaban practicando. En Lamentaciones 4:15 se muestra el rechazo contra aquellos que habían hecho esto, en Lamentaciones 4:21 advierte que la ira de Dios iba a llegar hasta a Edom.
		\end{subsubsection}
		\begin{subsubsection}{Petición por restauración (Lamentaciones 5)}
			Jeremías le pide a Dios que sean restaurados, que acabara la aflicción y todo fuera como el principio ya que estaban muriendo de hambre, eran perseguidos, los jóvenes habían sido esclavizados, todos en el pueblo eran miserables pero Jeremías vuelve a confiar en Dios.\\
			Se aboradan 6 temas fundamentales vinculados con el sufrimiento:
			\begin{enumerate}
				\item El sufrimiento fue a causa de su pecado.\\
					Ello se menciona en cada capítulo del libro. 
					\newpage
					Esto era aceptado, aún los caldeos reconocían que su sufrimiento no era casualidad. Al autor de Lamentaciones le interesaba la situación espiritual de su pueblo que debía de asumir su responsabilidad.
				\item Se percibía como que venía de Dios y no de los hombres.\\
					No menos de 44 versículos se refieren a este hecho. 
				\item Su sufrimiento podría llevarlos nuevamente a Dios.\\
					Se muestra Jeremías consciente de Dios, sus propósitos y Su relación con su pueblo, nada indica que el sufrimiento fuera porque Dios los hubiera abandonado definitivamente. 
				\item Sufrimiento, lágrimas y oraciones son cosa que van juntas.\\
					La gente fue alentada a abrir sus corazones a Dios y contarle los detalles de su dolor y sufrimiento, es ahí cuando las peronas tienen más comunión con Dios, cada capítulo excepto el 4 termina con una oración. Lamentaciones 5 es una oración que muestra que el pueblo verdadermamente lamentaba lo que habían pasado y estaban arrepentidos. Las oraciones contienen un lenguaje matizado por su dolor, indicando que así es como se puede tener comunión con Dios en la tribulación.
				\item La oración siempre debe buscar algún rayo de esperanza.\\
					Por muy difícil que sea la situación, no debemos claudicar. Una nueva consciencia aparece a partir de Lamentaciones 3:21 pues habla sobre la esperanza y la fidelidad de Dios. Una manifestación de la disciplina de Dios no significa la extinción de su amor por Su pueblo. Puede que Dios haya usado a Babilonia pero ello no significa que pueda equipararse a Israel. Dios no los volvería a abandonar de la misma forma.
				\item Aceptar su sufrimiento.\\
					Tenían que sufrir con paciencia su pena, seguros de que todo terminaría. El dolor que tuvo Jeremás le calaba muy hondo, por lo cual lloraba con un corazón verdaderamente quebrantado. Él sabía lo que le esperaba a Judá, lo que sería de la capital, su juicio caería y habrá destrucción completa. Sufrió mucho pero la verdadera causa de que llorara era que el pueblo rechazaba el mensaje del Dios que los había escogido y bendecido en múltiples ocasiones, la maldad y el egoísmo les iba a traer sufrimiento y los llevaría al cautivero.\\
					Los dos libros de Jeremías se centran en la destrucción de Jerusalén, Lamentaciones es una canción fúnebre escrita para la ciudad caída de Jerusalén. Podemos ver que Dios estaba en el corazón de Jeremías y permaneció sólo en la profundidad de sus emociones.\\
					Lamentaciones es uno de los libros que se leen en las fiestas judías como parte del Megillot junto con Cantares, Rut, Eclesiastés y Ester.
			\end{enumerate}
		\end{subsubsection}
	\end{subsection}
\end{section}
%\end{document}


