%        File: Proverbios.tex
%     Created: Wed Oct 23 05:00 PM 2019 C
% Last Change: Wed Oct 23 05:00 PM 2019 C
%
%\documentclass[12pt]{article}
%\usepackage{enumerate}
%\usepackage[spanish]{babel}
%\usepackage[margin=1.0in]{geometry}
%\begin{document}
\begin{section}{Proverbios}
	\begin{itemize}
		\item Título\\
			Proviene de la forma por la que empieza el libro, estableciendo que habla acerca de los proverbios de Salomón. La palabra proverbio en hebreo se traduce como un discurso, ejemplo o parábola.\\
			En hebreo se refiere a la capacidad que tiene una persona para decir refranes.
		\item Autor y Fecha\\
			Proverbios 1:1 hace una afirmación acerca del autor del libro aunque dentro del mismo texto, en los capítulos 30 y 31 se refiere a la autoría de otros hombres fuera de Salomón. Los otros autores son Agur y Lemuel, Lemuel era rey y su madre lo había instruido sobre la forma en que debía de referirse un rey. En la tradición se considera que Lemuel fue precisamente Salomón. Los primeros 9 capítulos se consideran escritos por Salomón mientras que los posteriores se consideran que solamente fueron agrupados por él. En Proverbios 25:1 se declara que fueron proverbios copiados por escribas de Ezequías.\\
			En $1^{o}$ Reyes 4:32 se habla de la habilidad de Salomón para escribir proverbios.\\
			Se considera que el libro fue escrito entre 971-931 a.C. durante el reinado de Salomón.
		\item Tema Principal\\
			La sabiduría de Dios. El libro muestra la sabiduría divina otorgada al hombre con el objetivo de que él pudiera vivir una vida plena buscando guardar la Palabra de Dios y siendo humildes. Ésto está basado en Proverbios 1:7.\\
			La sabiduría del libro no procede de la inteligencia humana, sin embargo, Dios se la da a quienes la buscan de acuerdo al mismo libro.
		\item Propósito\\
			Exortar a los simples a buscar estabildad. Entre ellos se les considera a las personas de poca edad, no maduras y que no tienen un sano criterio, es decir, jóvenes y gente sin experiencia.\\
			La sabiduría del libro busca que los jóvenes libren dolor, a través de todo el libro se expresa urgencia para que todos los hombres conozcan la voluntad de Dios y vivan una vida plena y agradable a Dios. Se trata de la salvación o la perdición deuna vida y por ello se transforma en un mensajero del evangelio que nos transmite la sabiduría de Dios.
	\end{itemize}
	\begin{subsection}{Bosquejo}
		\begin{subsubsection}{Introducción a los proverbios (1-1:7)}
			\end{subsubsection}
			\begin{subsubsection}{La sabiduría de los proverbios (1:8-29:27)}
			\end{subsubsection}
			\begin{subsubsection}{Proverbios de Agur y Lemuel (30:1-31:31)}
			\end{subsubsection}
			\begin{subsubsection}{Conceptos particulares del libro de proverbios}
				\begin{itemize}
					\item Sabiduría condensada en frases cortas que son penetrantes con contrastes dramáticos tomados de la vida real para que la gente las aprenda fácilmente. 
					\item Enseña a todos, es didáctico, es decir, nos sirve para dar instrucción orientado principalmente a la enseñanza de los jóvenes a través de repetir estos pensamientos juiciosos.
					\item Da enseñanza a toda persona pero al ser didáctico está orientado principalmente para enseñanza a los jóvenes que lleven una vida justa.
					\item Se establece lo que es y no es justo de manera objetiva aunque la filosofía humana le ha dado valores relativos a la moral. En la Biblia está claramente especificado lo que es bueno y lo que es malo, ello está establecido de acuerdo al carácter de Dios. La esencia de la sabiduría queda establecida.
					\item Aplican la enseñanza de Dios a toda la vida humana sin importar su situación, en todas las relaciones que tenemos con todas la personas, actitudes, reacciones a lo que hacemos, decimos y pensamos. Estas enseñanzas se pueden aplicar para toda situación.
					\item Dios muestra al hombre lo que es mejor para su vida aunque no sean creyentes y no conozcan de Dios, el hombre natural entiende que el libro está basado en las experiencias de la vida real de los autores. 
					\item Con frecuencia se encuentran mandatos expecíficos sobre la manera de actuar, así como las posibles consecuencias sobre no actuar de la forma en la que se menciona.
					\item Utliza mucho el lenguaje figurado. 
					\item Apuntan tanto a los jóvenes ingenuos como a hombres entendidos pues aun el más sabio tiene de qué aprender. Da discernimiento y enseña cómo evitar caer en la trampa de los necios. La juventud se refiere a un estado de ánimo en el cual predomina la alegría y el deseo de experimentarlo todo. Aún sintiéndose suficiente el joven no está capacitado para hacer lo que le conviene pues no tiene experiencia, por ello es que están abiertos a cualquier influencia y en su afán por experimentar pueden seguir dichas influencias a pesar de que haya personas que solamente buscan hacerles daño.\\
						Por otro lado en Proverbios 1:5 también se refiere a que los sabios también deben de aprender, siempre se puede encontrar sabiduría divina en este libro. El que los ecribió, Salomón, se perdió en su vejez a pesar de que él lo había escrito. Este libro debe de considerarse solamente como un libro de guías y de observaciones sabias que nos deja Dios pero se deben de tomar solamente como principios que se deben de cumplir. Son consejos, no algo que forzosamente se vaya cumplir (promesa de Dios). Es necesario que los estudiemos para conocer de Dios.\\
						Si seguimos estos consejos, los resultados son lógicos. Siguiendo estos consejos los piadosos y los sabios lograrán muchas recompensas, quienes los aprendan tedrán exito y estarán en condiciones de ser de ejemplo.\\
						En muchos de estos consejos ni siquiera se menciona a Dios ni la preocupación por el éxito, parecen hasta consejos mundanos pero el contexto general del libro busca una vida santa. El conocimiento de Dios solamente lo alcanzamos cuando buscamos su sabiduría. Se menciona en $1^{a}$ Juan que quien conoce a Dios es quien guarda Sus mandamientos.
						\newpage
					\item Está basado en el principio divino del temor a Dios. Cuando Israel temía a Dios, le obedecían y continuaban con una actitud correcta hacia Él. El secreto de la justicia se enuentra precisamente en la obediencia. El temor a Dios es un don otorgado por el Espíritu Santo.
				\end{itemize}
			\end{subsubsection}
			\begin{subsubsection}{El contenido teológico de los proverbios}
				En la septuaginta se omiten algunos versículos y hay un orden distinto, la razón de ello se desconoce. Sin embargo, podemos ver que prácticamente todas las traducciones separan los capítulos 30 y 31 como independientes del resto del libro.\\
				\begin{itemize}
					\item Aspecto cosmológico\\
						El libro habla sobre el origen del mundo y lo que vemos, cómo es nuestra naturaleza, cómo debemos de vivir, la esperanza que tenemos y qué ocurre después de la muerte. Todo ello  lo podemos entender en este libro con respuestas que nos edifiquen y nos brinden una comunión directa con Dios.
					\item Aspecto antropológico\\
						La teología hebrea clásica estaba basada en elección especial ya que el pueblo judío piensa que Proverbios está especialmente escrito para el pueblo judío como pueblo elegido pero en el libro no se pone únicamente al hombre israelita sino que habla del hombre en general, por ello es antropológico. Más de 100 veces se menciona al hmbre común y corriente, no a un judío, por lo que concluimos que es para toda la humanidad. Todo el contenido del libro se refiere a cualquier persona sin referirse a un pacto. En todo el Antiguo Testamento la teología hebrea estaba escrita de acuerdo a lo que Dios había pactado con ellos pero en este libro esta teología se extiende para todo hombre y cualquier situacion o condición.\\
						La teología hebrea estaba estructurada de acuerdo a las intervecnciones que Dios había hecho para salvar a Su pueblo pero Proverbios nos habla de la existencia cotidiana de cualquier hombre. Su teología estaba basada en la ley mosaica como norma en su vida y en el Nuevo Testamento la ley que seguimos es solamente la moral.\\ 
						En este libro hay muchos imperativos para todos lo hombres, propuestas concretas sobre cómo debemos de llevar a cabo nuestra vida en sabiduría. La sabiduría de Proverbios tiene preceptos que son de fruto de la experiencia de un hombre. No habla tanto de la obediencia sino que tengamos un discernimiento sobre lo que tenemos que hacer, debemos de entender que la teoogía del libro es universal.\\
						Nos muestra la sabiduría verdadera, práctica, que esta saca de la vida cotidiana del hombre sabiendo que el éxito de la vida está basado en que la persona haga lo que se pide o enseña en el libro.\\
						Hay una gran cantidad de provebrios citados en el Nuevo Testamento, entre ellos se ecnuentran: Proverbios 3 es citado en Hebreos 12, Santiago 4:6 y $1^{a}$ Pedro 5:5; Proverbios 11:31 es citado en $1^{a}$ Pedro 4:18; Prverbios 25:21-22 es citado en Romanos 12:20; Proverbios 26:11 es citado en $2^{a}$ Pedro 2:22, entre otras citas.
						\newpage
						También hay una utilidad en estudiar los proverbios de manera temática ya que vemos la relación que el hombre guarda con Dios.\\
						La relación del hombre con Dios
						\begin{itemize}
							\item Su confianza (Proverbios 22:19)
							\item Su humildad (Proverbios 3:34)
							\item Su temor a Dios (Proverbios 1:7)
							\item Su justicia (Proverbios 10:25)
							\item Su pecado (Proverbios 28:13)
							\item Su obediencia (Proverbios 12:15)
							\item Enfrentando recompensa (Proverbios 11:31)
							\item Enfrentado pruebas (Proverbios 17:3)
							\item Enfrentando bendición (Proverbios 10:22)
							\item Enfrentando la muerte (Proverbios 15:11)
						\end{itemize}
					\item La relación del hombre con el hombre
						\begin{itemize}
							\item Su identidad (Proverbios 20:11)
							\item Su sabiduría (Proverbios 1:5)
							\item Su insensatez (Proverbios 26:10-11)
							\item Su conversación (Proverbios 18:21)
							\item Su dominio propio (Proverbios 2:11-12)
							\item Su bondad (Proverbios 20:6)
							\item Su riqueza (Proverbios 11:4)
							\item Su orgullo (Proverbios 21:4)
							\item Su enojo (Proverbios 29:11)
							\item Su pereza (Proverbios 13:4)
						\end{itemize}
					\item Relación del hombre on otros
						\begin{itemize}
							\item Su amor (Proverbios 10:12)
							\item Sus amigos (Proverbios 17:17)
							\item Sus enemigos (Proverbios 16:7)
							\item Su veracidad (Proverbios 23:23)
							\item Su chisme (Proverbios 20:19)
							\item Como un padre (Proverbios 20:7, 31:2-9)
							\item Como una madre (Proverbios 29:15)
							\item Como hijos (Proverbios 3:1-3)
							\item Al educar hijos (Proverbios 4:1-4) 
							\item Al disciplinar hijos (Proverbios 22:6)
						\end{itemize}
				\end{itemize}
			\end{subsubsection}
		\end{subsection}
\end{section}
%\end{document}


