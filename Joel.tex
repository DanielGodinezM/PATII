%        File: Joel.tex
%     Created: Wed Nov 20 02:00 PM 2019 C
% Last Change: Wed Nov 20 02:00 PM 2019 C
%
%\documentclass[12pt]{article}
%\usepackage{enumerate}
%\usepackage[spanish]{babel}
%\usepackage[margin=1.0in]{geometry}
%\begin{document}
\begin{section}{Joel}
	\begin{enumerate}
		\item Título\\
			El profeta Joel profetizó para el reino del sur y en Joel 1:1 es reconocido como hijo de Petuel. El título del libro le fue dado debido a este profeta del cual se habla.
		\item Autor y fecha\\
			La autoría del libro le es dada a Joel mismo a quien se le piensa que era un agricultor debido a la narración que menciona conceptos que tienen mucha familiaridad con el contexto agrícola. No se considera que haya sido sacerdote o levita ya que cuando habla de los sacerdotes y los levitas habla de ellos en segunda persona. La tradición lo tiene considerado como parte de la tribu de Rubén a pesar de que ni siquiera su padre es mencionado en alguna otra parte de las Escrituras.\\
			La fecha de escritura se puede deducir debido a la similitud del estilo que tiene con el de Amós, además de que comparten 2 pasajes que tienen mucha semejanza: Joel 3:16 parecido a Amós 1:2 y también Joel 3:18 y Amós 9:13, prácticamente escritos con las mismas palabras.\\
			Joel es considerado como uno de los primeros profetas que escribieron sus profecías en la historia del pueblo judío. Aproximadamente 200 años antes de que el pueblo del sur escuchara las profecías de Jeremías, Joel profetizó el mismo mensaje. Por los contenidos del libro se concluye que probablemente vivía en Jerusalén y fue contemporáneo de Elías o Eliseo.\\
			Por estos detalles hallados dentro del libro podemos determinar que la fecha de escirtura fue en los años 835-796 a.C. aproximadamente, durante el reinado de Joás.
		\item Tema\\
			El día del Señor. Un tema escatológico que no referencía un día en especial sino un periodo general de juicio e ira de parte de Dios. Este periodo estaría acompañado de desastres naturales y de destrucción. Será exclusivamente el día en el que el Mesías reine por 1000 años, motivo de terror para sus enemigos.
		\item Propósito\\
			Mostrar el juicio que también se profetizó sobre el pecado de Judá. Juicio que impondrá Dios a Su pueblo por su pecado aunque también habla de su restauración. En aquel día habrá desastre naturales de toda índole y al final del juicio habrá restauración y esperanza de salvación.
	\end{enumerate}
	\begin{subsection}{Bosquejo}
		\begin{subsubsection}{El día del Señor: Langostas (Joel 1)}
El alejamiento que el pueblo estaba teniendo de Dios no se puede comparar con el alejamiento del reino del norte pero Joel estaba siendo testigo del inicio de su apostacía. Joel empieza a confrontar a Judá y les recuerda la historia de fenómenos físicos que habían vivido como la plaga de langostas que había destruido toda la vegetación y dejado los campos como desiertos.
\newpage
Esta plaga, también mencionada en Amós 4:9, fue una invasión masiva de langostas que provocó serios problemas económicos y Joel menciona que dicha plaga solamente era una advertencia de Dios ya que Él les podía traer mayores tragedias si se seguían apartando de Él. También hubo una sequía que provocó una gran falta de alimento y Joel les menciona que cada sufrimiento en la Tierra lo debían de considerar como un juicio por su pecado.
		\end{subsubsection}
		\begin{subsubsection}{El día del Señor: Armagedón (Joel 2-3)}
			Joel mira hacia el futuro, la destrucción de las langostas y la sequía era una muestra gráfica de lo que los hombres harían en Judá. Ahora Dios iba a utilizar un ejército extranjero para destruir a Judá y Joel los alienta a que rasguen sus corazones en arrepentimiento en lugar de rasgar sus vestiduras. \\
			Joel les da la promesa de bendición futura ya que Dios perdonaría a Su pueblo y también les profetiza que Dios iba a hacer grandes cosas. Comienza a narrar los hechos de los últimos días en Joel 2:28 acerca del Espíritu Santo, en Joel 2:30-32 acerca del Cristo y en Joel 3 se narra el juicio a las demás naciones.\\
		\end{subsubsection}
	\end{subsection}
\end{section}
%\end{document}


