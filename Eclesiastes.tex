%        File: Eclesiastes.tex
%     Created: Wed Oct 30 02:00 PM 2019 C
% Last Change: Wed Oct 30 02:00 PM 2019 C
%
%\documentclass[12pt]{article}
%\usepackage[margin=1.0in]{geometry}
%\usepackage[spanish]{babel}
%\usepackage{enumerate}

%\begin{document}
\begin{section}{Eclesiastés}
	\begin{itemize}
		\item Título\\
			En el idioma hebreo se le da un término que traducido significa asamblea. Designa un oficio o una función, el término en hebreo se refiere al encargado de dirigir dicha asamblea. La palabra eclesiastés viene del griego \textit{eclesia} que significa asamblea. Este vocablo equivale aproximadamente a un orador o predicador. El predicador es el título que se le puede dar al libro, \textit{qohelet} aparece en ocasiones con artículo por lo que concluimos que no se refiere a cualqier predicador sino a uno en particular.\\
			Es uno de los libros del megillot que es lo que se lee en las fiestas religiosas judías.
		\item Autor y fecha\\
			El autor era un sabio, posiblemente Salomón. Fue alguien que puso todo su empeño para buscar la verdad y encontrar las palabras adecuadas para comunicarla, fue un pensador profundamente original que no se contentó a aceptar ideas ya concebidas y como consecuencia de ello se le refiere un capítulo que lo destaca de entre los demás escritos de la Biblia.\\
			El nombre de Salomón no se menciona explícitamente pero el libro menciona que el predicador era hijo de David y rey de Israel. Salomón debió de haberlo escrito al final de su vida en el 935 a.C.\\
			Hay algunos que han argumentado que el hebreo usado por el autor parece ser lenguaje del siglo III a.C. además de que se nota la cultura helenística que coincidde con dicha época. No es indisénsable conocer la fecha de escritura tal y como lo estudiamos en el caso de Job.
		\item Tema principal\\
			Vanidad del hombre.\\
		El predicador repetirá esta frase hasta el cansancio, más de 30 veces de manera muy enfática para referirse a la vanidad más grande.\\
		Ello significa algo que es inutil, sin sentido, sin propósito, sin valor, que ha sido hecho sin causa, sin efectos, sin éxito.\\
		\item Propósito\\
			Enseñar que el que teme a Dios puede disfrutar de la vida.\\
			A pesar de la insignificancia aparente que muestra el libro acerca de la existencia del hombre, el hombre que teme a Dios puede gozar de su vida como un regalo de Dios. Son palabras que nos dan una orientación clara sobre la realidad de la vida. Esas palabras fueron además escritas por Dios, cosa que no debemos olvidar a pesar de la negatividad de las palabras del libro.\\
			Las conclusiones del predicador son que: lo más importante en la vida es el que le tengamos temor a Dios y que Dios va a juzgar todas nuestras obras. Independientemente de todas sus conclusiones hay que obedecer, creer y entender que Él va a castigar a los impíos y a los injustos, en verdad tiene sentido obedecer a Dios. El trabajo que hagamos para Él nunca será en vano. Tiene una expresión sumamente pesimista pues repite que nada tiene sentido en las acciones humanas.
			\newpage
			Sus mensajes siguen siendo relevantes en nuestras vidas ya que este sermón es un análisis de las experiencias de la vida. Salomón nos lleva en un viaje mental a través de toda su vida diciendo que todo lo intentó y concluye diciendo que todo solamente fue vanidad, inutil, irracional y vacío, lo dijo el hombre más sabio del mundo.\\
			A pesar de este inicial pesimismo es un libro de suma importancia que nos trae mucha enseñanza, se analizan los hechos desde el punto de vista humano y muestra cómo deberíamos de enteder la vida. Él obliga a sus lectores a mirar sin ilusiones la oscuridad y examinar los fundamentos de lo que creemos. Ofrece una buena oportunidad de que podamos crecer y madurar en nuestra fe.
	\end{itemize}
	\begin{subsection}{Bosquejo}
		\begin{subsubsection}{Debajo del sol (Eclesiastés 1-10)}
			La frase ``debajo del sol'' es usada 27 veces y se refiere a la vida del hombre razonando en forma estricamente natural.\\
			 En Proverbios 3:19 compara la vida de un hombre que vive de manera materialista, con el mismo propósito que los animales, no tiene ningún significado la vida para dichos hombres y por ello es que el hombre se frustra al no ver un propósito definido en su vida. Este libro de sabiduría nos habla de los errores cotidianos que cometemos todos los hombres.\\
			 Tener la expectativa de una felicidad invariante forzozamente termina en un desengaño. El predicador nos instruye a que hagamos lo humanamente posible para todo buen propósito y para toda buena obra.\\
			 El esfuerzo y la tristeza llenan al mundo cuando ven que no hay algo bueno que hacer en la corta vida del hombre. Este libro se está dirigiendo a personas que no creen en una eternidad y a este auditorio es con quienes el predicador hace unas consideraciones sobre su felicidad que está basada en sí mismos y en la naturaleza.\\
			 La vida tiene muchas cosas negativas, el hombre se siente constantemente desanimado pero más que un discurso del predicador ante una asambleas, parece ser un diálogo entre el autor y él mismo. Antepone realidades opuestas como la vida y la muertea, la sabiduría y la necedad, entre otras cosas. Lo que más se acentúa en dicho contraste es el aspecto negativo de la realidad sin negar la positividad de la vida.\\
			 La pregunta que más inquieta al predicador es acerca del sentido de la vida, ello se observa en Eclesiastés 1:3. Al término de sus tantos esfuerzos solamente concluye que todo es vanidad porque la obra que Dios realiza en el mundo es un misterio impenetrable y la sabiduría humana ofrece ayuda precaria para descubrir los misterios de la vida.\\
			 Eclesiastés ha querido decifrar el enigma y el sentido de las cosas de manera independiente de Dios apoyándose en su propia experienica y conocimiento. Esta actitud llevó a Salomón a distanciarse de lo que había escrito en Proverbios. En este pesismismo nunca llegó a tener fe en que habría una resurreción y vida eterna. Salomón lo tuvo todo, Dios le concedió sabiduría, poder y riqueza y al final escribió esto en donde muestra su pesimismo. Llegó a ser el hombre más sabio del mundo y con gran fama pero con todo ello Salomón nunca escuchó sus propio consejos.\\
			 Cuando escribió este libro casi al final de su vida hizo un recordatorio sobre lo que había vivido con la esperanza de que el lector lo entendiera. 
	\end{subsubsection}
	\newpage
	\begin{subsubsection}{Encima del sol (11-12)}
		Al final del libro Salomón exhorta a los jóvenes que desde temprana edad busquen a Dios, narrado en Eclesiastés 11:9. El predicador advierte al joven que puede disfrutar de su juventudd pero de manera limitada, con límites impuestos por Dios que debemos de respetar ya que la juventud también es vana y pronto pasa. En Eclesiastés 12:1 el predicador exhorta al joven a que desde temprana edad busque a Dios pues ahora es cuando tiene la plenitud física y mental para servir a su creador. Alienta a todos los hombres a temer a Dios y humillarse delante de Él. El fin del discurso es ``Teme a Dios y guarda Sus mandamientos''.\\
		Salomón hizo un enfoque muy sincero de la vida, todo ello está con el propósito de que la gente busque la verdadera felicidad que puede encontrar solamente en Dios. Todo lo temporal de la vida debe de verse a la luz de la eternidad, debemos escuchar estas advertencias que nos da el predicador para entender porque debemos comprometernos a honrar al creador para vivir nuestras vidas tal y como Él lo diseñó.\\
		El predicador trata de mirar la vida con realismo pues siempre hay problemas pero lejos está de que el predicar vea más allá de problemas pues ve hacia la recompensa que dará Dios sobre las cosas buenas que hagamos y las bendiciones que Dios da al que busca agradarle. El pesismismo inicial solamente es ilustrativo del hombre cuando no está con Dios pues no tiene a alguien, es su propia fortaleza ante los problemas de la vida. El sentido de la vida no está en perseguir las cosas materiales pues absolutamente todo es vanidad.\\
		Eclesiastés nos muestra que ciertos caminos nos llevan al vacío. Salomón nos enseña que el signficado de la vida no se encontró en las riquezas ni en la popularidad ni en el conocicmiento sino que en el saber que lo que hacemos es parte del propósito de Dios en nuestra vida. La vida es vacía y sin sentido al tener una vida despegada de Dios.\\
		A menos que el hombre conozca a su creador, nada de lo que haga le puede dar paz o felicidad. Dios está en todas las cosas y todas son para nuestro bien. El sabio acepta los buenos momentos y soporta los malos pues sabe que todo ello viene de Dios, que Él tiene la última palabra, Él no es tirano ni caprichoso y Él sabe lo que está haciendo. Al hombre que le agrada, Dios le da sabiduría pero al incrédulo solamente le da la capacidad de acumular sus bienes. La vida humana es la escuela de Dios pues enseña a los hombres a conducirse para cada circunstancia. Para el predicador, es Dios a través de Sus bendiciones que le enseña al hombre a ser bueno.\\
		\\
		La vanidad del hombre
		\begin{itemize}
			\item La sabiduría (Eclesiastés 2:14-16)
			\item El esfuerzo (Eclesiastés 2:18-23)
			\item Los logros (Eclesiastés 2:26)
			\item La vida (Eclesiastés 3:18-22)
			\item La rivalidad (Eclesiastés 4:4)
			\item El sacrificio egoísta (Eclesiastés 4:7-8)
			\item El poder (Eclesiastés 4:16)
			\item La avaricia (Eclesiastés 5:10)
			\item La riqueza (Eclesiastés 6:1-12)
			\item La religión (Eclesiastés 8:10-14)
		\end{itemize}
	\end{subsubsection}
\end{subsection}
\begin{subsection}{Temas específicos}
	\begin{subsubsection}{Sabiduría}
		La sabiduría humana no contiene todas las respuestas, es muy limitada. La Biblia es el libro donde podemos encontrar la sabiduría divina pues en la filosofía humana no se encontrarán soluciones precisas a este problema preciso. A fin de obtener la sabiduría de Dios primero debemos de llegar a conocerlo a través de Su palabra.
	\end{subsubsection}
	\begin{subsubsection}{Vanidad}
Salomón muestra cuan vacío es vivir para los placeres que esta vida nos ofrece, no tiene caso andar atrás de los placeres de la vida cuando nosotros podemos tener una relación con el Dios eterno, la búsqueda del placer a la larga nos va a desilusionar pues nunca nos sentimos saciados, no hay algo en este mundo que pueda satisfacer los anhelos más profundos de nuestros corazones.
	\end{subsubsection}
	\begin{subsubsection}{Búsqueda}
Descubrió que la vida sin Dios no tenía significado. Dios es quien nos dará la sabiduría y el gozo en esta vida.
	\end{subsubsection}
	\begin{subsubsection}{Trabajo}
Sin Dios no existe recompensa, por mucho que nos esforcemos el resultado siempre estará bajo la soberanía de Dios. El trabajo que se acepta como designio de Dios puede verse com regalo de Dios.
	\end{subsubsection}
	\begin{subsubsection}{Muerte}
Debemos de estar preparados pues después de la muerte vendrá el juicio de Dios.
	\end{subsubsection}
	\end{subsection}
\end{section}
%\end{document}


