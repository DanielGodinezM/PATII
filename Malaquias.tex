%        File: Malaquias.tex
%     Created: Wed Nov 20 06:00 PM 2019 C
% Last Change: Wed Nov 20 06:00 PM 2019 C
%
\documentclass[12pt]{article}
\usepackage{enumerate}
\usepackage[spanish]{babel}
\usepackage[margin=1.0in]{geometry}
\begin{document}
\begin{section}{Malaquías}
	\begin{enumerate}
		\item Título\\
			libro pos-exílico. El título el des daod por el peronsaje principal, fue el útlimo de los profetas enviado a los exiliados. La tradición dice que también fue miembro de la Gran Sinagoga que organizó Esdra
			libro pos-exílico. El título el des daod por el peronsaje principal, fue el útlimo de los profetas enviado a los exiliados. La tradición dice que también fue miembro de la Gran Sinagoga que organizó Esdrass
		\item Autor y fecha\\
			Dicen que el autro es anónimo porque Malaquías significa ``mensajero'' pero basándose en que ningínlibro de la Biblia se ha nombrado sin el nombre del autor se considera queél en realidad fue el autor. Asemjeja que el profeta discurte con Dios en preguntas y respuestas. Nehemías, después de la labro que hizo en Jerusalén, tuvo que regresar a la corte del rey porque solamente habia pedido un permios y regrsó a Jerusalén por segunda oacasión y encontró que el pueblo se estaba desviando y es entonces cuando malaquías proclama el mensjae de Dios reprendiednolos por los pecados que estaban haciendo que habían sido la razón del exilio, destrucción del telmplo y de jresuaen.
		\item Tema\\
			Constancia en la represnión y la restauración. La comnidad estaba marcada or el materialismo y el baandono de la fe y cisnismoen su manera de comportares y 3:16-18 nombra  a un remanente que temee aJehova qa quiene dIDo sreocnoce como s verdadero pueblo y los nombra su epecual teosoro una vez más. Est re remnanente permananecion fiere y llebvana araca de justiica y servicio que siempr etuvierpn ahacia el Sñeor. malaquías se une  ala tracidi+on de sus antencesores y lo definde que el gran dían e que Jehová salvará de una vez por todas a su puelbo.
		\item Propósito\\
			Llamdao final al arrepentimietno a la fideloidad, termina con una exhortación y proesa. termina con lo que emepzó el At, la lyed de Moises creano una unión teológica q, tod el AT estaa alomentada por la isntrucción que dio Dios en Sinaí. termina donde el mensjae del vangelio del NT con la llegadae de un profeta uq eprepararía el camino del mesias. Depspués de Malaquías no voli+o a haber ningún profeta hasta el NT, fue instrucción para una comunidad de transición, no habrían más mensajes auqnue sabemos que se escribieron mensaje sap+ocirbos sobre peronsa que noera profetas de Dios isno quer ean falsos maestros que scribiern en esos 400 años de silencio entre la terminación del mninsterio de malaquías y el injicoio de los NT.
	\end{enumerate}
	\begin{subsection}{Bosquejo}
		\begin{subsubsection}{Amor de Dios (Malaquías 1:1-1:5)}
			Dios les die que él los ha amado, les recorvdaba que siempre había sido ifle ,os había preservado y escodgiod, los había amdo siemproe pero el opueblo en sus c ceguera creada pr sus momentos de duad y siufrimiento no puede ver las maneras en que Dios les muestra su amor. Al pueblo le moltesa que DIos les crecuerde su aor en el pasado pues considerad  que ene realidad no los habían amado. Además de haber odiado a Esau aborreció su descendeencia por lo que despareció.
		\end{subsubsection}
		\begin{subsubsection}{Reprensión de Dios (Malaquías 1:6-3:15)}
			Se centra en una reprensión, paara ese momento Nehemías a habpia regresado y había visto que se habían desviado.la represnisón va prinicalmente idirigado a los saerdotes pues ellos son los culpadbles qde qe el resto del pueblo se desviara, vs 6. Dios les recrimina por aher llevado ofrendas defectuososas y que dejaran a sus epsosas uniendose con mujeres de pueblos pagasnos vs 13-14. Les hace la recriminación de que le estaban robando al no dar sus ofrendas ni diezmos. El oeublo deaba de tenre reverencia ahacias dios y le dejaban de temerle. El puelbo rechaza a DIos porque sus bendiciones no coindciden con su concepto materialista de bendicones .\\
			A pesar de que concocían la fidelidad de Dios se quejaban qdde que no tenía sentido seguirlña porque no veían recomplesan, pero a pesar de todo Dios iba a cumplir los pactos que había hehco con sus antepasados, qque se cunmplirían con el remanente.
		\end{subsubsection}
		\begin{subsubsection}{Promesa de Dios (Malaquías 3:16-4:6)}
			Malaquías les recuerda que Dios iempre tendrá un remanetne que le será fiel que sería librado de la ira de Dios cuando llegara a ajuzgar a los soberbios y ellos será a los que Dios le streraá slavación y ondra a los malvados a sus pies. \\
			En 4:1 Malaquías mencjoina el dáde Jehová. Malaquías termina donde el mensaje del evangelo empieza,con la llegada de quien prepararía al pueblo para recibir al Mesías. Son también un enlace con los libros edel NT que serán ecscritoos aprox 400 años desués de Malaquías. Dios promete que enviaría a otro profeta antes del día de Jehová, 4:5.\\
			El regreso de elías es controversial ya que él no murió físicamente. Venrdía uno con el espíritu de Elías así cmo Eliseo, uno que aparecería 400 después que fue Juan el Bautista, un mesnaje delante del mesías con espiritu y poder de Lias Lucas 1:17. Al final del último vs significa que la humanidad tendrá que arrepentirse si quieren evitar destrucción como la que sfrió el pueblo judío debido a su pecado.\\
			Se cierra la escritura del AT con estas palabaras, lo siguiente sería anunciar la vendia del Mesías prometida, Dios no velvería a hbaar en 400 años hatas eque el Mesías naciera por medode  una virgen en carne como de un hombre común y corriente.
		\end{subsubsection}
	\end{subsection}
\end{section}
\end{document}


