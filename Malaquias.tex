%        File: Malaquias.tex
%     Created: Wed Nov 20 06:00 PM 2019 C
% Last Change: Wed Nov 20 06:00 PM 2019 C
%
%\documentclass[12pt]{article}
%\usepackage{enumerate}
%\usepackage[spanish]{babel}
%\usepackage[margin=1.0in]{geometry}
%\begin{document}
\begin{section}{Malaquías}
	\begin{enumerate}
		\item Título\\
			Malaquías es el nombre del profeta del cual trata el libro post-exílico. El título del libro le es dado debido al personaje principal, fue el último de los profetas enviado a los exiliados. La tradición dice que también fue miembro de la Gran Sinagoga que organizó Esdras.
		\item Autor y fecha\\
			Se dice que el autor es anónimo porque Malaquías significa ``mensajero'' pero basándose en que ningún libro de la Biblia se ha nombrado sin el nombre del autor se considera que él en realidad fue el autor. Asemeja que el profeta discute con Dios en preguntas y respuestas. Nehemías, después de la labor que hizo en Jerusalén tuvo que regresar a la corte del rey porque solamente había pedido un permiso y regresó a Jerusalén por segunda ocasión a encontrar que el pueblo se estaba desviando y es entonces cuando Malaquías proclama el mensjae de Dios reprendiéndolos por los pecados que estaban haciendo y que había sido precisamente la razón del exilioy la destrucción del templo y de Jerusalen.
		\item Tema\\
			Constancia en la reprensión y la restauración. La comunidad estaba marcada por el materialismo y el abandono de la fe y cinismo en su manera de comportarse y en Malaquías 3:16-18 nombra a un remanente que teme a Jehová a quienes Dios reconoce como Su verdadero pueblo y los nombra su especial tesoro una vez más. Este remanente permanecería fiel y llevarían la justicia y el servicio que siempre tuvieron hacia el Señor. Malaquías se une a la tradición de sus antecesores y define que el gran día es cuando Jehová salve de una vez por todas a Su puelbo.
		\item Propósito\\
			Llamado final al arrepentimietno y a la fidelidad. Termina con una exhortación y una promesa, con lo que empezó el Antiguo Testamento que fue la ley de Moisés, creando una unión teológica de todo el Antiguo Testamento. Termina, también, con el mensaje del evangelio del Nuevo Testamento con la llegada de un profeta que prepararía el camino del Mesías. Después de Malaquías no volvió a haber algún profeta hasta el Nuevo Testamento.\\
			Fue un libro que sirvió como instrucción para una comunidad de transición, no habrían más mensajes proféticos aunque sabemos que se escribieron mensajes apócrifos sobre personas que no eran profetas de Dios sino que eran falsos maestros que escribieron en esos 400 años de silencio entre la terminación del ministerio de Malaquías y el inicio de los mensajes del Nuevo Testamento.
	\end{enumerate}
	\begin{subsection}{Bosquejo}
		\begin{subsubsection}{Amor de Dios (Malaquías 1:1-1:5)}
			Dios les dice que Él los ha amado, les recordaba que siempre había sido fiel, los había preservado y escogido, los había amado siempre pero el opueblo en su ceguera creada por sus momentos de duda y sufrimiento no puede ver las maneras en que Dios les muestra su amor. Al pueblo le molesta que Dios les recuerde su amor en el pasado pues consideraban que en realidad no los había amado. Además de haber odiado a Esaú aborreció su descendencia por lo que desapareció.
		\end{subsubsection}
		\begin{subsubsection}{Reprensión de Dios (Malaquías 1:6-3:15)}
			Para ese momento Nehemías ya había regresado y había visto que se habían desviado. La reprensión va principalmente dirigida a los sacerdotes pues ellos son los culpadbles de que el resto del pueblo se desviara, Malaquías 1:6. Dios los recrimina por haber llevado ofrendas defectuosas y que dejaran a sus esposas uniéndose con mujeres de pueblos paganos, Malaquías 2:13-14. Les hace la recriminación de que le estaban robando al no dar sus ofrendas ni diezmos, el pueblo debía de tener reverencia hacia Dios pero le dejaban de temerle.\\
			El puelbo rechaza a Dios porque sus bendiciones no coinciden con su concepto materialista de bendiciones.\\
			A pesar de que conocían la fidelidad de Dios se quejaban de que no tenía sentido seguirlo porque no veían recompensa pero a pesar de todo Dios iba a cumplir los pactos que había hecho con sus antepasados que se verían cumplidos con el remanente.
		\end{subsubsection}
		\begin{subsubsection}{Promesa de Dios (Malaquías 3:16-4:6)}
			Malaquías les recuerda que Dios siempre tendrá un remanente que le será fiel, que sería librado de la ira de Dios cuando llegara a juzgar a los soberbios y ellos serán a los que Dios les traerá salvación y pondrá a los malvados a sus pies.\\
			En Malaquías 4:1 Malaquías menciona el día de Jehová. Malaquías termina donde el mensaje del evangelo empieza, con la llegada de quien prepararía al pueblo para recibir al Mesías. Es también un enlace con los libros del Nuevo Testamento que serían escritos aproximadamente 400 años después de Malaquías. Dios promete que enviaría a otro profeta antes del día de Jehová, Malaquías 4:5.\\
			El regreso de Elías es controversial ya que él no murió físicamente. Vendría uno con el espíritu de Elías así como Eliseo, uno que aparecería 400años después que fue Juan el Bautista, Lucas 1:17. El final del último se refiere a que la humanidad tendrá que arrepentirse si quieren evitar destrucción como la que sufrió el pueblo judío debido a su pecado.\\
			Se cierra la escritura del Antiguo Testamento con estas palabaras, lo siguiente sería anunciar la venida del Mesías prometido, Dios no volvería a hablar en 400 años hasta que el Mesías naciera por medio de una virgen, en carne como de un hombre común y corriente.
		\end{subsubsection}
	\end{subsection}
\end{section}
%\end{document}


