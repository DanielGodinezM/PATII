%        File: Salmos.tex
%     Created: Wed Oct 23 04:00 PM 2019 C
% Last Change: Wed Oct 23 04:00 PM 2019 C
%
%\documentclass[12pt]{article}
%\usepackage{enumerate}
%\usepackage[spanish]{babel}
%\usepackage[margin=1.0in]{geometry}
%\begin{document}
\begin{section}{Salmos}
	\begin{enumerate}
		\item Título\\
			Corresponde a la traducción del título en hebreo que significa ``canto''. La palabra ''salmos`` proviene del griego.\\
			       También significa en el griego básicamente ''rascar``, en mención de las cuerdas de un instrumento musical. Los salmos se escribireron para cantarse con música, se utilizaban como cantos de adoración en los cultos que tenían en el templo y posteriormente, después del exilio, en las sinagogas.\\
			En el libro de Esdras y de Nehemías se mencionan los avivamientos del pueblo, que fue cuando se le dio el orden a la colección para que fuera un libro de oraciones y cantos. \\
			En los salmos hay párrafos que se repiten. Por ejemplo: el salmo 40 se repite en el salmo 53, el salmo 70 está contenio en el salmo 40, entre otros. Por ello es que hay algunas traducciones que disminuyen el número de salmos. \\
			Este libro puede considerarse como un santuario literario de Dios que se usa como enseñana para guiar a Su pueblo en oración. Además, es la parte del Antiguo Testamento que más citan los autores del Nuevo Testamento ya que en todos ellos se puede encontrar la expresión sentimental de los autores ante todo tipo de situación.. 
		\item Autor y fecha\\
			A lo largo del libro se ve de manera más explícita la inspiración que tuvo el Espíritu Santo sobre los varios autores de los salmos. Muchos de ellos fueron escritos por David, algunos fueron escritos por diversos levitas y algunos tienen autor anónimo. \\
			En Números 26:11 se narra que los hijos de Coré no murieron junto con él y por ello continuó su descendencia. Esto es importante ya que a sus descendientes de Coré se les atribuyen 12 salmos.\\
			Se escribieron en aproximadamente 1000 años, desde que salieron de Egipto hasta los últimos tiempos después del exilio.
		\item Tema principal\\
			La adoración a Dios.\\
			Dios es adorado por Su poder, por Su majestad y Su santidad. Los hombres maravillados por los atributos de Dios buscan una comunión íntima con ese Dios que personalmente los había escogido. \\
			Es necesario que haya una relación de amor mutuo entre el creador y su criatura, los salmos forman la colección más antigua de himnos y oraciones que tiene el pueblo de Dios. Podemos aprender como alabar a Dios y cómo es permitido que tengamos acceso a Él en oración, en el libro viene cómo debemos de orar y pedir.
		\item Propósito\\
			Tener una comunión con Dios, la prioridad que todo hijo de Dios debe de tener.\\
			Es una necesidad para todo adorador que brote la súplica que debe de salir desde del fondo del corazón. La persona que ora motiva su petición como para convencer a Dios aunque no es exactamente así, debemos de orar con humildad aceptando lo que Dios haya determinado darnos. La oración es un acto de confianza basado en sus hazañas que ha hecho en el pasado.
			\newpage
			En el culto, un creyente experimenta el poder y la enseñanza de Dios. La adoración colectiva es parte vital de nuestra parte espiritual y no solamente es un requisito sino que es un deleite, una necesidad espiritual.
	\end{enumerate}
	\begin{subsection}{Bosquejo}
		Es un tesoro de lectura y meditación, igualmente apreciado por los cristianos y los judíos. Más de una cuarta parte de las citas del Antiguo Testamento en el Nuevo Testamento son precisamente de los salmos pues se compusieron para que su uso fuera repetido.\\
		Tienen una multitud de aplicaciones ya que son utilizados por individuos diferentes en situaciones diferentes. La divisón en 5 libros fue hecha por la tradición judía tratando de hacer un paralelismo con cada uno de los libros del Pentateuco y al final de cada uno de estos libros se termina con una doxología, es decir, una alabanza a Dios.
		\begin{subsubsection}{Libro I (Salmos 1-41)}
			En estos primeros 41 salmos predomina la autoría de David y su tema principal es el hombre. Ello se puede leer en Salmos 1, Salmos 8:4, Salmos 14:2, Salmos 21:10, Salmos 34:8, Salmos 37:24, entre otros.\\
			En Salmo 41:13 viene la doxología del libro. Dentro de las citas en el Nuevo Testamento que se hacen acerca del libro I se encuentra que Salmo 41:9 es citado en Juan 13:18. El paralelismo que se hace de este libro con el Pentateuco es con el libro de Génesis ya que el tema predominante de Génesis es la creación de los seres humanos.
		\end{subsubsection}
		\begin{subsubsection}{Libro II (Salmos 42-72)}
			Los salmos en este libro fueron escritos principalmente por David aunque también hay salmos que fueron escritos por los descendientes de Coré. El tema principal del libro es la redención, ello se puede leer en Salmos 44:26, Salmos 50:15, Salmos 51:10, Salmos 71:3, entre otros.\\
			La doxología del libro se encuentra en Salmos 72:18-19. El libro que habla sobre la redención en el Pentateuco es Éxodo pues en él se habla de la redención del pueblo de Israel cuando salió de la esclavitud de Egipto.
		\end{subsubsection}
		\begin{subsubsection}{Libro III (Salmos 73-89)}
			Escritos principalmente por asaf. Asaf sistutyó a Uza, quien murió por tocas el arca. El tema es la alabaza y la adoración. Eg 73:25, 77:2, 73:13, 84:4, 86:5, 86:26. La doxología se encuentra en 89:52. El libro del pentateuco con el que guarda paralelismo es Levítico que habla sobre la adoración a dios en el tabernáculo.
		\end{subsubsection}
		\begin{subsubsection}{Libro IV (Salmos 90-106)}
			La mayoría de los autores en este libro son anónimos y hablan de la Tierra como el tema principal. Esto se puede leer en Salmos 96:9, Salmos 96:12, Salmos 97:9, Salmos 100:1, Salmos 102:25. La doxología se encuentra en Salmos 106:48.\\
			El libro con el que guarda paralelismo en el Pentateuco es Números ya que habla de la Tierra y la tierra prometida.
		\end{subsubsection}
		\begin{subsubsection}{Libro V (Salmos 107-150)}
			Los autores principales son David y algunos anónimos con la Escritura como el tema principal, esto se puede leer en  Salmos 107:20, Salmos 112:1, Salmos 119:9, Salmos 119:12, Salmos 119:36, Salmos 119:43, Salmos 119:105, entre otros. Se alaba a Dios por todas las obras escritas en la historia del pueblo de Dios y por Su grandeza. Con ese salmo entramos a la salvación eterna ya que  Apocalipsis 19:1-10 menciona los cánticos celestiales. La doxología de este libro es todo el salmo 150 y el libro del Pentateuco con el que guarda paralelismo es Deuteronomio ya que en el Deuteronomio 5 se repite toda la ley a la generación nueva que iba a entrar a conquistar la tierra prometida.
		\end{subsubsection}
	\end{subsection}
	\begin{subsection}{Géneros de los Salmos}
		Los salmos son la descripción de la manera en que el hombre le responde a Dios por lo que la respuesta del escritor es de asombro y de temor inclusive. Dios es el mismo en todos los salmos pero las situaciones por las que se le alaba o se le pide son distintas.\\
		Hay que reconocer que están representando la realidad del corazón humano en diversas situaciones, alegría, prueba, entre otras. Pero aún en estos salmos de lamentación el salmista siempre tiene la confianza de que Dios ha de acudir a su ayuda. Podemos expresarle a Dios todos nuestros sentimientos por más negativos que sean pero estemos seguros de que nuestras oraciones son oídas por Él, debemos de esperar su respuesta. Hablan también mucho de la persona y la obra de Jesús ya que se encuentran muchas profecía mesiánicas.
		\begin{subsubsection}{Salmos de Alabanza General}
			Tanto la alabanza personal como la colectiva se aprecia en la gran mayoría de los salmos, en estos se exaltan las maravillas de Dios, Su bondad y Su misericordia. Normalmente el salmista hace llamados para que los cercanos a él se unan a esta alabanza. Podemos ver también que el salmista normalmente se humilla y acepta que todo depende de Dios y expresa una alegría de que hay alguien que es mayor a cualquier hombre y que además somos objetos de amor y misericordia. Se incluyen los salmos de acciones de gracias y algunos salmos cuentan con la intervención de Dios para salvar al salmista de algún peligro los cuales son los salmos de alabanza específica.\\
			Entre ello se encuentran varios salmos como por ejemplo Salmos 24, Salmos 29, Salmos 33, entre otros.
		\end{subsubsection}
		\begin{subsubsection}{Salmos de Clamor}
			Más de un tercio de los salmos son de lamentos, quejas para Dios por alguna enfermedad o amenaza de sus enemeigos. Incluyen normalmente petición para protección, guía, intervención a su favor, reconciliarse con Él, sanidad, larga vida, entre otras cosas. El clamor es algo que tenemos todos los días cuando le pedimos a Dios, no hay algo que podamos hacer pues sólo la intervención de Dios puede camibar nuestras situaciones.
			\newpage
			Nos damos cuenta que solamente clamando a Dios es que podríamos obtener algo a nuetsro favor. También incluye elogios a Dios aunque no es el objeto principal de estos salmos.\\
			Las oraciones de nosotros, como las de los salmistas, tienen que ver mucho los pormenores de la vida cotidiana. Por eso, siempre los salmos de clamor se parecen a las oraciones que nosotros hacemos.\\
			En la Biblia se habla de personas desolada que muchas veces sienten que fueron abandonados por la sociedad y por Dios, se sienten perseguidos, esperando en la gracia de Dios que puedan salir de su problema siendo consoladas. La gracia de Dios la podemos entender cuando las penas son alviadas por Dios.\\
			Independientemente de que le clamemos, Dios sabe qué es lo que está pasando en nuestra vida pero le agrada que le clamemos a Él pidiendo ayuda. No podemos esconder nuestra culpa cuando pasamos por tribulación, Dios lo sabe pero es necesario que antes de pedir confesemos nuestro pecado. El primer paso para ibrarse de la angustia es sincerarse con Dios.
		\end{subsubsection}
		\begin{subsubsection}{Salmos de Confianza}
			Es otra forma de adorar a Dios pues Él desea que externemos lo que sentimos para con Él. Hay salmos completos que expresan una fe abbsoluta del salmista a Dios, algunos exhortan en que solamente podemos esperar en Él y no en alguien más. Por lo regular incluyen una confesión de la certidumbre de la misericordia que Dios tiene para con sus hijos. Normalmente también hacemos una exhortación para que también los demás confíen en Dios que es el único que nos puede sacar de las tribulaciones. Ello se ve, por ejemplo, en Salmos 23, Salmos 42, entre otros.\\
			En estos salmos, el salmista relata las virtudes de Dios. En Salmos 23 vemos la confianza que tiene el salmista en Dios y muestra las razones por las cuales tiene esa confianza. En Salmos 23:4 David confía en la protección de Dios, mostrando que cuando nos hayamos en peligro es cuando más buscamos a Dios. Los incrédulos confían en su inteligencia, fuerza y en su obra creyendo que no necesitan de algún Dios pero los que ya nos hemos humillado ante Dios sabemos que no podemos tener confianza ante algún otro ser mayor pues siempre Él nos ha preservado, sabemos que estando en Él estamos a salvo de todo mal. Los hombres naturales no quieren ser afligidos, su confianza está en las cosas del mundo y sus propios atributos pero nosotros sabemos que estamos desprovistos de cualquier poder para sustentarnos a nosotros mismos y por ello le pedimos a Dios, en humildad, que muestre su poder.\\
		\end{subsubsection}
		\begin{subsubsection}{Salmos de Adoración}
			Estos salmos alaban a Dios por lo que Él es, no por lo que hace ni por mi confianza ni clamor. El objetivo es exaltar la persona de Dios reconociendo quien es Él.
			\newpage
		\end{subsubsection}
		\begin{subsubsection}{Salmos Sapienciales}
			El término ''sapiencial`` significa literalmente ''algo relacionado con la sabiduría``, estos salmos son aquellos cuyo tema principal es la sabiduría y fueron escritos para responder las preguntas que se hacían los hijos de Dios. La sabiduría consiste en llevar una vida con prosperidad, buena reputación y la felicidad que Dios quiere para nosotros pero debemos de aclarar que no es sabio quien ha estudiado la Biblia y sabe lo que tiene que hacer sino que es sabio aquel que en realidad lo lleva a acabo. La persona que lee estos salmos recibe una instrucción como, por ejemplo, Salmos 1. El propósito principal de este salmo es comunicar que el justo será bienaventurado. En él podemos ver un apralelsimo interesante como poesía hebrea acerca de lo que ha hecho el bienaventurado y el consejo de malos consiste en decir que la prosperidad se consigue por medio de caminos ilícitos. Estar en camino de pecadores significa más o menos lo mismo pero en el paralelismo hay una progresión pues sugiere permanencia más que andar. La secuencia continúa diciendo, ''ni se ha sentado`` comunicando ese mismo estado de permanencia en su conducta.\\
			En Salmos 1:2 habla del por qué es bienaventurado, es la sabiduría de Dios que nos enseña en el salmo lo que tenemos que hacer y de las consecuencias en caso de que las hagamos.
		\end{subsubsection}
		\begin{subsubsection}{Salmos de la Ley}
			Estos salmos exaltan la ley de Dios, anuncian que la obediencia a los mandamientos es necesario para alcanzar el éxito, sin embargo, su tema no es el éxito sino la ley misma. Magnifican el exaltar la ley que nos ha dejado, esto lo vemos en Salmos 119, Salmos 19, entre otros.\\
			En particular Salmos 119 está dividido en 22 partes de manera acróstica. Dentro de cada parte se comienza con una letra distinta de forma que aparecen las 22 letras hebreas en orden. De 176 versículos en dicho capítulo, en 171 se menciona la Palabra de Dios así como sus sinónimos.\\
			A pesar de este énfasis en la ley, el salmo no adora a la ley sino que recuerda los mandamientos de Jehová y al exaltar Su Palabra está adorando a Dios. Estos versículos recalcan que la Palabra de Dios debe de ser obedecida pues se menciona que es fiel, recta, pura, la verdad y que podemos confiar en sus estatutos con el fin de guiarnos pues son justos.\\
			Solamente a través de las Escrituras Dios nos enseña cuál es el camino que debemos de seguir.
		\end{subsubsection}
		\begin{subsubsection}{Salmos Mesiánicos}
			Estos salmos hablan sobre la existencia de Jesús y lo que habría de ser cumplido por Él. Esto se puede leer, por ejemplo, en Salmos 89, Salmos 110, Salmos 132. Se presenta en ellos la gloria y la universalidad del Mesías y de su reino. Cuando leemos estos salmos parece que no hablan específicamente del Mesías sino de algún persoanje en la época. entonces jno podría fácilmente preguntarse por qué se aplican en el Nuevo Testamento estos pasajes a la vida de jesús. Estos salmos tienen 3 tipos de aplicaaciones, se pueden aplicar en el momento en el que estsaban ocurriendo, se pueden aplicar a Jesús o a nosotros cuando pasemos por algo similar.
			\newpage
			Por lo regular estos salmos narran de manera exagerada lo que le podría pasar al autor pero cuando lo aplicamos a Jesús es de manera literal. Esto lo vemos, por ejemplo, en Salmos 110 pues  Jesús aseguró que David había escrito acerca de Él y no se refiriera a alguna otra persona.\\
			Algunos de estos salmos dan a conocer que Cristo reina en el cielo desde su ascención, cosa que ningún hombre puede hacer, otros hablan de Su reino en el milenio.\\
			Como ejemplo, en Hebreos 4-10 habla acerca del sacerdocio de Cristo y se fundamenta en Salmos 110 pues profetiza que el Mesías sería tanto rey como sacerdote. En Salmos 22 vemos varios pasajes que en el Nuebo Testamento se aplican a Jesús, por ejemplo, Jesús mismo citó Salmos 22:1 en Su crucifixión.\\
			Los 4 evangelios informan además que los guradias echaron suertes sobre la ropa de Jesús, aspecto que se menciona de manera explícita en Salmos 22, se entiende claramente que es dedidaco al Mesías pues aplicado a David sería una exageración mientras que Jesús lo cumplió tal y como lo menciona. David se sentía abandonado por Dios y, como sabemos, Jesús vivió por una condición semejante pues el Padre le da la espalda en el momento en el que carga con nuestros pecados.\\
			Habla también de la desarticulación de sus huesos, de un paro cardiaco, de la sed, la perforación de sus manos y pies, humillación pública, repartición de su ropa y la muerte. Todo ello se escribió siglos antes de que incluso se declarara la crucifixión como castigo romano.\\
			Hay una alabanza profetizada en Salmos 22:27-31, Dios es adorado en todo el mundo y sabemos que algún día cuando regresé será dorado por todo el mundo. Algunos de estos salmos sí hablan del Mesías en su contexto original mientras que algunos hablan de otra persona durante la época en la que se escribió.  
		\end{subsubsection}
		\begin{subsubsection}{Salmos Acrósticos}
			Hay diversas foramas de escriir acrósticos pero en la poesía hebrea se utiliza el ac,E.g. 25, 34,145. Los almos fueron escritos para que se aprendieran de memoria, éstos fueron escritos para leerse y así aprendérselos.
		\end{subsubsection}
		\begin{subsubsection}{Salmos de Retribución}
			La cultura judía tenía un criterio muy especial en cuanto al carácter de Dios a pesar de que Él lo ha revelado claramente en Su Escritura. Algunos salmistas consideran el principio de la retribución, afirman que el justo siempre prospera y que el malvado sufre pero sabemos que no es así ya que se establecieron maldiciones y bendiciones dependiendo si obedecían o no su ley. Ésto lo consideraban como una retribución, es decir, que forzosamente tenía que pasar así. Ahora sabemos que podemos obtener las bendiciones de Dios pero sabemos que la verdadera bendición está en la vida eterna.\\
			Este concepto no era muy claro para el creyente del Antiguo Testamento, no pensaban en la eternidad pues pensaban que su bendición tenía que ser en la Tierra. Pensaban que las recompensas por sus acciones y el castigo de los malvados debía de ser en esta vida. Debido a lo estudiado previamente sabemos que precisamente ésta era la forma de pensar de los amigos de Job.
		\end{subsubsection}
		\newpage
		\begin{subsubsection}{Salmos Imprecatorios}
			La palabra ''imprecatorio`` significa ''proferir palabras con el deseo de que alguien sufra un mal``. Este tipo de salmos los encontramos, por ejemplo, en Salmos 55:15, Salmos 137:9, Salmos 109:8-9, entre otros.\\
			Para entender e interpretar estos pasajes debemos de tomar en cuenta la ira de Dios sobre el pecado que constantemente se nombra en la Escritura, aún en el Nuevo Testamento hay palabras imprecatorias como, por ejemplo, Lucas 19:27, Mateo 13:50, Mateo 23:33, entre otros.\\
			El Nuevo Testamento especifica que lo que el hombre sembrare, eso segará. Cuando escribió esto el salmista vivía bajo la ley de retribución, eran oraciones que hacía para que se cumpliera la Palabra de Dios y su petición se basaba en las promesas de la protección de Dios. Los salmistas no estaban buscando una vengaza cruel y personal, ellos se abstenían de vengarse y le entregaban su causa a Dios. Los justos deben de desear siempre el castigo del mal.\\
			Podemos aprender del salmista el celo que tenía por darle la gloria a Dios, debemos de usar estas imprecaciones a la luz de lo que Jesús nos ha dejado en el Nuevo Testamento. Debemos atacar con nuestra  fuerza el pecado y la corrupción sin olvidar que debemos de orar por la salvación de las personas. Debemos de apreciar que el lenguaje de estas personas se adapta también a la lucha espiritual pues el cristiano puede desear la destrucción de estos espíritus.\\
			David rogó a Dios que le salvara de sus enemigos y que los catigara, él solamente exponía el problema que tenía, pide que sus enemgos sean avergonzados, afligidos y destruidos, ruega que sean públicamente humillados y que el Ángel de Jehová los arrase.\\
			El salmista desea para sus enemigos la destrucción repentina y completa, que Jehová los acabe. Debemos de verlos también como expresiones de confianza en la retribución que recibirían los malvados. Ahora podemos nosotros perdonar a nuestros enemigos porque Dios ya nos perdonó, ya no es necesario que pidamos que se aplique la justicia de Dios pues Su justicia ya fue aplicada. Ahora el Espíritu Santo nos impulsa a orar por el perdón de esas personas. Cristo, cuando lo ofendieron, no lanzó ninguna imprecación contra esas personas y ahora sabemos que cualquier injusticia sobre nosotros será condenada mientras que el justo será recompensado. La autoridad está puesta por Dios para castigar al impío.
		\end{subsubsection}
		\begin{subsubsection}{Salmos Penitenciales}
			Al comienzo del siglo V, Agustín de Hipona otros 4 salmos a la lista de salmos penitenciales ya que anteriormente el único salmo que se consideraba penitencial era el 51. En Salmos 51 vemos que David confiesa su pecado y ora por que el Señor le perdone. Actualmente se consideran 7 salmos de confesión: Salmos 6, Salmos 32, Salmos 38, Salmos 51, Salmos 102, Salmos 130 y Salmos 143.\\
			Estos salmos fueron escritos por hombres que verdaderamente tenían arrepentimiento en su corazón y suplicaban perdón. En el momento en el que caemos en algún pecado, nos dolerá y lo confesaremos de forma abatida pidiéndole que nos perdone y que no nos eche de Su gracia.
		\end{subsubsection}
	\end{subsection}
	\newpage
	\begin{subsection}{Teología en los Salmos}
		\begin{subsubsection}{El centro es Dios}
			Aunque se hable mucho acerca del ser humano en los salmos, Dios sigue siendo el centro en todo momento. En ellos no se le alaba al hombre, simplemente se le ve sujeto a la voluntad de su creador.\\
			Uno de los personajes más importantes en la historia del pueblo de Israel es el rey David, que además, también fue uno de los escritores más detacados de los salmos y a pesar de ello en ninguno de los salmos se le alaba a David ni a ningún otro hombre a pesar de sus hazañas militares ni por otros de sus triunfos, siempre quien recibe la gloria y la honra es Dios. Se le alaba por Su poder, por quien es, por Su creación, por Su providencia, por Su ampor, por Su misericordia, por el acto salvífico y por Su señorío sobre las naciones y lo creado. Los salmos llevan al conocimiento más completo y profundo de Dios, a través de ellos tenemos una comunión más íntima con él. \\
			Si quere conocer a Dios observe la manera en que los autores se expresan acerca de Dios y sus atributos, en ellos se puede encontrar esperanza.
		\end{subsubsection}
		\begin{subsubsection}{La antropología}
			Nos dan una buena visión sobre qué y cómo somos los seres humanos, en los salmos se encuentran todas las emociones que puede tener el hombre. Describen al hombre cuando es feliz, bendecido, las pruebas que tiene y la confianza que debe de tener en Dios. Los salmos hablan del juicio de Dios en contra de las personas que no lo honran y no lo conocen, son un estudio de los creyentes y de los incrédulos. \\
			La lectura constante de los salmos nutre nuestra mente y nuestro corazón, la persona que los lee hace que se da cuenta de la grandeza de Dios y de su propia insignificancia, de esa manera podemos temer a Dios y dicho temor nos alejará del pecado, repercutiendo en una vida en santidad que es lo que Dios pide.\\
			La lectura de los salmos provoca que nuestro testimonio y nuestra vida le dé verdadera gloria y honra a Dios.
		\end{subsubsection}
		\begin{subsubsection}{La oración}
			El libro también se puede considerar como un manual de oración pues los salmos enseñan lo que es orar y cómo lo debemos de hacer. Toda actitud y tipo de oración la encontramos en este libro, no dejan la oración en un estado académico ni intelectual, la viven con cálido amor para con Dios así como muestran su sumisión ante la voluntad de Dios. \\
			Los salmistas reconocen su pecado y cuando los leemos reconocemos nuestra debilidad y necesidad de depender de Dios. Hablan acerca de la ética y los madamientos de Dios, siempre con una obediencia sumisa ante el creador.\\
			Los estados de ánimo de los escritores fueron iguales que los que nosotros podemos experimentar en alguna situación semejante y en ellos vemos cómo podemos dirigirnos a Dios, apelan al pacto y a las promesas que Dios ha hecho.\\
			Durante siglos se ha encontrado en ellos inspiración para dirigirnos ante Dios, en ellos vemos inmediatamente esas palabras del fondo de nuestro corazón para dirigirnos ante Él.
			\newpage
			Esas oraciones muestran cómo podemos tener una actitud de súplica y cómo adorar, bendecirlo y ser bendecidos. Los salmos nos muestran cómo orar por nuesra iglesia, por la obra que Él esté haciendo, por un enfermo, entre otras cosas. Ésto hará que seamos fervientes lectores de los salmos.\\
			Los escritores nunca dudaron que Dios los estaba escuchando, contrario al caso de Job que aparentemente Dios no escuchaba a Job. En el Nuevo Testamento se explica el resultado de hacer oraciones con salmos en  1 Juan 5:14-15\\
			Los salmistas pedían de acuerdo a la voluntad de Dios y no de manera egoísta.
		\end{subsubsection}
		\begin{subsubsection}{El pecado}
			La pecaminosidad del hombre es documentada en el libro de los salmos. Los creyentes también pecamos y tenemos tropiezos periódicos, esto lo podemos ver en Salmos 38:17-18 y Salmos 97:10\\
			El sentir dolor por el pecado no es expiación del mismo pero es la actitud correcta para acercarse al salvador, ahí es cuando empieza a haber la actitud de arrepentimiento genuino. David escribió 4 salmos acerca de la aflicción y el terror que había sentido a causa de sus pecados: Salmos 6, Salmos 31, Salmos 51 y Salmos 38. En ellos recordaba el tormento que había sentido en su conciencia por la oscuridad en la que vivía en aquel tiempo.\\
			Cuando tenemos algún pecado que no hemos confesado Dios hace que nuestras cadenas sean más pesadas con el propósito de que nos veamos empujados a confesarlo y lleguemos a un verdadero arrepentimiento, pidiendo perdón por la ofensa que hicimos. David cometió adulterio y asesinó a un hombre, cosas que nunca había confesado y por ello él tenía confusión y dolor hasta que Natán lo redarguyó con una parábola y David terminó escribiendo Salmos 51, David muestra que fue tan pecador como todos nosotros.\\
			Los salmos nos enseñan la forma correcta en la que debemos de dirigir nuestro arrepentiemiento. Cuando leemos estos salmos de pecadores, nos exhortan para arrepentirnos ante Dios de manera sincera y buscar la reconstrucción con la comunión ante Dios.
		\end{subsubsection}
	\end{subsection}
	\begin{subsection}{Términos musicales en los salmos}
		La música para Israel era algo de la vidad diaria, empezaron a utilizar dicha música para sus fiestas tales como cuando David trajo el arca a Jersualén o cuando Salomón hizo la edificación del templo. Hay mucha relación entre los salmos y la música tal como nosotros en nuestros templos tenemos conjuntos de alabanza.\\
		Utilizaban instrumentos como el salterio, el arpa, entre ortos. Compusieron música para acompañar los salmos buscando alabar a Dios.
		\begin{itemize}
			\item Neginot\\
				Se menciona en Salmos 4 y se refiere a un determinado instrumento de cuerdas.
			\item Nehilot\\
				Se menciona en Salmos 5. Significa herencia y se refiere a una flauta que se utilizaba en los lamentos, tradición que seguramente se copió de Egipto pues ahí se utilizaban en los velorios.
				\newpage
			\item Seminit\\
				Se menciona en Salmos 6. Se traduce como un instrumento que tenía 8 cuerdas y la referencia parece ser de que solamente se usara la $8^{a}$ cuerda que era sumamente aguda.
			\item Sigaión\\
				Se menciona en Salmos 7 y significa aullar o lamentar, es indicador de que el salmo es de lamento. La indicación es para que al momento de cantarlo o leerlo se exagerara el lamento.
			\item Masquil\\
				Se menciona en Salmos 32 Salmos 42, entre otros. Significa comprender y se utilizaba para darle un título didáctico, daban enseñanza y daban aliento para que el pueblo adorara a Dios.
			\item Mictam\\
				Se menciona en Salmos 16, Salmos 56-60. Siempre se nombra Mictam acompañado por ''\ldots de David`` se reifere a canciones que se cantaban de manera ritual y se grababan sobre una piedra.
			\item Selah\\
				Está escrita en muchos párrafos y se utiliza 71 veces, aún aparece en otros libros como Habacuc. No se sabe exactamente a qué se refiera pero puede que se refiera a un interludio o una identificación para que se repita.
		\end{itemize}
		Hay otras indicaciones en los salmos que en ocasiones son traducidas:
		\begin{itemize}
			\item ''No destruyas`` mencionada en Salmos 57.\\
				Es probable que se refiera a las primeras palabras de otro texto que conocían ellos y que era una parte muy importante de ese salmo. 
			\item ''Sobre lirios``\\
				Indica una tonada especial, puede ser mención a la tonada de otra alabanza que así iniciaba.
			\item ''La muerte del hijo``\\
				Traducción de una palabra hebrea que significa la tonada de una cacnción que inicia con esas palabras.
			\item ''La paloma silenciosa en un paraje muy lejano`` mencionada en Salmos 56.\\
				Indicación para que se utilice un título y tonada especial al entonar el salmo.
			\item ''La cierva de la mañana``\\
				Traducida del hebreo, es una indicación para el director para que ordenara su ejecución con la tonada de una alabanza llamada la cierva de la mañana.
			\item ''Oración`` mencionada en Salmos 17.\\
				Salmos 17 tiene como título ''oración de David``. Es un salmo utilizado para llamar al pueblo que acudiera al templo para adorar a Dios. Era un llamado a presentarse incluso en ropa de luto.
				\newpage
			\item ''Salmo``\\
				Se refiere a utilizar un intrumento de cuerdas, sugiere que se toque de una manera particular y distinta a lo acostumbrado.
			\item ''Cántico``\\
				Término general para referirse a un canto religioso.
			\item ''Cántico gradual`` mencionado en Salmos 120-125.\\
				Se refiere a cánticos de ascenso debido a que se cantaban cuando subían una escalinata. Los peregrinos la cantaban cuando subían a la ciudad de Jerusalén.
			\item ''Cántico de amores'' mencionado en Salmos 45.\\
				También se le llama canto nupcial. Se trata de un salmo que se cantó en la boda de un rey con una princesa. Fue cantado en una ceremonia importante que fue la boda de Acab y Jezabeel pero se siguió cantando en eventos similares.
		\end{itemize}
	\end{subsection}
Los salmos varían en tiempo desde la creación hasta los tiempos postexílicos, cuando fueron liberados de Babilonia. Vemos que fue en el tiempo de David cuando se escribieron varias victorias que tuvo el pueblo de Israel.\\
Sabemos qe las victorias de David se debieron a una relación estrecha que tuvo David con Dios. Él siempre lo apoyó pues David no perdió ni una batalla. En algunos salmos se palpan los cantos de tristeza que tuvo el pueblo de Israel com cuando estuvieron en el exilio. Vemos que todos estos hechos históricos, tanto problemas como triunfos, fueron de inspiración para escribir estos salmos. Los salmos presentan una amplia gama de la teología en la vida diaria del pueblo. La soberanía de Dios es reconocida en todos lados sin omitir la responsabilidad del hombre. Todos los scontecimientos históricos que sucedieron son entendidos a la luz de la providencia divina cuando estaba haciendo lo correcto ante Dios. 
\end{section}
%\end{document}


