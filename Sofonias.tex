%        File: Sofonias.tex
%     Created: Wed Nov 20 05:00 PM 2019 C
% Last Change: Wed Nov 20 05:00 PM 2019 C
%
%\documentclass[12pt]{article}
%\usepackage{enumerate}
%\usepackage[spanish]{babel}
%\usepackage[margin=1.0in]{geometry}
%\begin{document}
\begin{section}{Sofonías}
	\begin{enumerate}
		\item Título\\
			Sofonías, que probablemente significa ``Jehová esconde'', fue profeta del sur, el título que se le dio al libro lleva el nombre del personaje central de la narración.
		\item Autor y fecha\\
			Se le da la paterniadad del libro al profeta. Por haberse remontado 4 generaciones atrás sabemos que era de linaje real y él mismo dice que descendía del rey Ezequías, problamente por ello era pariente del rey en turno que era Josías en los años 640-609 a.C.\\
			La profecía de Sofonías fue previa a las reformas religiosas que hizo Josías al encontrar una copia del libro de la ley y debido a que Josías había visto que la fe del pueblo se había debilitado y estaban copiando las prácticas religiosas del reino del norte. Toda esa idolatría había empezado con Manasés, el peor rey que tuvo el reino del sur. De ese decaimiento espiritual nunca se recuperó el pueblo de Judá y las reformas que estableció Josías no duraron mucho.\\
			Habacuc fue contemporáneo de Sofonías. Sofonías tuvo una gran diferencia con Isaías, Isaías profetizó de pruebas que tuvo que pasar el pueblo mientras que Sofonías profetizó dentro de la corte del rey.
		\item Tema\\
			El juicio y la restauración en el día final. Sofonías invita al pueblo a congregarse en la plaza central para que reflexionaran sobre los pecados que habían tenido. Tenían poco tiempo para actuar y ello no era garantía de que se impidiera el juicio de Dios.\\
			No debían de temerle a los pueblos enemigos sino a la ira de Dios sobre el pueblo, la proxima invasión sería un presagio sobre el día de Jehová. Habla de un presagio que sería lo que constantemente repetían los profetas, el día en que todos los juicios caerían sobre la Tierra para dar lugar a la restauración de Israel y de las naciones que creyeran en el reino mesianico.
		\item Propósito\\
			Ver el cumplimiento de los planes de Dios. El futuro de la ciudad santa se veía sombrío a corto plazo pero a lagro plazo sabía que sería la morada de los fieles para siempre, cosa que había dicho Dios en el pacto davídico. Solamente habría una purificación de la ciudad y su restauración por parte del fiel remanente. Dios eliminaría la degeneración moral y espiritual de Su pueblo, quitaría a los arrogantes, los motivados por sus pecados y su avaricia que corrompían al pueblo y son de vergüenza para los justos de la ciudad y permitiría que los humildes fueran los líderes espirituales quienes hallarían su refugio y seguridad en Dios.\\
	\end{enumerate}
	\newpage
	\begin{subsection}{Bosquejo}
	\begin{subsubsection}{El juicio del Señor (Sofonías 1-2)}
		El interés principal de la profecía de Sofonías era que el pueblo entendiera que iba a venir un día de juicio inminente  que no sería cuando se destruiría la ciudad de Jerusalén sino que se refería al día de Jehová, el día del Señor, el gran día, aquel día y el día de la ira del señor, nombres distintos que ocupaba para referirse a la misma época. Sofonías advierte que ese día sería de juicio y terror acompañado de grandes catástrofes naturales donde se predecía que iba a morir la mayorparte de la gente y Dios permitiría que se salvara un remanente. El profeta habla de los días del fin.\\
		Sofonías mezcla en su profecía solo acontecimientos de los últimos días con la invasión de los caldeos en Sofonías 2:4-6.\\
		Al final de los días habrá un juicio no exclusivo para Judá sino sobre todas las naciones sobre la Tierra. En esto se enfatiza el mensaje de Sofonías a partr de Sofonías 2. En Sofonías 3 también habla acerca de Nínive, son advertidos por el profeta los pueblos que se deleitaban en hacer el mal hacia el pueblo que Dios había dicho que era su especial tesoro.\\
		Aún de esos pueblos habría quienes se vuelvan al Dios verdadero pues Dios manda una invitación general para que se arrepientan.
	\end{subsubsection}
	\begin{subsubsection}{La bendición del Señor (Sofonías 3)}
		Sofonías da una esperanza en su mensaje y advierte que habrá en los útlimos días futuro de salvación y bendición para su nación en Sofonías 3:9. Dice que cuando Jehová volviera a ser el rey de Israel no voverían a ver el mal, haciendo referencia al tiempo del milenio. Cuatro años después de que Josías muriera, los juicios profetizados por Sofonías empezaron a tener su cumplimeinto pues Dios no quería que Josías viera el juicio porque él se había humillado ante su Señor. Babilonia invadió Jerusalén mostrando que se cumplirán los juicios del futuro que profetizó también Sofonías.
	\end{subsubsection}
	\end{subsection}
\end{section}
%\end{document}


