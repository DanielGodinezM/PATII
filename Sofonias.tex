%        File: Sofonias.tex
%     Created: Wed Nov 20 05:00 PM 2019 C
% Last Change: Wed Nov 20 05:00 PM 2019 C
%
\documentclass[12pt]{article}
\usepackage{enumerate}
\usepackage[spanish]{babel}
\usepackage[margin=1.0in]{geometry}
\begin{document}
\begin{section}{Sofonías}
	\begin{enumerate}
		\item Título\\
			Fue profeta del sur, tl título lleva el nombre del perosnaje central de la narración.
		\item Autor y fecha\\
			Se le da la paterniadad del libro al profeta. Por haberse remontado 4 generaciones atrás sabemos que ra edel linaje real y él mismo dice qu descendía del rey Ezequías y problamente por ello era partiente del rey en turno que era Josías 640-609ac.\\
			La profecía de Sofonóiasn fue previa a las reformas religiosas que hizo Jodías que encontró un acopia del libro de la ley y debido a que Josías había visto que la fe del puebo se había debilitado y estababn copiando sus prácticas religiosas del reino del norte. Toda esa idolatría había empezado con manasés, el peor rey que tuvo el reino del sur. De ese decaimeietno espiritual nunca se recuperó e pueblo de Judá y las refomras que estableció Josías no duraron mucho.\\
			Habacuc fue contemporáneo de Sofonías. Sofonías tuvo una grna diferencia co Isaías. Isaías profetizó de preuba que tuvo que psasr el pueblo y Sofonías profetizó dentro de la corte del rey.
		\item Tema\\
			Da 2 invitaciones de Dios, acongregarse en la plaza central para que reflexionaran sobre sus pecads ue habían tenido, tenían poco tiempo y no era garantía de que elos impediría el juicio de Dios.\\
			Nod ebáin de temerle a los pueblos enemigos sinoa a la ira de Dios sobre el ueblo, la proxima nvasiónse ría un presagio sobre el día de Jehová. Habla deun presagio que sería lo que constantemente repetían los profeta el día en que todos lo sjuicios caerían sobre a tierra para dar lugar a la returación de Israel y de las naciones que creyeran en el reino mesianico.
		\item Propósito\\
			Ver elcumplimiento de los planes de Dios. El furuto de la ciudad santa se veía sombrío a corto plazo pero a lagro plazo sabía que sería la morada de los fieles para siempre ocsa que había dicho dDios en el pacto davídico. Solamente habría una pruficaión de la ciudad y su restauración por aprte del fiel remanente. Dios eliminaría la degeneración moral y espiritual de su pueblo, va a aquitar a los arrogantes, los motivados por sus pecaods y su avaricia que corromían alpueblo y son de vergüenza para los justos de la ciudad y permitiía que los humildes fueran los líderes espirituales.\\
			Iban a hallar su refugio y seguridad en Dios
	\end{enumerate}
	\begin{subsection}{Bosquejo}
	\begin{subsubsection}{El juicio del Señor (Sofonías 1-2)}
		El nterés princiapl de la rofecía de Sofonías era que el puebo enteendiera que iba a venir un día de juicio inminente  uq eno sería cuando se destruiría la ciudad de Jerusalén sino que se referia al día de Jehová, el día del Sñeor, el gran día, auqle día, el día de la ir adel señor, el día. Sofonías advierte que se día sería de juicio y terror acompañado de grandes cataástrofers naturales donde se predecía que iba a morir la mayorpande la gente y Dios permitiría que se salvara un remanente. El porfeta habla de los días del fin.\\
		Sofonías mezlca en su profecía slo acontecimientos de los últimos días con la invasión de los caldeos. EN los vs 4-5, vs 6.\\
		Al final de os días habrá un juicio no exclusivo para udá sino sobre todas las naciones sobre la tierra, universal. En esto se enfatiza el mensaj de Sofonías a partr del cap 2. EN el cap 3 también habla acerca de Nínive, son advertido or el rofeta los pueblos que se deleitaban en hacer el mal hacia el pueblo que Dios había dicho que era su epsecial tesoro.\\
		Aun de esos pueblos habría quienes se vuelvan al Dios verdadero pues Dios manda una invitación general pra que se arrepientan.
	\end{subsubsection}
	\begin{subsubsection}{La bendición del Señor (Sofonías 3)}
		Sofonías da una esperanza en su mensaje y advierte que habrá en lls útlimos días futor de salvación y ebdnició para su nación, 3:9. Dice que cuando Jehová volviera a ser el rey de Israel no voverían a ver el mal haciendo referencia al tiempo del milenio. 4 años después de que Josías muriera, ls juiicos profetizados por Sofonías empezaron a tener su cumplimeinto pues Dios no quería que Josías viera el juicio orque él se había humillado ante su Señor. babilonia invadio Jerusalén mostrando que se cumplirán los juicios del futuro que profetizó tamién Sofonías.
	\end{subsubsection}
	\end{subsection}
\end{section}
\end{document}


