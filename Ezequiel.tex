%        File: Ezequiel.tex
%     Created: Wed Nov 13 07:00 PM 2019 C
% Last Change: Wed Nov 13 07:00 PM 2019 C
%
%\documentclass[12pt]{article}
%\usepackage{enumerate}
%\usepackage[spanish]{babel}
%\usepackage[margin=1.0in]{geometry}

%\begin{document}
\begin{section}{Ezequiel}
\begin{itemize}
	\item Título\\
		Le fue dado debido al profeta Ezequiel quien no se menciona en algun otro libro de la Biblia. En la primera deportación fue cautivo el rey Joacim, la segunda deportación se narra en $2^{o}$ Crónicas 36:9-10 cuando fue llevado cautivo el rey Joaquín. En la tercera deportación fue cuando se destruyó Jerusalén y el templo, narrada en $2^{o}$ Reyes 25:1-5, el rey que fue llevado cautivo fue Sedequías a quien le sacaron los ojos. En cada deportación, además de llevarse al rey se llevaban a cierta parte del pueblo.\\
		Se habla de la segunda deportación en el libro por lo que se considera que Ezequiel fue llevado a Babilonia. Dice el libro que era sacerdote, se utiliza el lenguaje simbólico de manera extensa por lo que es un libro difícil de entender e interpretar.
	\item Autor y fecha\\
		Ezequiel es el autor ya que se narran sus profecías y en los primeros versículos aparece su nombre como si él mismo lo estuviera narrando en primera persona. Ezequiel fue hijo de Buzi, quien le enseño el deber de ser sacerdote. Ezequiel nació en Judá, cuando tenía 25 años ocurrió la segunda deportación y Joaquín fue llevado, se dice que fueron hasta 10,000 cautivos.\\
		Sedequías fue el tercero de los hijos de Josías, el último rey antes de la destrucción de la capital de Judá. Sedequías era el nombre que le había puesto Nabucodonosor ya que en un prinnipio se llamaba Matanías como se narra en $2^{o}$ Reyes 24:17.\\
		Ezequiel permaneció, probablemente, en Babilonia por el resto de su vida. Vivía en Tel-abib, junto a un río. A los 30 años empezó a ejercer su ministerio sacerdotal y empezó a tener revelaciones a través de visiones que vienen narradas en el libro. Cuando estaba Ezequiel en el cautiverio fue cuando ocurrió la destrucción de Jersualén y la deportación del resto del pueblo de Judá, aunque Ezequiel estaba cautivo sus profecías seguían siendo acerca de la destrucción de Jersusalén. Isaías les profetizó a los habitantes de Jerusalén mientras que Ezequiel les profetizó a los que estaban con él en el exilio.\\
	En la segunda parte del libro habla sobre la restauración futura.
\item Tema\\
	El juicio y la restauración de Israel. Deberían de ser obedientes y seguir buscando a Dios pero su restauración final seguía estando prometida en un futuro remoto, habría un nuevo templo y habrían de contar con un reino eterno del Mesáis prometido.
\item Propósito\\
	Mostrar el futuro escatológico de Israel. La restauración final de Israel sería hasta los últimos tiempos donde estaría establecido el reino milenial del Mesías, todas las profecías acerca de ello no se han cumplido ya que son para los últimos tiempos pero son la esperanza de que algún día habrá una restauración gloriosa del pueblo de Israel y del trono de David.
\end{itemize}
\begin{subsection}{Bosquejo}
	\begin{subsubsection}{Profecías de juicio desde el exilio: Judá, antes de la caída (Ezequiel 1-24)}
Aparecen las primeras visiones del profeta, este tipo de visiones que tuvo no son fáciles de interpretar ni de entender y para llegar a comprenderlas es necesario que no tomemos el estudio de manera superficial.\\
Vio fuegos, resplandores, seres vivientes con 4 caras y cuatro alas, entre otras cosas. De manera superficial podemos determinar que Ezequiel recibió por medio de una revelación de Dios, una revelación de Su gloria particularmente indescriptible. En Ezequiel 2-3 vemos la forma en la cual es comisionado el profeta, siendo advertido que estaba siendo enviado a gente que tenía el corazón duro.\\
En Ezequiel 2:26 se dice que Dios lo enmudece porque no quería que siguiera reprendiendo a un pueblo que no quería escuchar las advertencias de Dios pero le da la facultad de hablar en ciertas ocasiones, las señales mostraban como ibaa ser disciplinado el pueblo de Dios. Al final de Ezequiel 6 Dios anuncia que dejaría un remanente, dice que será conformado por aquel que se avergüence y se arrepienta. La mayor parte del pueblo iba a ser rechazada pero al remanente le concedería la gracia y misericordia de Dios.\\
\\
En Ezequiel 8 nuevamente Dios le concede al profeta que vea una visión de Su gloria, Ezequiel es trasladado en espíritu a Jerusalén donde vio las abominaciones que su pueblo seguía haciendo.\\
Lo que vio es que en el templo de Jehová los ancianos estaban adorando a pinturas de ídolos, después le dice que estaban haciendo cosas peores dentro del templo y vio que mujeres estaban adorando a Tammuz, dios babilónico de la fertilidad. La adoración consistía en abominaciones sexuales.\\
En Ezequiel 8:16 Dios le lleva a que vea peores abominaciones que se realizaban dentro del templo donde adoraban a cualquier otra cosa, le mostró que habían 25 hombres dándole la espalda al templo al darle culto al sol. Por ello se encendió la ira de Dios y dijo que no tendría misericodria de ellos.\\
Ezequiel vio a 6 varones con instrumentos para destruir, uno de ellos era un escribano con un tintero y Dios le indica que pusiera una señal a los hombres que gemían ante las abominaciones que hacía el pueblo, a los demás varones se les dió la misión de matar al resto del pueblo. Al final de Ezequiel 11 Dios habla de un remanente que recogerá de toda la Tierra y que los traería de nuevo a Jersualén.\\
\\
Después, Ezequiel ve en su visión cómo la gloria de Dios se separa de la ciudad santa. Dios, al final, abandona a Su pueblo y los deja a su suerte.\\
En Ezequiel 12 Dios manda a Ezequiel que hiciera un simulacro sobre la forma en que iban a partir al destierro y que así debían de irse de la ciudad en un futuro no muy lejano, Sedequías mismo intentó escapar de la misma forma.\\
La visión que tuvo Ezequiel concluye diciendo que no tardaría mucho tiempo en cumplirse. En Ezequiel 13-14 vuelve a ver visiones sobre lo que les ocurriría a los sacerdotes. Dios le confirma que de ninguna manera Jerusalén sería librada de su castigo.
\newpage
En Ezequiel 16 se personifica a Jerusalén, se le compara con una prostituta en la que nadie tenía interés, había sido arrojada al campo pero Dios la había rescatado para que viviera.\\
Menciona que Dios permitió que esta ``niña'' creciera y se convirtiera en una mujer muy hermosa, describe su pelo y sus pechos. Dios la protegió y se unió a ella, la arregló, la vistió, la cubrió de joyas pero a partir de ese momento se dedció a la protitución pues dice que hizo ídolos con sus joyas. Los hijos que tuvo los sacarificó a sus ídolos. Fornicó con los egipcios, filisteos y menciona que nunca se saciaba de pecar.\\
Dios le dice que a sus amantes la entregaría para destruirla. Habla de sus hermanas Samaria y Sodoma y dice que Jerusalén las había superado en pecado. Se ve el carácter de la misericordia de Dios en los últimos versículos pues Dios recordaría el pacto que tenía con ella y que haría un pacto eterno para que recordara los caminos que Dios le había enseñado.\\
En Ezequiel 17-21 se sigue hablando de la ira de Dios en contra de Su pueblo, del castigo a los padres y a los hijos que sería sobre los príncipes de Israel y sus ancianos. Habla de las espadas de Dios que se convertiría en homicida, una espada desenvainada y lista para consumir. En Ezequiel 23 vuelve a comparar a Jerusalén con una prostituta y que Jerusalén había superado la iniquidad de Samaria. Les da nombres particulares, a Jerusalén le llama Aoida y Aoda a Samaria.\\
En Ezequiel 24 narra su visión acerca del día de la destrucción de Jerusalén, el mismo día que fue destruida Jerusalén muere la esposa de Ezequiel y Dios le pide que no llore y que no haga luto con el propósito de que fuera de testimonio para que el pueblo no hiciera luto por la destrucción, reprimiendo el dolor que sentían. Internamente debían de dolerse por sus pecados\\
Ese mismo día vendría un fugitivo que regresaría de la matanza, le contaría todo lo que había pasado en Jerusalén y le quitaría Dos la mudez a Ezequiel.
	\end{subsubsection}
	\begin{subsubsection}{Profecías de juicio desde el exilio: Gentiles (Ezequiel 25-32)}
En Ezequiel 25 Ezequiel profetiza juicio sobre 4 pueblos que querían venganza sobre Israel, Dios iba a castigar a todo pecador. Estas naciones suponían que la ira de Dios sería solamente sobre Su pueblo pero también sería sobre de ellos. Habla primero en contra de Amón que se había unido con los caldeos en contra de Judá. En Ezequiel 25:7 Dios da sentencia sobre Amón.\\
Después Ezequiel en Ezequiel 25:8-11 profetiza sobre el juicio venidero sobre Moab, castigado por su alegría ante la caída de Jerusalén. Ambos pueblos fueron posteriormente absorbidos por los pueblos árabes. En Ezquiel 25:12-14 se habla sobre la profecía sobre Edom, nación condenada a desaparecer también. Dios utilizó a su mismo pueblo para este juicio.\\
En Ezequiel 25:15-17 se ve el juicio sobre los filisteos por la hostilidad perpetua que tuvieron en contra de Israel desde que salieron de Egipto. Jeremías 47 menciona que los filisteos también fueron invadidos por Nabucodonosor.
Durante Ezequiel 26-28 se habla sobre la ciudad de Tiro, quienes habían sido, en el tiempo de la monarquía unida, amigos y aliados de los judíos pero en el libro de Joel habla de que inclusive los de Tiro tenían esclavizados a los judiós y los vendían a otros pueblos. 
\newpage
En Ezequiel 26:3, se cumple dicha profecía cuando Babilonia ataca a Tiro y se le nombra a Nabucodonosor como rey de reyes quien fue el instrumento de Dios para castigar de manera parcial a Tiro, la desaparición total fue hasta el tiempo intertestamentario cuando el ejército griego al mando de Alejandro Magno completó la destrucción que había empezado Nabucodonosor en la ciudad de Tiro.\\
En Ezequiel 28 se hace una descripcipón del rey de Tiro de manera similar al pasaje de Isaías donde se describe a Satanás a través de un rey humano. Lo que se menciona en Ezequiel 28:12-13 se entiende como una descripción exclusiva de Satanás, mencionando además que estaba en el Edén. En los versículos siguientes los adjetivos que le aplican también llevan a pensar que se refiere a Satanás ya que se refiere a él como querubín perfecto (Ezequiel 28:15). Ninguno de los adjetivos que menciona pueden ser cumplidos por un ser humano.\\
  El siguiente juicio es sobre Cidón, el cual era un puerto marítimo. Ciudad centro de la adoracíon a Baal, sus habitantes aceptaron que los juicios venían de parte de Dios, tanto las enfermedades como las espadas enemigas.\\
  En Ezequiel 29-32 se ve la profecía contra Egipto escrita en el año 568 a.C. aproximadamente. En Ezequiel 30:26 Dios manifiesta misericordia hacia los egipcios diciendo que los recogería y los volvería traer de su cautiverio. 
  \end{subsubsection}
  \begin{subsubsection}{Profecía de restauración de Israel (Ezequiel 33-39)}
	  Se dan las instrucciones para el arrepentimiento y restauración de Israel, al momento que llega a Ezequiel un fugitivo de la destrucción de Israel el profeta recupera el habla. En Ezequiel 37 se observa una visión particular acerca de un valle lleno de huesos secos en Ezequiel 37:7-10. El propóstio de esta visión fue que Dios quería dar consuelo a Su pueblo cuando se enteraron de la destrucción de Jerusalén, los huesos los representaban como muertos espiritueales, no físicos. Dios promete la resurreción espiritual de Israel. En Ezequiel 37:15-18 el pueblo estaba dividido en dos entonces Ezequiel junta dos palos, uno representando al reino del sur y otro representando el reino de norte, en Ezequiel 37:22 la promesa es que una vez unido el pueblo Dios les iba a poner un rey, el Mesías prometido. En Ezequiel 38-39 la profecía es en contra de una nación llamada Gog, profecía escatológica que habla de una confederación que vendría del norte a invadir a Israel, profecía que también se menciona en Apocalipsis. \\
  \end{subsubsection}
  \begin{subsubsection}{Visión escatológica de la restauración (Ezequiel 40-48)}
Todas estas profecías son escatológicas, se incluyen detalles específicos acerca del reino milenial. Debido a que las profecías están aún por cumplirse, esto constituye la restauración final que Dios había prometido. Es estos últimos capítulos se menciona un nuevo templo en la era milenial, un templo que iba estar en el futuro. Del Ezequiel 44-46 se menciona cómo debería de ser el nuevo culto de adoración, en Ezequiel 47-48 se describe la nueva repartición de la tierra prometida la cual no se comprende en su totalidad.

	\end{subsubsection}
\end{subsection}
\end{section}
%\end{document}


