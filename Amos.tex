%        File: Amos.tex
%     Created: Wed Nov 20 02:00 PM 2019 C
% Last Change: Wed Nov 20 02:00 PM 2019 C
%
\documentclass[12pt]{article}
\usepackage[spanish]{babel}
\usepackage{enumerate}
\usepackage[margin=1.0in]{geometry}
\begin{document}
\begin{section}{Amós}
	\begin{enumerate}
		\item Título\\
			El profeta Amós procede del reino del norte. Es elnombre del profeta al que Dios el dio este mensaje, fue un humilde pastor. No era instruido según dice el libro se dedicaba a recoger fruta. No era del reino del norte pero allí fue a profetizar. En 1:1 se ve que era de una región cercana de Jerusalén.\\
			Oseas fue la última voz de advertencia para isarel pero Amós fue el primero que empezó a profetizar sobre el juicio en contra de Israel. Oseas enfatizó el amor de Dios en su ministerio mientras que Amós enfatiza el juicio. segurametne mIqueas oyeron sus prefacías en su jueventud.
		\item Autor y fecha\\
			Se le considera el escritor al mismo porfeta, contemporanoe de Jonas, Miqueas y Oseas. el rey era Uzías en Judá. En el vs 1 profetizo en los 793-753ac En el vs 1 dice que fue 2 años antes del terremoto aunque Palestina no es una zona sísmisa. Zacarías 14:15 menciona también un terremoto.\\
			Había mucha prosperidad, paz pero al mismo tiempo estaban muy corrompidos. Hubo una corrupción desenfrenada, decadencia moral y espiritual. Dios envía a Amós para transmitir mensaje a los apóstatas.
		\item Tema\\
			Juicio sobre el pecado de Israel. En el cap 1 y en el principio de 2 Amós profetizó juicio pero lo hace de forma particular para Israel, le recueda que a pesar de la constante disciplina no entendían y seguían pecado. dios les recimina que no se volvieron a Él.
		\item Propósito\\
			Advertir el juicio de Dios. Amós lanza la advertencia sobre Juda el Israel al igual que las naciones vecinas. Amós advirtío que tuverian en cuenta que era inevitable las disciplina si seguían viviendo en pecado. Todo aquel que pecare ha de morir.
	\end{enumerate}
	\begin{subsection}{Bosquejo}
		\begin{subsubsection}{Juicio (Amós 1-2)}
			Por 3 pecados nombra a la nación y por cuarto no revocaría su castigo, repite ese mnejsae. Amós dice que 3 pecados eran más que sufcientes para airar a Dios. Amós lanza la devertencia del juicio sobre las naciones vecinas como Damasco, Gaza, Tiro, Amón, Moab 2:1 y en el cap 2 l sentencia habla también para Israel y Judá. La sentencia era igual para todos. Dios no es ciego ante el peacdo y siempre lo castiga por igual y lo hace aplicando lo que le había prometido a Abraham de que bendeciriía alos que lo bendijesen y maldeciría a los que lo maldijeran, todos los pueblos que despreciaron a Judá fueron desaparecidos.
		\end{subsubsection}
		\begin{subsubsection}{Condenación (Amós 3-6)}
			Se inica co n las mismas palabras en cada uno de los cap 3,4 y 5. Después viene la advertencia particualr a Israel. Dios acusará a Israel de sus pecados y les recuerda la disciplina constante. Ahora les dice que se preparaara para el encuentro con Dios.\\
			En el vs 4:1, Amós al reeferirse a las vascas se refieren a las ujeres que vivían en Samari con lujos que eran de la alta sociedad de Samaria, quizá con buena salud, joyas pero su acitud hacia los pobres era reprochables. Estas mujeres y sus maridos oprimías y malatrataban a los pobres para tener abundancia de dinero. Al mencionar a estas mujeres como vacas se ve reflejado su trasfondo agrícola\\
			En 5:21-23 Dios rechaza sus ofrendas y alabanza que eran solamente muestras de mreligiosidad. Se ve una profecía ya que los habitantes de JUdá que eran los nobles esclavizando alos pobres fueron las primera personas en ser llevadas en las deportaciones.
		\end{subsubsection}
		\begin{subsubsection}{Restauración (Amós 7-9)}
			Amós muestra que el juicio de Dios está listo para castigar a SIrael. En el ca 7 se ven 5 visiones del porfeta, en 1-3 langostas , fuego 4_6, plomada 7-9 y en 10-13 Amós menciona cómo él es corrido del reino porun sacerdote que ra leal al rey y no a Dios.\\
			En el cap 9:1-10 una visión de Dios que mensiona cómo sería el juicio en donde también se menciona la palabra restauración en relación al futuro. La dinastía davídica seria restaruada e Israel sería enriquecida una vez más.\\
		\end{subsubsection}
	\end{subsection}
\end{section}
\end{document}



