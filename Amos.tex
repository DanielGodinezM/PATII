%        File: Amos.tex
%     Created: Wed Nov 20 02:00 PM 2019 C
% Last Change: Wed Nov 20 02:00 PM 2019 C
%
%\documentclass[12pt]{article}
%\usepackage[spanish]{babel}
%\usepackage{enumerate}
%\usepackage[margin=1.0in]{geometry}
%\begin{document}
\begin{section}{Amós}
	\begin{enumerate}
		\item Título\\
			El profeta Amós, quien fue un humilde pastor, procede del reino del norte y es por el cual se le otorga el nombre a este libro. No era instruido y según dice el libro se dedicaba a recoger fruta. No era del reino del norte pero allí fue a profetizar, en Amós 1:1 se ve que era de una región cercana de Jerusalén.\\
			Oseas fue la última voz de advertencia para Israel pero Amós fue el primero que empezó a profetizar sobre el juicio en contra de Israel. Oseas enfatizó el amor de Dios en su ministerio mientras que Amós enfatiza el juicio.
		\item Autor y fecha\\
			Se le considera el escritor al mismo porfeta,  quien fue contemporáneo de Jonás, Miqueas y Oseas. De acuerdo a Amós 1:1 se observa que empezó su ministerio desde el reinado de Uzías en Judá y hasta 2 años antes del terremoto, por ello se puede determinar que su ministerio fue en los años 793-753 a.C.\\
			El terremoto que menciona Amós también es mencionado por el profeta Zacarías en Zacarías 14:5.\\
			Había mucha prosperidad y paz pero al mismo tiempo estaban muy corrompidos. Hubo una corrupción desenfrenada, decadencia moral y espiritual y Dios envía a Amós para transmitir un mensaje a los apóstatas.
		\item Tema\\
			Juicio sobre el pecado de Israel. En Amós 1 y en el principio de Amós 2, Amós profetiza juicio pero de manera particular para Israel pues les recuerda que a pesar de la constante disciplina que mostraban en realidad no la tenían pues seguían pecando, Dios les recrimina que no se volvieron a Él.
		\item Propósito\\
			Advertir el juicio de Dios. Amós lanza la advertencia sobre Judá e Israel al igual que las naciones vecinas. Amós les advirtió que tuvieran en cuenta que era inevitable la disciplina si seguían viviendo en pecado. Todo aquel que pecare ha de morir.
	\end{enumerate}
	\begin{subsection}{Bosquejo}
		\begin{subsubsection}{Juicio (Amós 1-2)}
			Por 3 pecados nombra a la nación y por 4 no revocaría su castigo, este mensaje se los repite constantemente y con ello les dice que 3 pecados eran más que sufcientes para airar a Dios. Amós lanza la advertencia del juicio sobre las naciones vecinas como Damasco, Gaza, Tiro, Amón, Moab y en Amós 2 la sentencia habla también para Israel y Judá.\\
			La sentencia era igual para todos pues Dios no es ciego ante el peacdo y siempre lo castiga por igual aplicando lo que le había prometido a Abraham de que bendeciría a los que lo bendijeren y maldeciría a los que lo maldijeren, todos los pueblos que despreciaron a Judá fueron desaparecidos.
		\end{subsubsection}
		\newpage
		\begin{subsubsection}{Condenación (Amós 3-6)}
			Se inicia con las mismas palabras en cada capítulo de Amós 3-5. Después de ello es que viene la advertencia particular para Israel. Dios acusará a Israel de sus pecados y les recuerda la disciplina constante que habían vivido. Ahora les dice que se prepararan para su encuentro con Dios.\\
			En Amós 4:1, Amós se refiere a las vacas como a las mujeres que vivían en Samaria con lujos que eran de la alta sociedad, quizá gozaban de buena salud y joyas pero su actitud hacia los pobres era reprochable. Estas mujeres y sus maridos oprimían y maltrataban a los pobres para tener abundancia de dinero. Al mencionar a estas mujeres como vacas se ve reflejado su trasfondo agrícola.\\
			En Amós 5:21-23 Dios rechaza su ofrenda y alabanza ya que eran solamente muestras de religiosidad superficial. Este capítulo termina con una profecía ya que los habitantes de Judá, que eran los nobles esclavizando a los pobres, fueron las primeras personas en ser llevadas en las deportaciones.
		\end{subsubsection}
		\begin{subsubsection}{Restauración (Amós 7-9)}
			Amós muestra que el juicio de Dios está listo para castigar a Israel. En Amós 7 se ven 3 visiones del profeta:
			\begin{itemize}
				\item En Amós 7:1-3 se ve al Señor preparando langostas para Israel.
				\item En Amós 7:4-6 el Señor prepara fuego para castigar a Israel.
				\item En Amós 7:7-9 Jehová dice que usaría una plomada para probar a Su pueblo. 
		\end{itemize}
				En Amós 7:10-13 Amós menciona cómo él es corrido del reino por un sacerdote que era leal al rey Jeroboam y no a Dios.\\
			En Amós 9:1-10 se narra una visión de Dios que menciona cómo sería el juicio próximo y la restauración futura. La dinastía davídica sería restaurada e Israel sería enriquecida una vez más.\\
		\end{subsubsection}
	\end{subsection}
\end{section}
%\end{document}



