%        File: Miqueas.tex
%     Created: Wed Nov 20 05:00 PM 2019 C
% Last Change: Wed Nov 20 05:00 PM 2019 C
%
\documentclass[12pt]{article}
\usepackage{enumerate}
\usepackage[spanish]{babel}
\usepackage[margin=1.0in]{geometry}
\begin{document}
\begin{section}{Miqueas}
	\begin{enumerate}
		\item Título\\
			Es el nombre del profeta que protagoniza el libro. su ministerio profético fue llevaod a cbao cuando estaban siendo invadidos por Asiria en el reino del norte. Contemporaneo de isaías en el reino del sur. era origniario del reino del sur y profetiza para los dos reinos. 
		\item Autor y fecha\\
			Tradicionalmente se dice que Miqueas lo escribió, se sitúa la sescritura en 735-710 ac. Sus rofecías e suman a las profecías que había dado Oseas al reino del norte e Isaías estaba prediacdno lo mismo para el reino del sur.
		\item Tema\\
			Dios restaurará a Israel. Se une al tema del compromiso inmutable de Dios por sus promesas, un compromiso riguroso del pacto provee una revelación del carácter de Dios en el que se revelba como sberano y todopoderoso. Así como envió juicio sobre el pecado de su peublo, traería bendición con aquellos que fueran fieles. En las mismtas Escritauras etabam replteas de promesad e que nunca abandonaría a su pueblo, uiqne recibirá las bendiciones que Él prometió en el fin de los tiempos.\\
			Todo estaba en el plan de Dios, no iba  as er todo el pueblo el salvado, solamente el remanente.
		\item Propósito\\
			La gracia de Dios. Israel siempre estuvo en contraposición a los mandatos de sDios pero al esencia de lacdoctrina de la gracia des que Dios está por todos nosotros, inclusive aquellos que están en desobediencia. La gacia de Dios está asociada con el pactoq ue ahbía heco con ISrael, dice jeremías que los había amdo con amor eterno. El apcto nuevo que Dios hizo con Israle hace énfasis en las bendiciones espirituales que sus escogidos habrían de recbir.\\
			El pacto anterior al que se reifer es el mosaico y Dis dice que Él había sido fiel pero su pueblo no. Dios estblece este nuevo pacto que se distnigue porque por medio del recibirían esa capacidad de obedecer porque solamente habín tenido una ley escrita en piedra. Dios en su nuevo pacto les había dicho que iba a estar escrita en sus corazones obtendría de Dios el deseo y la habilidad de cumplir los mandamientos que Él les había dado. todos conoceránq quien verdadermanete era Dios  todo lo que hacía por ellos era por su gracia y no porque lo merecieran, una gracia extendidad par todas laas naciones.
	\end{enumerate}
	\begin{subsection}{Bosquejo}
		\begin{subsubsection}{Mensaje de castigo (Miqueas 1-2)}
			Miqueas centra su mensaje acerca del juicio sobre Israel y Judá. Primero profetiza que Samria sería destruida por idolátria, Asiria la invadiría y la destruiría pero también ello llegaría a Jerusalén.Muchas del reino del norte emigraron al sur pero ello fue algo negativo posteriormente ya que ellos traían maldad y religión contamidad de cree en múltiples deidades y contaminaron a Judá. En el 2:1-2 da razones del juicio para Judá. 
		\end{subsubsection}
		\begin{subsubsection}{Mensaje de promesa (Miqueas 3-5)}
			Miqueas les recuerda que Dios guarada sus promsesa y pactos. En el cap 4 les dice que aunque la destrucción es inminente hay un mensaje de promesa, un remanetene sería librado Miqueas 4:6-7. Termina en el cap 5 hablando de la promesa del Mesías y su reino milenial que vendría de Belén, vs 2. Esta es la profecía sobre el lugra denacimiento del Mesías.
		\end{subsubsection}
		\begin{subsubsection}{Mensaje de perdón (Miqueas 6-7)}
			A pesar de la infidelidad de su pueblo Dios había decidido perdonar a un remanente, Él siempre le había ayudado al pueblo. Ellos mismo se preguntaban cómo se presentarían frente a dios para adorarlo. realmente Dios les pdeía, 6:8, que hicieran justicia, amar misericordia y humillarse delante de Él, eso es lo que Dios requiere de todo hombre.\\
			Lo que pide Dios es mucho más importante que cualquier sacrificio que s ele quiera ofrecer o cualquier hipocresías que quisiera ocultar el pecado. El mensaje de Miqueas termina con el reconocimiento de la misericordia y compasión de Dios. El libro es una gra esneñana de que Dio siempre castiga el pecado pero seiempre está detrrás su misericordia y perdń para el que se arrepiente.
		\end{subsubsection}
	\end{subsection}
\end{section}
\end{document}


