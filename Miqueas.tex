%        File: Miqueas.tex
%     Created: Wed Nov 20 05:00 PM 2019 C
% Last Change: Wed Nov 20 05:00 PM 2019 C
%
%\documentclass[12pt]{article}
%\usepackage{enumerate}
%\usepackage[spanish]{babel}
%\usepackage[margin=1.0in]{geometry}
%\begin{document}
\begin{section}{Miqueas}
	\begin{enumerate}
		\item Título\\
			Miqueas, que significa ``¿Quién es como Jehová?'', es el nombre del profeta que protagoniza el libro, razón por la cual se le puso el título. El ministerio profético de Miqueas fue llevado acabo cuando estaban siendo invadidos por Asiria en el reino del norte. Miqueas fue contemporáneo de Isaías en el reino del sur, era originario del reino del sur y profetiza para los dos reinos. 
		\item Autor y fecha\\
			Tradicionalmente se dice que Miqueas escribió el libro, se sitúa la escritura en los años 735-710 a.C. Sus profecías se suman a las profecías que había dado Oseas al reino del norte e Isaías estaba predicando lo mismo para el reino del sur.
		\item Tema\\
			Dios restaurará a Israel. Se une al tema del compromiso inmutable de Dios por sus promesas, un compromiso riguroso del pacto provee una revelación del carácter de Dios en el que se revelaba como soberano y todopoderoso. Así como Dios envió juicio sobre el pecado de Su peublo, traería bendición con aquellos que fueron fieles. Las mismas Escrituras que tenían estaban repletas de promesas de que nunca abandonaría a Su pueblo quienes recibirán las bendiciones que Él prometió en el fin de los tiempos.\\
			Todo estaba en el plan de Dios, no iba a ser todo el pueblo el salvado sino solamente el remanente.
		\item Propósito\\
			La gracia de Dios. Israel siempre estuvo en contraposición a los mandatos de Dios, a la esencia de la doctrina de la gracia de que Dios está por todos nosotros, inclusive aquellos que están en desobediencia. La gracia de Dios está asociada con el pacto que había hecho con Israel, dice Jeremías que los había amado con amor eterno. El pacto nuevo que Dios hizo con Israel hace énfasis en las bendiciones espirituales que sus escogidos habrían de recibir.\\
			El pacto anterior al que se refiere es el mosaico y Dios dice que Él había sido fiel mientras que Su pueblo no. Dios establece este nuevo pacto que se distnigue porque por medio de él recibirían esa capacidad de obedecer. Dios les había dicho que su nuevo pacto iba a estar escrito en sus corazones, obtendrían de Dios el deseo y la habilidad de cumplir los mandamientos que Él les había dado. Todos conocerán quien verdaderamente era Dios, todo lo que hacía por ellos era por Su gracia y no porque lo merecieran, una gracia extendida para todas las naciones.
	\end{enumerate}
	\begin{subsection}{Bosquejo}
		\begin{subsubsection}{Mensaje de castigo (Miqueas 1-2)}
			Miqueas centra su mensaje acerca del juicio sobre Israel y Judá. Primero profetiza que Samaria sería destruida por idolatría, Asiria la invadiría y la destruiría pero también ello llegaría a Jerusalén.
			\newpage
			Muchos del reino del norte emigraron al sur pero ello fue algo negativo ya que ellos traían maldad y religión contaminada de creencias en múltiples deidades y contagiaron esa contaminación. En Miqueas 2:1-2 se dan las razones del juicio divino para Judá. 
		\end{subsubsection}
		\begin{subsubsection}{Mensaje de promesa (Miqueas 3-5)}
			Miqueas les recuerda que Dios guarda sus promesas y pactos. En Miqueas 4 les dice que aunque la destrucción es inminente hay un mensaje de promesa, un remanente sería librado según Miqueas 4:6-7. Termina Miqueas 5 hablando de la promesa del Mesías y su reino milenial que vendría de Belén, Miqueas 5:2. Ésta es la profecía sobre el lugar de nacimiento del Mesías.
		\end{subsubsection}
		\begin{subsubsection}{Mensaje de perdón (Miqueas 6-7)}
			A pesar de la infidelidad de Su pueblo Dios había decidido perdonar a un remanente. Ellos mismos se preguntaban cómo se presentarían frente a Sios para adorarlo, mientras que Dios realmente les pedía que hicieran justicia, amaran misericordia y se humillaran delante de Él en Miqueas 6:8. Eso es lo que Dios requiere de todo hombre.\\
			Lo que pide Dios es mucho más importante que cualquier sacrificio que se le quiera ofrecer o cualquier hipocresía que quisiera ocultar el pecado. El mensaje de Miqueas termina con el reconocimiento de la misericordia y compasión de Dios. El libro es una gran enseñanza de que Dios siempre castiga el pecado pero siempre está detrás Su misericordia y perdón para el que se arrepiente.
		\end{subsubsection}
	\end{subsection}
\end{section}
%\end{document}


