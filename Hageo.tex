%        File: Hageo.tex
%     Created: Wed Nov 20 06:00 PM 2019 C
% Last Change: Wed Nov 20 06:00 PM 2019 C
%
%\documentclass[12pt]{article}
%\usepackage{enumerate}
%\usepackage[spanish]{babel}
%\usepackage[margin=1.0in]{geometry}
%\begin{document}
\begin{section}{Hageo}
	\begin{enumerate}
		\item Título\\
			Hageo, que significa ``el festivo'', profetizó después del exilio así como Zacarías y Malaquías cuando regresó el pueblo a Jerusalén. EL título del libro lleva el nombre del profeta del que se habla en el libro. 
		\item Autor y fecha\\
			Hageo es considerado como el autor del libro, en Esdras 5:1 se menciona a Hageo junto con Zacarías. Es probable que Hageo haya sido una perosna joven que incluso pudo haber nacido durante el exilio y no conocía Jerusalén.\\
			Él mismo describe en el libro que ocurrió en los años 521-486 a.C. que fue el tiempo en que Israel regresó de la cautividad, se considera que Hageo regresó de Babilonia a Jerusalén en el primer grupo que regresó junto con Zorobabel.
		\item Tema\\
			La exhortación a reconstruir la casa de Dios. Las obras del templo de Dios habían sido detenidas por muchos años debido a que los enemigos de Judá les habían estorbado, sin embargo, ellos lo habían tomado de pretexto para dejar de dedicarse a la casa de Dios, dedicándose a construir sus propias casas de manera ostentosa.\\
			El reino de Dios se vio detenido en su desarrollo pues sus súbditos decidieron atender sus propios deberes en vez de hacer los deberes divinos. Los tiempos les demandaban compromiso con Dios pero ese compromiso lo tomaron para con sus propias necesidades personales.\\
			Demandaba esto una palabra profética de manera urgente y Esdras todavía no había llegado a Jerusalén por lo que se le invita al pueblo a reflexionar sobre el amargo resultado que tenían por su mala actituda. Se estaban ocupadno en sus propios asuntos que les tomaba todo el tiempo pero en la obra para la cual habían sido enviados estaban ausentes.\\
			Hageo los exhorta a continuar la obra de Dios que habían abandonado, el cual había sido el verdadero motivo de su regreso.
		\item Propósito\\
			Reflexionar que las bendiciones de Dios son para los que le obedecen. Invita a que reflexionen si iban a trabajar para Dios y se les había prometido bendición si dejaban su pereza. Las consecuencias de sus pecados pasados todavía se dejaban sentir porque inclusive estando en la ciudad de Jerusalén Dios había mandado una sequía como disciplina pero el profeta Hageo les dice que ya podían ver que la lluvia se acercaba cuando ellos empezaban a obedecer y a partir del día que continuaran la obra, Dios los volvería a bendecir. En ese momento ya escaseaba el grano y aunque no era tiempo de cosecha Dios los iba a bendecir nuevamente con la bendición de la lluvia.
	\end{enumerate}
	\newpage
	\begin{subsection}{Bosquejo}
		\begin{subsubsection}{Acabando la casa de Dios (Hageo 1:1-2:9)}
			Se mencionan 4 mensajes del profeta hacia el pueblo de Dios, claramente señalados en Hageo 1:1, 2:1, 10,20. En los 4 casos se especifica que quien hablaba era Dios. Los detalles que se han tomado en cuenta para determinar la inspiración divina son precisamente la existencia de estas palabras mencionadas al principio de cada profecía.\\
			En estos textos se dan los primeros 2 mensajes, 70 años después del cautiverio el pueblo regresó a Jerusalén como había sido profetizado y en el primer grupo regresaron 50,000 exiliados aproximadamente. Una vez que se volvieron a establecer se llegaron a detener las obras como consecuencia de una carta que sus enemigos le había mandado a Artajerjes, Hageo en esos momentos empieza su ministerio en Jerusalén. El mismo pueblo tomó esto como excusa para atender en lugar sus propios interéses.\\
			Hageo predica su mensaje al pueblo y lo reprende por haber interrumpido la obra, la prioridad dada por el rey era que regresaran a edificar la casa de Dios pero su prioridad había sido reconstruir y, más aún, adornar sus casas cuando no había alguien trabajando en la casa de Dios.\\
			Ya se estaban separando de su prioridad nuevamente y esto se empezaba a notar ya que Dios no los estaba bendiciendo, en Hageo 1:6-7 dice de que forma ya no estaban siendo bendecidos por Dios.\\
			Se dieron cuenta de lo fácil que es desviarse, se arrepienten de manera sincera y vuelven a reedificar el templo después de que habían abandonado la obra. Este templo fue mucho más modesto que el de Salomón pues ya no se encontraba la presencia del arca del Dios pero aún así este templo le dio más gloria a Dios porque fue cosntruido por el verdadero esfuerzo de ellos, sin los recursos tan grandes que tuvo Salomón, sin la preparación ni las herramientas debidas el remanente reedificó el templo de Dios con sus prpias manos. En Hageo 2:7 se hace una referencia a Cristo.
		\end{subsubsection}
		\begin{subsubsection}{Alcanzando la bendición de Dios (Hageo 2:10-2:23)}
			Dios les dice que ya había acabado el periodo de su amonestación y que estaría de nuevo con ellos. En Hageo 2:20-23 se menciona un mensaje particular para Zorobabel.
		\end{subsubsection}
	\end{subsection}
\end{section}
%\end{document}


