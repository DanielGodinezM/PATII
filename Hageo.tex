%        File: Hageo.tex
%     Created: Wed Nov 20 06:00 PM 2019 C
% Last Change: Wed Nov 20 06:00 PM 2019 C
%
\documentclass[12pt]{article}
\usepackage{enumerate}
\usepackage[spanish]{babel}
\usepackage[margin=1.0in]{geometry}
\begin{document}
\begin{section}{Hageo}
	\begin{enumerate}
		\item Título\\
			Profetizó después del exilio, así como Zacarías y alaquías cuando regresó el pueblo a Jerusalén. EL títullo lleva el nombre del orfeta del que se habla en el libro. 
		\item Autor y fecha\\
			Hageo es considera como el autor del libro, en Esdras 5:1 se menciona a Hageo junto con Zacarías. E sprobalbe que Hageo haya sido una perosna Joven que incluso pudo haber nacido en el exilio y robalblemte ni conocía Jerusalén.\\
			El mismo describe en el libro que ocurrió en los años 521-486ac. que fue el tiempo en que Israle regresó de la cautividad, se considera que Hageo regresó de Babilonia a Jersualén el primer grupo que regresó junto con Zorobabel.
		\item Tema\\
			La exhortación a reconstruir la casa de Dios, las obras del templo de Dios habían sido detendias dpor muchos años debido a que los enemgisode Judá les habían estorbado si embargo ellos lo habían de tomado de pretexto para dejar de decicarse a la casa deDios y se dedicaron a cosntruir sus propias casa hermoseandolas.\\
			EN el reino de Dios se vio detenido en su desarrollo pues sus súbidtos decidieron atender uss propios deberes en vez de hacer los deberes divinos. Los tiempoes le demandaban compormiso con Dios pero ese compromiso lo tomaron para con sus propias necesidades personales.\\
			Demnadaba esto una palabra profética de manera urgente y Esdras todavía no había llegado a Jerusalén y se le invita alpueblo a reflrexionar sobre el amargo resutlado que tenían por su mala actituda. Se estaban ocupadno en sus propoois asuntos y ellos les tomaba todo el timepo pero la obra para la cual habían sido envidao estaban ausente.\\
			hageo los exhrta a continuar la obra de Dios que habían abandonado y había sido el verdadero motivo de su regros.
		\item Propósito\\
			reflexionar que las bendiciones de Dios son para los que le obedecen. Invita a que reflexionen sis iban a trabajar para Dios ys e le había prometido bendición si dejaban su pereza. Las consecuencias de sus ecados pasados todavía se dejaban sentir porque inclusive estando en la ciudad de Jerusalén Dios había mandado una sequía que debían de interpretar como disciplina de Dios pero el profeta Hageo lesd ice que ya podían ver que la lluvia se acercaba cuando ellos emézaban a obedecer ya  artir del dai que continuaran la obra Dios los volvería a bendecir, en ese momentno ya escaseaba el grano y aunque no era tiemo de cosecha dios os iba  abendecir  uevamente con la bendición de la lluvia.
	\end{enumerate}
	\begin{subsection}{Bosquejo}
		\begin{subsubsection}{Acabando la casa de Dios (Hageo 1:1-2:9)}
			Se parecian 4 mensajes del profeta hacia el pueblo de Dios, claramente señalado en 1:3,3:12:11,22:20. en los 4 casos se especifica que quen habalaba era Dios. Los detalles que se han tomado en cuenta para terminar la inspiración divina son precisamente la existencia de estas palabras.\\
			Enesos textos de dan los primeros 2 mensjaes, 70 años depsués del cautiverio el pueblo regresó a Jerusalén como había sido porfetizado y en el primer grupo regreasron 50000 exiliados aproc. Una ve que se volvieron a establecer se llegarona detenr las obras coomo consecuencia de una carta que sus enemigos que había mandado a Artajerjes, hago en esos momentos eméza su minsitero en Jerusalén. el mismo pueblo lo tomó como excusa para atender en lugar sus propios interéses.\\
			Hageo predica su mensaje alpeublo y lo reprend epor haver interrumpdo la obra, la prioridad dad por el rey era que regresearan a edificar la casad eDios pero su prioridad había sido reconstruir y más áun adornar sus casas cuando no habíaa alguien trabajando en la casa de Dios, estaban mal sus prioridades.\\
			Ya se estaban separando de su prioridad nuevamente y esto se empezaba a  notrar ya que Dios no lo estaba ebndiciendo, en vs 6-7 dice qen que forma ya noestaban siendo bendecidos por Dios.\\
			Se dieron cuenta de lo fácil que es desviarse, se arrepienten de manera siner ay vuelven a reedificar el templo después de 15 que habían abandonado la obra. Este templo fue mucho más modesto que el de Salomón pues ya no se encontraba la presencia del arca del Dios pero aún así este templo le dio más gloria a Dios porque fue cosntruido por el verdader esfuerzo de ellos,sin los recursos tan grandes que tuvo salomón, sin la preparación ni las herramientas debidas. El remanente reedificó el templo de Dios con sus prpias manos. en 2:7 se da una referencia a Cristo.
		\end{subsubsection}
		\begin{subsubsection}{Alcanzando la bendición de Dios (Hageo 2:10-2:23)}
			Dios les dice que despué de un periodo de amonestación que ya hbaía acabaod y estaría de nuevo con ellos. A  partir de los vs 20-23 se enciona un mensaje partiuclar para Zorobabel.
		\end{subsubsection}
	\end{subsection}
\end{section}
\end{document}


