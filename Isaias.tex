%        File: Isaias.tex
%     Created: Mon Nov 04 07:00 PM 2019 C
% Last Change: Mon Nov 04 07:00 PM 2019 C
%
%\documentclass[12pt]{article}
%\usepackage[spanish]{babel}
%\usepackage{enumerate}
%\usepackage[margin=1.0in]{geometry}
%\begin{document}
\begin{section}{Isaías}
	\begin{itemize}
		\item Título\\
			Le fue dado debido al profeta que tuvo las visiones que se narran en este libro. Isaías fue, quizá, el más grande de todos los profetas. Nació en Jersusalén como hijo de Amoz (distinto del profeta Amós) y se convirtió en el predicador de la corte, era primo del rey Uzías, a quien también se le nombra Azarías, por lo que era también de la descendencia de David. Isaías es considerado como el profeta de la redención ya que el tema del libro es el de la salvación por fe.\\
			Isaías es el profeta que más se menciona en el Nuevo Testamento. En este libro se menciona por primera vez de manera explícita al Mesías y las bendiciones que habrían de recibir todas las naciones a través de Él. Encontramos, además, los juicios y las promesas de Dios.
		\item Autor y fecha\\
			Se le da la autoría de todo el libro a Isaías ya que la primera parte está narrada en primera persona aunque a partir de Isaías 39 hay un cambio en el estilo del escrito y del tema. Isaías vivió en el período de los reyes Uzías, Jotam, Acaz, Ezequías y Manasés, en los años 739-686 a.C. aproximadamente. Durante el reino de Manasés, tradicionalmente se dice que él mando a asesinar a Isaías aunque eso no se menciona en la Biblia. Isaías profetizó cuando todavía existían los dos reinos pero sus profecías estaban especialmente dirigidos al reino del sur, Judá. \\
			Asiria estaba conquistando a la mayoría de los pueblos de esa época, entre ellos, conquistó al reino del norte. A Isaías se le ha nombrado como el evangelista del Antiguo Testamento, se le ha dado la prioridad de entre los demás por su capacidad de transmitir los mensajes que Dios le pidió que transmitiera a su pueblo, mostró que tenía un amplio conocimiento de las Escrituras y una entrega total a la tarea que Dios le había encomendado, también destaca por la gran cantidad de profecías mesiánicas que se encuentran en este libro.
		\item Tema\\
			La redención de Israel. Se habla de que finalmente Dios iba a redimir a Su pueblo. El nombre de Isaías significa ``Jehová es slavación''. Isaías tenía la tarea de mostrar que la salvación por medio de Jehová iba a llegar a pesar de la rebelión del pueblo.\\
			Este libro tiene muchas palabras de consuelo para el pueblo, mencionando que sería a través de un Mesías que ellos serían librados del castigo del cual eran merecedores a causa de su pecado. Dicha revelación se extendería para el futuro y para todas las naciones.\\
			 Isaías basó sus profecías en la seguridad del pacto que Dios había hecho con el pueblo de Israel. Isaías se muestra seguro sabiendo que iba a haber salvación para su pueblo y tendrían un glorioso porvenir. Isaías consideraba la fe en el santo de Israel como una condición necesaria para la vida en el mundo y siempre insistió y rogó a los gobernantes para que el pueblo tuviera fe en su Dios y confianza en el poder que tenía para satisfacer sus necesidades. Apesar de ello, Isaías tenía confianza y quería que su pueblo también confiara. Dios iba a proteger a Judá de Asiria pero en el mensaje a Judá se les exhortaba a que regresaran a la fidelidad que Dios les había mandado.
			 \newpage
			 Un remanente volvería a Dios y sería un remanente colmado de esperanza debido a sus profecías.	A esta simiente Isaías 6:13 le llama ``simiente santa''. El primer hjo de Isaías lo llamó ``Sear-jasub'' según Isaías 7:3 que significa ``un remanente volverá''. La palabra ``volverá'' hablaba exclusivamente de un remanente, algo difícil de entender ya que el pueblo pensaba que todos ellos serían salvados independientemente de su fe y de sus vidas. Más que el regresar del cautiverio físico, se refería a que tenían que volver a un estado espiritual que tenían con Dios.\\
		\item Propósito\\
			Mostrar la promesa de ese redentor.Este libro contiene referencias importantes respecto al Mesías, cosa que no se incluye en algún otro libro. Seescribió en el siglo VIII a.C. de aquél Mesías que habría de nacer 700 años después. Las profecías son exactas y nos revelan cómo iba a nacer a través de una virgen y cómo iba a morir a través de una crucifixón, todo ello profetizado por Isaías.\\
			En medio de terribles guerras por las que estaban pasando, Judá sería preservada por Dios. Isaías anunciaba la paz mundial y las grandes promesas del amor de Dios. A lo largo del libro presenta 3 imágenes del Mesías, primero como rey, luego como siervo y al final como conquistador ungido. Acerca del Mesías también se menciona en Isaías 11 que habría de ser descendiente de David.\\ 
	\end{itemize}
	\begin{subsection}{Bosquejo}
		Hay dos secciones naturales en el bosquejo del libro; de los capítulos 1-39 se habla del juicio a Judá que coincide con el número de libros del Antiguo Testamento,  en dichos capítulos se habla particularmente del juicio de Dios para con Su pueblo. La otra parte habla acerca de la salvación de Judá, de los capítulos 40-66.\\
		\begin{subsubsection}{Críticas sobre el origen del libro}
Isaías habla de que iba a haber un gran imperio en Babilonia y aún el conocía él nombre del rey, Ciro, mencionado en Isaías 45. En el tiempo en que lo escribió Isaías, Babilonia era solamente una provincia de Asiria pero Isaías predijo que se iba a convertir en la nación dominante de la región.\\
Isaías 1-39 es el precedente de lo que iba a seguir, Isaías va narrando cómo se pasa del imperio de Asiria al imperio caldeo, habló del inevitable juicio que se venía sobre Judá por seguir el camino del reino del norte, este juicio era ya inminente y esto llevó al desarrollo de la doctrina de la expiación vicaria que se observa en el Nuevo Testamento.\\
La salvación para Israel no pudo haber ocurrido sin el sacrificio vicario del cristo sufriente, una prueba más de la total autoría de Isaías la dan los mismos autores del Nuevo Testamento ya que citan ambas partes del libro. Jesús mismo cita palabras de la segunda parte de Isaías en Mateo 12:18-21 y hace énfasis en Mateo 12:17 que eran, en realidad, palabras del profeta Isaías. 
\newpage
Otra demostración del verdadero ministerio del profeta Isaías la podemos ver por la similitud que tienen sus profecías con las profecías de otros profetas que fueron contemporáneos a él como Sofonías y Jeremías, por ejemplo, que fueron profetas en el teimpo previo a la conquista de Babilonia.\\
		\end{subsubsection}
		\begin{subsubsection}{El juicio de Judá (Isaías 1-39)}
			\begin{itemize}
				\item Los problemas (Isaías 1-12)\\
					El juicio comienza con el profeta denunciando la perversidad de su propio pueblo en Isaías 1:2, 4, 6.\\
En Isaías 1:11 habla de la conducta religiosa que tenían para acercarse a Dios. Israel había apostatado y persistía en sus pecados pero quería satisfacer a Dios con un culto externo, trataban de cumplir de manera superficial con las prácticas que Dios les había ordenado que hicieran, esoo era sumamente grave ya que querían agradar a Dios de acuerdo a su propia juicio. \\
Lo importante de todas sus ofrendas era que debías de ir acompañadas de un verdadero arrepentimiento. Isaías ahora escribe en tono misericordioso para con el pueblo, pidiéndole que se arrepienta.\\
\\
Dios demanda que Su pueblo tenga pureza espiritual y moral, cualquiera que no la tuviere estaba sujeto al castigo y rechazo divino. Dios les pedía que demostraran lo que había dentro de su corazón. En Isaías 1:20,24 les advierte sobre un juicio que habría de venir en caso de que no se arrepintieran y, finalmente, en Isaías 1:25-27 se habla de la redención y restauración de su pueblo.\\
Durante el primer capítulo Isaías habla del pecado, los invita a que se arrepientan y habla de la gloria de Dios al redimir a su pueblo, haciendo del capítulo 1 un resumen de todo el libro.\\
En Isaías 2-3 se empiezan a ver los castigos que Dios le manda al pueblo por perseverar en su desobediencia. En Isaías 4:2, Isaías empieza a hablar sobre la esperanza del Mesías.\\
\\
En Isaías 5 se habla de 6 ``ayes'', un término que se refiere a condenación. Describe la vileza y el clamor que observa en la conducta de su pueblo. Confrontan a los ricos acaparadores cuya falta erradica en que nunca iban a tener lo suficiente para satisfacer su avaricia.\\
El siguiente ``ay`` es para los que adoran  los placeres y deleites sensuales, entregándose al desenfreno. En el Isaías 5:19 el ''ay`` va dirigido a los que retan a Dios de manera retórica, menciona un reto insolente a que Dios apresure su juicio y con burla le llaman el ''santo de Israel``. En Isaías 5:20 se encuentra el ''ay`` dirigido a los moralistas que la corrompen.\\
Todas las cosas para Dios son absolutas ya que Él es la verdad. Las reglas de Dios son inmutables mientras que las del hombre son cambiantes, los hombres confunden el bien y el mal para solapar una vida de pecado que llevan. En Isaías 5:21 el ''ay`` es para aquellos que son sabios de acuerdo a su propio entendimiento.
\newpage
En Isaías 6 se habla del llamado de Dios a Isaías para mandarlo a profetizar Su palabra. Esta visión profundizó la vida espiritual del profeta ya que fue una experiencia purificadora y centrada alrededor de una crisis de confesión y consagración.\\
En esta parte del libro se narra por medio de distintas etapas cómo fue el llamado del profeta, un llamado especial por la gran crisis social en Judá.
\begin{enumerate}
	\item Primera etapa\\
		El trono estaba vacante por la reciente muerte del rey Uzías, murió debido a una enfermedad a la cual la Biblia le llama ``lepra'' que contrajo como juicio divino por una actitud presuntuosa que tuvo al quemar incienso en el altar como se narra en $2^{o}$ Crónicas 26:16-20.
	\item Segunda etapa
	\item Confesión humilde\\
Ante la visión que tuvo Isaías, el profeta solamente pudo sentir su propia insignificancia. Él se confiesa como un hombre de labios inmundos y que no podía estar ante la presencia de Dios. La carnalidad de Isaías contaminaba todo su ser, ello nos ejemplifica cómo debe de sentirse un hombre que verdaderamente contempla la pureza de Dios. Isaías comprendió que aquellos entre quienes vivía era un pueblo de labios inmundos.\\
\item Purificación personal\\
	Uno de los serafines tocó a Isaías en los labios con un pedazo de carbón en Isaías 6:6-7 con el propósito de purificarlo para su tarea como profeta.
\item Llamado al servicio\\
	Dios necesitaba a una persona como instrumento para que llevara Su mensaje.
\item Entrega irrevocable\\
	Isaías responde con impulso de su corazón en amor a Dios por la tarea que le encomienda. Él habría de ser el mensajero de la ruina para su pueblo que seguía obstinado en su pecaminosidad. Sólo cuando entendemos la santidad de Dios es posible que nos veamos tal y como somos. Una de las enseñanzas más imortantes del libro es que solamente Dios es quien nos puede capacitar para Su obra.
\end{enumerate}
En Isaías 7-12 se encuentran los problemas políticos de Judá, Isaías no deja de anunciar que hay una gran esperanza para el pueblo de Dios siendo Isaías 7:14 una profecía de consuelo.\\
En Isaías 9 se habla de que habrían de pasar por varias tribulaciones, profecías queson tanto a corto como a largo plazo. Dios habló en todo momento de castigo y restauración.\\
En Isaías 10 Isaías habla de Asiria como instrumento de Su ira pues Dios es quien manda a Asiria en contra de Su Pueblo. 
\item Profecías (Isaías 13-23)\\
	Describe la profecía sobre la caída de Babilonia y menciona que los medos serían quienes sustituirían a los babilonios. En Isaías 14:12-17 hay una referencia sobre Satanás, quien está detrás del rey de Babilonia. 
	\newpage
	Lucero es otro nombre que se la da a Satanás quien tiene poder real aunque es invisible. Este pasaja narra la caída de Satanás, cosa que sucedió antes de la caída del hombre.\\
	En Isaías 21 se dan 3 profecías que anticipan la invasión de un ejército comandado por Senaquerib, habla de una región llamada ``desierto del mar'', al sur de babilonia y cerca del golfo pérsico. La tercer profecía se centra en los territorios de Arabia según Isaías 21:13, la segunda en Isaías 21:11 se refiere a Duma.\\
	En Isaías 22 se menciona una profecía acerca del valle de la visión, el lugar donde se encontraba Jerusalén.\\
En Isaías 23 se da la profecía de la destrucción de Tiro que vendría siendo muchos años después por los griegos al mando de Alejandro Magno.
\item Las tribulaciones (Isaías 24-39)\\
	Se mezclan las predicciones de aflicción que tendría Israel, en Isaías 24:20-23 se hace una descripción de la Gran Tribulación de Apocalipsis, empieza con el juicio divino que también sería para todas las naciones. Se hace una descripción más detallada acerca de los juicios de Dios, al final de Isaías 27 advierte que Dios regresará a Su pueblo de la cautividad.\\
	En Isaías 28-29 Isaías profetiza la caída del reino del norte 10 años antes de que ocurriera, describiendo como habían sido rebeldes a la Palabra de Dios. La advertencia y esperanza que aparece es para Judá, la cual a pesar de su pecado, seguía contando con la protección divina. La advertencia que estaba haciendo Isaías era para todos los tiempos ya que los corazones actuales no están verdaderamente con Dios. El hombre le teme a la condenación en lugar de a Dios.\\
	\\
	En Isaías 30-32 se le hace una advertencia a Israel por hacer una alianza con Egipto, Isaías nuevamente vuelve a nombrar que vendría una esperanza, se habla constantemente del Mesías que habría de corregir aquello que el pueblo había hecho mal, narrado en Isaías 32:1-2.\\
	En Isaías 33 hay más promesas y amonestaciones.\\
	En Isaías 34 hace más menciones del fin de los tiempos, habría un camino de santidad para ellos y el pueblo regresaría a Jerusalén con perpetuo gozo. Isaías mezcla las profecías a corto y largo plazo.\\
	En Isaías 36-39 se narran los acontecimeitnos de finales de $2^{o}$ Reyes, durante el reinado de Ezequías.
		\end{itemize}
		\end{subsubsection}
		\begin{subsubsection}{Salvación a Judá (Isaías 40-66)}
			\begin{itemize}
				\item La liberación (Isaías 40-48)\\
					Empieza con un cambio total, en vez de juicio Isaías centra sus palabras en consolación. En Isaías 40:3 empieza a hablar del mensaje que se iba a proclamar en el Nuevo Testamento y habla acerca de una voz que prepararía el camino, cosa que Juan el Bautista llevó a cabo. En este capítulo Isaías invita a Israel a no desmayar en Isaías 40:31. En Isaías 41:10 Dios les dice que no teman. En todo momento, las palabras que menciona Isaías son palabras de consolación para Judá. Se vuelve a mencionar al Mesías en Isaías 42:1. Dios le llama ahora ``mi siervo'' al Mesías.
					\newpage
					En Isaías 42:8 muestra celo por su propia persona.\\
					En Isaías 43-44 Isaías sigue exaltando a Dios. Dios los pone como testigos de que la salvación solamente puede provenir de Él. Todo Isaías 44 habla acerca de la insensatez de la idolatría.\\
					En Isaías 45 y parte final del capítulo 44 está escrita la profecía donde Dios dice cómo utiliza a un gentil para llevar a cabo su obra. Mucho antes de que exisitiera este rey, ya se había determinado que Dios lo habría de utilizar incluyendo hasta el detalle del nombre de Ciro, 150 años de que Ciro enviara a Zorobabel y a Esdras a reedificar el templo.\\
					En Isaías 46 se sigue condenando la idolatría.\\
					En Isaías 48 habla de la restauración de Israel por parte del Mesías prometido.
				\item El redentor (Isaías 49-55)\\
					Isaías exalta la esperanza de que habría de haber un redentor, cuyos sufrimientos serían propiciatorios y harían posible al justificaión del hombre delante de Dios. \\
					En Isaías 49 dice que Jehová lo llamó desde el vientre, el Mesías tendría que ser a semejanza de un hombre común que nacería a través de una mujer. Isaías 49:2 habla de que su boca sería como espada aguda, su poder estaría en su palabra eficaz y salvífica. El redentor de Israel sería menospreciado en cuanto al trato humillante al cual sería sujeto.\\
					En Isaías 50 se narra con más detalle esa humillación. Jesús cumplió Isaías 50:6 al permanecer en sumisión a la voluntad de su padre. En Isaías 52:14 se exalta al Mesías pero señala que sería desfigurado.\\
					\\
					En Isaías 53 se da una prueba irrefutable de que Jesús era el Mesías prometido, escrita en el 700 a.C. aproximadamente. El Mesías, un judío, sería grandemente exaltado y humillado. Haría un impacto al mundo entero, un día entraría este judío a la historia y maravillaría a los líderes mundiales cuando entendieran que era el mismo Dios. La predicción continúa diciendo que la mayoría del pueblo judío no ia a creer en él y que sería rechazado, sentiría un peso total de los problemas de la humanidad, experimentaría nuestros sufrimientos.\\
					En realidad era el siervo de Dios quien no merecía castigo, vendría para llevarnos de regreso a nuestra casa celestial. Isaías 53:4 menciona que las enfermedades que él llevó en realidad fueron las nuestras. Está escrito en pretérito profético ya que se conjugan en pasado pero prediciendo eventos futuros. El castigo de nuestra paz fue sobre Él, por medio de sus llagas hay salvación para nosotros. El hombre prefiere su propio camino que el camino de Dios, sus propios deseos y su propio intelecto para ser completamente egoísta. Dios se conviritó en el siervo sufriente y cargó en Su hijo las iniquidades de cada uno de nosotros, fue afligido y humillado de manera voluntaria, eso caracterizó al Cristo manso. \\
					Dos veces aparece la observación de que no abrió su boca ya que no necesitaba defenderese, ni siquiera había una acusación hacía Él. Fue llevado al altar del sacrficio, de ahí que sufriera la suerte del cordero que normalmente era sacrificado.\\
					En Isaías 53:8 describe que su captura fue un crimen judicial.
					\newpage
					Nadie pareció preocuparse por su suerte, los jueces no estaban preocupados en explicar su realidad sino que su interés era de librarse de Él. Isaías 53:9 explica que los hombres sin Dios asignaron al siervo el entierro de un injusto opresor, es decir, el deshonor lo persiguió hasta el sepulcro. José de Arimatea fue aquél hombre rico ya que su sepulcro fue donde Jesús fue sepultado. Isaías 53:10 dice que aunque el siervo no merecía morir, la voluntad de Dios fue que así sucediera.\\
					El sacrifico único del siervo provee una solución permanente al pecado del hombre. El siervo conocía con exactitud lo que era necesario para traer salvación al hombre. Isaías 53:12 menciona que la recompensa del siervo será la de gozar del botín de sus victorias espirituales. Murió, no como una víctima, imploró perdón para quienes lo estaban asesinando. En Jesucristo, este sueño profético se ha hecho realidad. Dios habló al munod por medio de un Isaías cuya boca había sido previamente cauterizada. No nos cabe la menor duda de que Isaías estaba profetizando acerca de Jesús.\\
					Los Isaías 54-55 hablan de la restauración de Israel y salvación eterna.
				\item La gloria futura (Isaías 55-66)\\
					Se hace énfasis sobre los días gloriosos que le esperan a Israel cuandos sea redimida por el Mesías. Describe una vida que deberá de llevar el pueblo redimido haciendo claro que no todos entrarían a dicho reino con el Mesías. \\
					En Isaías 61 se nombra los dos advenimientos de Cristo, Isaías 61:1-2 es citado textualmente en Lucas 4:18-21\\
					En los capítulos finales se hablan de las bendiciones para aquellos que estarán en el reino mesíanico, termina Isaías 66 con una sentencia hacia los apóstatas. En Isaías 66:24 se hace una alusión a las palabras de Jesús acerca del infierno. 
			\end{itemize}
		\end{subsubsection}
	\end{subsection}
\end{section}
%\end{document}


