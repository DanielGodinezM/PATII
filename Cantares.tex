%        File: Cantares.tex
%     Created: Wed Oct 30 03:00 PM 2019 C
% Last Change: Wed Oct 30 03:00 PM 2019 C
%
%\documentclass[12pt]{article}
%\usepackage[margin=1.0in]{geometry}
%\usepackage[spanish]{babel}
%\usepackage{enumerate}
%\begin{document}
\begin{section}{Cantar de los cantares}
	\begin{itemize}
		\item Título\\
			El libro consta de una recopilación de cantos acerca del amor humano y la interpretación literal es sumamente incorrecta. Las metáforas las debemos de entender como lo que quiso transmitir el autor, en el sentido erótico.\\
			Es necesario explicar que el sexo, como dio la instrucción Dios, siempre ha sido como bendición. Es un manual acerca del amor entre un hombre y una mujer en el matrimonio. El amor humnao está entendido como un sentimiento o atracción física. El amor del que habla la Biblia es distinto pues no es una emoción sino que es un acto de servicio, sacrifico y es una acción de entrega.\\
			Cuando se traduce al griego vemos que hay 3 palabras para referirse al amor pero en este caso la que se ocupa es \textit{eros} que es el amor físico y sensual, un amor que mal entendido puede llegar a lo obsceno y lujurioso. El \textit{eros} desaparece con el tiempo ya que es el que surge cuando hay amor a primera vista.\\
			De \textit{eros} se deriva erotismo y de este tipo de amor es del que habla el libro como una parte del amor que debe de haber en un matrimonio. 
		\item Autor y fecha\\
			Desde el principio se dice que el autor es Salomón, parece que se escribió al principio del reinado de Salomón, es probable que lo escribiera en su juventud antes de su obseción por tener más mujeres. Es una historia conmoveodra, un poema de amor entre una mujer y su amado. Describe los detalles más íntimos y sus deseos de permanecer juntos. Las relaciones sexuales se ponen bajo la perspectiva divina al igual que el matrimonio.\\
			Por todo ello, la esposa a la que se refiere debió de haber sido la primera esposa de Salomón. Una de ellas se sugiere que pudo haber sido la hija del faraón, otros dicen que pudo haber sido la sulamita que le trajeron a David en su vejez. \\
		\item Tema\\
			La unión conyugal.\\
			Se refiere a la unión sexual que hay dentro del matrimonio, que debe de haber entre 2 esposos pues así lo estableció Dios. En el matrimonio es necesario que haya una entrega recíproca, la relación sexual también es una obligación.
		\item Propósito\\
			Mostrar el compromiso que hay con una pareja dentro del matrimonio. Las relaciones sexuales no están en contra de lo bueno para el creyente, lo malo es cómo el hombre ocupa el sexo, no el sexo en sí. Si se deja de ejercer la relación sexual en el matrimonio se corre el peligro de caer en adulterio.
	\end{itemize}
	\begin{subsection}{Bosquejo}
	Se ha hecho una división que es más o menos congruente pero algunos pasajes encajan mejor en otra parte.
	\newpage
	Lleva la secuencia natural de la relación amorosa entre un hombre y una mujer.\\

		\begin{subsubsection}{El cortejo (1:1-2:17)}
			Vemos el resplandor del romance en una pareja, se puede pensar como una etapa de noviazgo. Siendo esto una poesía obviamente el lenguaje no es tan entendible. Se trata de una hombre y una mujer que esperan casarse y tener una consumación sexual de su amor. Ella habla al amado que dice que lo ha escogido para amarle profundamente. El joven recibe este mensaje y le contesta en Cantares 1:8 llamándola la más hermosa de las mujeres, notando que la sigue calificando de esta forma a lo largo de este libro.\\
			La mujer se describe acostada junto al rey. Ella dice que se pertenecen de manera mutua. Es el inicio de pertenencia para que la pareja deje de pertenecer a su famila. 
		\end{subsubsection}
		\begin{subsubsection}{La boda (3:1-5:1)}
			Los primeros 5 versículos parecen ya la escena de un matrimonio consumado, a partir del versículo 6 esta sección constituye el corazón del libro. La preparación de la pareja que concluye en la boda y consumación del acto sexual. Se ve la culminación de un verdadero amor. Se narra la belleza de la novia, el cortejo de la pareja y la boda real. Aparece el nombre de Salomón en 3 ocasiones, probalbemente como indicación de este evento que es de la clase real y que contiene lo mejor que podría haber. La mención que hace de Israel hace suponer que esa parte fue escrita antes de la separación del reino. En el Cantares 4 se describe de forma poética el acto sexual, esta larga sección está dividida en 2 partes que resume la belleza de la amada y habla de sus cabellos, sus ojos, sus dientes, entre otros.\\
			En un profundo sentimiento poético el escritor describe de forma retórica la consumación del acto sexual.
		\end{subsubsection}
		\begin{subsubsection}{El matrimonio (5:2-8:14)}
			A partir de ese momento, todo habría de ser compartido por los dos. La unión en específico se refiere a una unión en carne y a partir de aquí se mencionan también problemas cotidianos, donde empiezan los desacuerdos. También inevitable en un matrimonio bendecido por Dios es la restauración y finalmente estos desacuerdos traen un crecimiento que afirma la unión de estas dos personas.\\
			En Cantares 7-9 la dama danza para el esposo, la danza era algo importante en la cultura hebrea pues era expresión de alegría y gozo. La esposa solía bailar a su esposo para complacerlo, para que él la admirara. El marido empieza la alabanza desde la altura de sus pies y primero se fija en los pies porque es lo primero que ve, se mueve ella de forma que lo atrae a él. \\
			La mujer es quien toma la iniciativa para expresar sus sentimientos, cosa que no era muy común. En varias ocasiones vemos que se refiere a la relación sexual con un manzano y pide ser refrescada con manzanas, es decir, pide ser satisfecha sexualmente. \\
			En Cantares 8:6-7 se ve el pasaje más hermoso, la amada presenta 3 imágenes que a su vez sugiere unas características del amor, la muerte, la tumba y el fuego. Que corresponden con la fuerza, pasión y vehemencia.
			\newpage
			El libro habla de un amor profundo de una pareja que desde la creación Dios estableció entre un hombre y una mujer.\\
			Habla de una pareja de enamorados que continuamente se expresan su amor mutuo, amor erótico. Estos poemas están redactados en el más elevado estilo poético y con deslumbrantes imágenes y metáforas que dificultan su interpretación. Habla de un amor expresado en cualquier lugar, en todo momento la pareja se manifiesta su amor. No solo es el varón el que inicia las acciones también la joven que expresa sus deseos.\\
		 Dentro del plan de Dios estaba que se llenara la Tierra con su imagen. Habla tanto de la imagen del hombre como la de la mujer y de como hay una pertenencia de uno al otro y el complemento mutuo que forman.\\
			Dios hizo específicamente a la mujer con las caracterísitcas particulares de que se complementara con el hombre, además Dios les dio las caracterísitcas espirituales también.\\
			Debemos de considerar este libro con mucha importancia por la corrupción que se tiene acerca del sexo. Actualmente hay una corrupción moral que toma como pretexto el amor y se satura la mente humana con aspectos románticos como algo que es bueno, se comunica que la inmoralidad sexual está permitida y no tiene malo. La sexualidad que Dios creó y declaró como buena ya ha sido deformada y es continuamente explotada, el amor es una lujuria.\\
			El sexo y todo lo bueno que tiene es para disfrutarse dentro del matrimonio. Dentro de este erotismo encontramos pureza y santidad en un amor representado dignamente por una pareja. Las parejas honran a Dios cuando se aman y se disfrutan mutuamente. 
		\end{subsubsection}
	\end{subsection}
	\begin{subsection}{Temas clave}
		\begin{itemize}
			\item Amor\\
				Enseña a verlo desde la perspectiva correcta de Dios, a manera que se desarrolla esta relación , el impacto lo hace madurar entre la pareja. No debe de mirarse como algo casual, sin manipular a los demás para que nos amen.
			\item Belleza\\
				Se alaban mutuamente la belleza de los dos enamorados. Muestran las espontaneidad de su amor, sabemos que la belleza física se acaba, la belleza espiritual y física también es digna de alabar. 
			\item Relaciones sexuales\\
				Son un regalo de Dios, las aprueba pero las restringe para que sean utilizadas solamente dentro del matrimonio. No es por lujuria sino para el placer mutuo.
			\item Compromiso\\
				Se hace un compromiso al momento del matrimonio y de mantenerse día con día. 
			\item Problemas\\
				Con el paso del tiempo los sentimientos y la indiferencia surgieron entre los enamorados pero vemos la restauración entre esa pareja y sabemos que no puede existir esa restauración sin compromiso con la pareja. No debemos permitir que surjan muros entre las parejas.
		\end{itemize}
	\end{subsection}
\end{section}
%\end{document}


