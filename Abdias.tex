%        File: Abdias.tex
%     Created: Wed Nov 20 02:00 PM 2019 C
% Last Change: Wed Nov 20 02:00 PM 2019 C
%
\documentclass[12pt]{article}
\usepackage[spanish]{babel}
\usepackage{enumerate}
\usepackage[margin=1.0in]{geometry}
\begin{document}
\begin{section}{Abdías}
	\begin{enumerate}
		\item Título\\
			Profetizó para edom. Su nombre se le da del nombre del profeta ``Siervo de Jehová''. Es el más pequeño de los profeta menores, cosnta de un capítulos. E suna nuncio de Juicio sobre Edom formada por los descendientes de Esaú. Edom también es advertido de la justicia de dios por haber ayudado a os enemigos de Israel.
		\item Autor y fecha\\
			Una referencia a un ataque que hizo edom a Israle le da una fecha del 845ac. aporx. Una fecha en la que Edom, en lugar de ayudar a su hermano, ayudaron a los enemigos de Judá. Este libro es el más antiguo de esta sección. Era el tiempo en el que estaba reinando Atalía en el reino del sur, Abdías fue originiario del reino del sur aparentmente, contemporáneo de elías y Eliseo.
		\item Tema\\
			Juicio sobre el pecado de Edom. Edom fue víctima de su propio carácter. Dios le muestra todas su maldad como un pueblo violento, Abdías le die al pueblo de Dios que no se preocupen por los que se alegrna sus probelasm, se juntan con otros para aatacar a Dios pues Él se encargaría de ellos. Dios quiere que edom sepa que la ngustia que llegó sobre Judá también llegaría sobre ellos. Lo mismo que le hicieron a Judá es lo que iban a sufir.
		\item Propósito\\
			El juicio de Dios contra Edom fue un cumpimiento de la promesa que le hizo a Abraham ya que los habían maldicho. la respuets ade Dios por su conducta aberrante en ontra de Su pueblo. En el vs 15 se les dice que les haría de acuerdo a lo que habían hecho, retribución justa en contra de Edom. 
	\end{enumerate}
	\begin{subsection}{Bosquejo}
		\begin{subsubsection}{La humillación de Edom (Abdías 1-14)}
			EL mensaje de Abdías etsá dirigido principalmente a la tierra habitada por los descedneiente de Esaú (Edom). Probalbmente tenían el mismo idioma que el pueblo de Dios pero su religión a habían cambiado pues era un pueblo politeísta, desde el regreso del pueblo de Dios de Egipto hubo muchas diferencias y aquí se narran las consecuencias de dichas difereencias. Los edomitas ayudaron al enemigo de Judá por loq ue Abdía predice el juicio de Edpm por esta actitud y en Jueces se menciona que hasta los habían esclavizado en un momento.\\
			La destrucción total dse dio con Tito el general que destruyó Jerusalén, los que defendían la torre Antonia era los edumeos /edomitas) que fueron destruidos porlos romasnos. En el vs 10 el profeta le da sentencia a edom, lo mismo que le sucedió a Judá le sucedería a edom.Dios reprende al pueblo por sus actitud hacia su hermano de sangre lo que le trajo el juicio de Dios.
		\end{subsubsection}
		\begin{subsubsection}{La exaltación de Judá (Abdías 15-21)}
			Decara en vs 17 que en el Sion habría un remantente que recuperaría sus pertenencias y se quedaría tambien con las pertenencias de Edom. El cumplimiento parial de la profecía se dio en el 582 cuando los caldeos destruyen edom posteriormente,l uqe hbaía quedado de los edomitas en el periodo intertestamentario Judas macabeo los derrotó y los obligó a adoptar las costumbres judías y posteriormente fueron eliminados por los romanos.
		\end{subsubsection}
	\end{subsection}
\end{section}
\end{document}


