%        File: Abdias.tex
%     Created: Wed Nov 20 02:00 PM 2019 C
% Last Change: Wed Nov 20 02:00 PM 2019 C
%
%\documentclass[12pt]{article}
%\usepackage[spanish]{babel}
%\usepackage{enumerate}
%\usepackage[margin=1.0in]{geometry}
%\begin{document}
\begin{section}{Abdías}
	\begin{enumerate}
		\item Título\\
			Abdías, cuyo significado es ``siervo de Jehová'', fue un profeta que realizó su ministerio para Edom. El nombre del profeta se ocupa para nombrar al libro el cual es el más pequeño de los profeta menores ya que consta de un sólo capítulo. El libro es un anuncio de juicio sobre Edom que fue formada por los descendientes de Esaú. Edom también es advertido de la justicia de Dios por haber ayudado a los enemigos de Israel.
		\item Autor y fecha\\
			No se conoce mucho acerca del profeta ya que el mismo nombre es usado por distintos hombres en el Antiguo Testamento por lo que cualquier referencia que se tiene no parece referirse a este profeta.
			Una referencia que hace el autor acerca de un ataque que hizo Edom a Israel le da una fecha de escritura del año 845 a.C. aproximadamente, una fecha en la que Edom, en lugar de ayudar a su hermano, ayudaron a los enemigos de Judá. Este libro es el más antiguo de esta sección ya que era el tiempo en el que estaba reinando Atalía en el reino del sur, Abdías fue originiario del reino del sur aparentemente, contemporáneo de Elías y Eliseo.
		\item Tema\\
			Juicio sobre el pecado de Edom. Edom fue víctima de su propio carácter. Dios le muestra todas su maldad como un pueblo violento, Abdías le dice al pueblo de Dios que no se preocupen por los que se alegran por sus problemas pues Dios se encargaría de llevarles justicia. Dios quiere que Edom sepa que la angustia que llegó sobre Judá por su culpa también llegaría sobre ellos, es decir, lo mismo que le habíasn hecho a Judá es lo que iban a sufir.
		\item Propósito\\
			El juicio de Dios contra Edom fue un cumplimiento de la promesa que le hizo a Abraham ya que los habían maldecido, ese juicio sería la respuesta de Dios por la conducta aberrante que habían tenido en contra de Su pueblo. En Abdías 15 es en donde se les dice que Dios les haría de acuerdo a lo que habían hecho, retribución justa en contra de Edom. 
	\end{enumerate}
	\begin{subsection}{Bosquejo}
		\begin{subsubsection}{La humillación de Edom (Abdías 1-14)}
			El mensaje de Abdías está dirigido principalmente a la tierra habitada por los descendientes de Esaú (Edom). Probablemente tenían el mismo idioma que el pueblo de Dios pero su religión la habían cambiado pues era un pueblo politeísta, desde el regreso del pueblo de Dios de Egipto hubo muchas diferencias entre ambos pueblos y aquí se narran las consecuencias de dichas difereencias. Los edomitas ayudaron al enemigo de Judá por lo que Abdías predice el juicio para Edom por esta actitud. Las diferencias entre ambos pueblos fueron tales que en Jueces incluso se menciona que habían esclavizado a Israel por un momento.\\
			La destrucción total se dio con Tito, el general que destruyó Jerusalén.
			\newpage
			Los que defendían la torre Antonia eran los edomitas que fueron destruidos porlos romanos. En Abdías 10 el profeta le da sentencia a Edom, lo mismo que le sucedió a Judá le sucedería a ellos. Dios reprende al pueblo por sus actitudes hacia su hermano de sangre por medio de un juicio divino.
		\end{subsubsection}
		\begin{subsubsection}{La exaltación de Judá (Abdías 15-21)}
			Declara en Abdías 17 que en Sión habría un remanente que recuperaría sus pertenencias y se quedaría también con las pertenencias de Edom. El cumplimiento parcial de la profecía se dio en el año 582 a.C. cuando los caldeos destruyen Edom. En el periodo intertestamentario, el remanente de los edomitas fue derrotado por Judas Macabeo y fueron obligados a adoptar las costumbres judías para ser después eliminados por los romanos.
		\end{subsubsection}
	\end{subsection}
\end{section}
%\end{document}


