%        File: Zacarias.tex
%     Created: Wed Nov 20 06:00 PM 2019 C
% Last Change: Wed Nov 20 06:00 PM 2019 C
%
\documentclass[12pt]{article}
\usepackage{enumerate}
\usepackage[spanish]{babel}
\usepackage[margin=1.0in]{geometry}
\begin{document}
\begin{section}{Zacarías}
	\begin{enumerate}
		\item Título\\
			Ministerio delespues del exilio, título del porfeta del que trata. Zacarías fue de los profetas que regrsó del exilio y tuvo el mismo oficio de su padre.\\
			Se formó un grupo de eruditos que restituyeron todas las escrituras que habían sido robabdas o destruidas y estaba en el templo formando Esdras el canon del AT. A este grupo se llamó la Gran Sinagoga y perteneció Zacarías.
		\item Autor y fecha\\
			Zacarías regresó juntos con Hageo en el primer viaje de los exiliados por lo que lo podemos ubicar en el mismo periodo que hgeo. A Zacarías se le ha dado la autoría del libro.
		\item Tema\\
			Mensajes que Dios le dio después del exilio. da 10 mensajes que recibió en visiones 5 meses después del reinicio de la cosntrucción del templo, fueron de gran ánimo ya que trataban de ala purificaión de Israel y su restauración final. El profeta amonesta a su puelbo por su pecado y sus ritles y nuevamente llamama al pueblo al arrepentimiento y les da la esperanza de la restauración y les explica el futuro del pueblo de ISrael y que n habría necesidad de ayuno ni de luto porque para ese tiempo sólo habría gozo absoluto. 
		\item Propósito\\
			Anunciar las bendiciones futuras por medio del Mesías prometido. En aquel día esperado el profeta dice que sucederán grandes cosas, el pecado y la impueraza serán erradicados por un manantial de a casa de David que limpiaría a Jerusalén de forma que presentará  a Dios la ciudad como una esposa sinmancha ni arrugas. Es una dulce mnsja del evangelio que trae perdón. EN el NT nos mjeysta que es el señor Jesucritos,el hombre ue creado para adorarr a dios pero se ve que el hombre se envaneció y quiso ser igual a Dios, se hundió en el pecado con actituedes rebeldes en contra del creador. recibirían un castigo por su maldad y sería sólo en la cruz del calvario donde la sangre ddel cordero de Dios se derramaría para etner erendición, la humnaidad encontraŕia consuelo y el lan de Dios se cumpliríae lṕueblo redimidio eterndía jubilo y prolcamaría la gloria de Dios.
	\end{enumerate}
	\begin{subsection}{Bosquejo}
		\begin{subsubsection}{Diez visiones (Zacarías 1-6)}
			Aparecen las 10 visiones del profeta. Aparentemente todas fueron en la misma noche que tenía como fianidlad levantar el ńimo del puelbo judío para que siguieran edificando la casa de Dios. Este libro es un libro donde se narran varias visiones apocalítptocas y vveremos profecías mesiánicas difíciles de enter debido s u gran simbolimso. Zacarías se centra en el consuelo para el pueblo, después de que habían decidido contunar con la recontrucción del templo por la profecía de Hageo.\\
			Zacarías haroa les habala de las futuras bendiciones que Dios tenía para elllos, las 10 visioenes les muestraa información acerca de cómo s erá el futuro de ISrael, los profetas ven cosas que no podían entender ni comprender y entonces dice que un ángel le ayuda a interpretar el significado de dichas visiones.\\
			Al prinpio las visones le reuerdan la histria de Israel de manera cronológica hata que ve qlel adveimiento del mesías en elr eino mileniua. Dice qu el Mesías tenái un comrómiso con el pueblo a corto y largo plazo.\\
			En el cp 6 ve 4 carros tirados por caballos de distintos colores que salen tre 2 msotañas de bronce. Esto caballos habían sido enviados por Dios para ejecutar su juicio en todo el mundo, los del norte tenían la tarea de ejecutar el acastigo sore el pueblo cladeo.
		\end{subsubsection}
		\begin{subsubsection}{Cuatro mensajes (Zacarías 7-8)}
			4 Respuestas se dan a una pregunta que hizo el pueblo a Zacarías, en 7:3 se ve la pregutna del pueblo. se referían a quellas situaciones religiosas que acostumbraban hacer. Zacarías le snfroma que el ayuno más que por seguir el ritual de la lye debía de ser una manifeatcio  externa de su arrepentimiento.\\
			la segunda respuesta era el por qué Dios ya no atendía sus oraciones, de la misma forma que ya no atendía las peticiones de los profetas. Les reitera que a pesar de todo seguían siendo s pueblo y Él seguía siendo su Dios así ahra despuees de arrepentirse volvería a bendecirlos, 8:7-8. El pueblo debía de scuhar a los mensjaeros que Dios les estaba enviando.
		\end{subsubsection}
		\begin{subsubsection}{Dos cargas Zacarías 9-14}
			Se refiere a los 2 advenimientos del MEsías relacionados con el pueblo gentil. El MEsías lo vería el pueblo de Judá y los gentiles también, la rpimera vez que lo verían sería en la primera venida cuando Jesúas serías rechazado, 9:9.También habla de la segunda benido en 14:3-4 cuand el Mesías sería aceptado.\\
			Podemos encontrar varios textos mesiánicos como 6:12, 9:9, 11:12 (traición de Judas), 12:10 (herida del soldado cuando ya estaba crucificado).
		\end{subsubsection}
	\end{subsection}
\end{section}
\end{document}


