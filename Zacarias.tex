%        File: Zacarias.tex
%     Created: Wed Nov 20 06:00 PM 2019 C
% Last Change: Wed Nov 20 06:00 PM 2019 C
%
%\documentclass[12pt]{article}
%\usepackage{enumerate}
%\usepackage[spanish]{babel}
%\usepackage[margin=1.0in]{geometry}
%\begin{document}
\begin{section}{Zacarías}
	\begin{enumerate}
		\item Título\\
			Zacarías, que significa ``Jehová recuerda'',  fue un sacerdote que ejerció su ministerio como profeta después del exilio, el título del libro se le da a causa del profeta del que trata. Zacarías fue de los profetas que regresó del exilio y tuvo el mismo oficio que su padre.\\
			Se formó un grupo de eruditos que restituyeron todas las escrituras que habían sido robadas o destruidas y estos eruditos se juntaron en el templo, liderados por Esdras, para formar el canon del Antiguo Testamento. A este grupo se le llamó la Gran Sinagoga del cual formó parte Zacarías.
		\item Autor y fecha\\
			Zacarías regresó junto con Hageo en el primer viaje de los exiliados por lo que lo podemos ubicar en el mismo periodo que Hageo. A Zacarías se le ha dado la autoría del libro.
		\item Tema\\
			Mensajes que Dios les dio después del exilio. Da 10 mensajes que recibió en visiones 5 meses después del reinicio de la construcción del templo, fueron de gran ánimo ya que trataban de la purificación de Israel y su restauración final. El profeta amonesta a su pueblo por su pecado y sus rituales y nuevamente llama al pueblo al arrepentimiento. Les da la esperanza de la restauración y les explica el futuro del pueblo de Israel y que no habría necesidad de ayuno ni de luto porque para ese tiempo sólo habría gozo absoluto. 
		\item Propósito\\
			Anunciar las bendiciones futuras por medio del Mesías prometido. En aquel día esperado el profeta dice que sucederán grandes cosas, el pecado y la impureza serán erradicados por un manantial de la casa de David que limpiará a Jerusalén de forma que se le presentará a Dios como una esposa sin mancha ni arrugas.\\
			El hombre fue creado para adorar a Dios pero se ve que se envaneció y quiso ser igual a Dios, se hundió en el pecado con actitudes rebeldes en contra del creador. Recibirían un castigo por su maldad y sería sólo en la cruz del calvario donde la sangre del cordero de Dios se derramaría para tener redención, la humanidad encontraría consuelo y el plan de Dios se cumpliría, el pueblo redimido tendría júbilo y proclamaría la gloria de Dios.
	\end{enumerate}
	\begin{subsection}{Bosquejo}
		\begin{subsubsection}{Diez visiones (Zacarías 1-6)}
			Aparecen las 10 visiones del profeta. Aparentemente todas fueron en la misma noche que tenía como finalidad levantar el ánimo del pueblo judío para que siguieran edificando la casa de Dios. En este libro se narran varias visiones apocalípticas y veremos profecías mesiánicas difíciles de entender debido su gran simbolismo.
			\newpage
			Zacarías se centra en el consuelo para el pueblo después de que habían decidido continuar con la recontrucción del templo por la profecía de Hageo.\\
			Zacarías ahora les habla de las futuras bendiciones que Dios tenía para ellos, las 10 visiones les muestra información acerca de cómo será el futuro de Israel, los profetas ven cosas que no podían entender ni comprender y entonces dice que un ángel le ayuda a interpretar el significado de dichas visiones.\\
			Al principio las visones le recuerdan la historia de Israel de manera cronológica hasta que ve el advenimiento del Mesías en el reino milenial. Dice que el Mesías tenía un compromiso con el pueblo a corto y largo plazo.\\
			En Zacarías 6 ve 4 carros tirados por caballos de distintos colores, que salen de 2 montes de bronce. Estos caballos habían sido enviados por Dios para ejecutar su juicio en todo el mundo, los del norte tenían la tarea de ejecutar el castigo sobre el pueblo caldeo.
		\end{subsubsection}
		\begin{subsubsection}{Cuatro mensajes (Zacarías 7-8)}
			Cuatro respuestas se dan a una pregunta que hizo el pueblo a Zacarías, en Zacarías 7:3 se ve la pregutna del pueblo. Se referían a aquellas situaciones religiosas que acostumbraban hacer, Zacarías les informa que el ayuno más que por seguir el ritual de la ley debía de ser una manifestación externa de su arrepentimiento.\\
			La segunda respuesta era el por qué Dios ya no atendía sus oraciones de la misma forma que ya no atendía las peticiones de los profetas. Les reitera que a pesar de todo seguían siendo Su pueblo y Él seguía siendo su Dios así ahora después de arrepentirse volvería a bendecirlos, Zacarías 8:7-8. El pueblo debía de escuhar a los mensajeros que Dios les estaba enviando.
		\end{subsubsection}
		\begin{subsubsection}{Dos cargas Zacarías 9-14}
			Se refiere a los 2 advenimientos del Mesías, relacionados con el pueblo gentil. Al Mesías lo vería el pueblo de Judá y los gentiles también, la pimera vez que lo verían sería en la primera venida cuando Jesús sería rechazado, Zacarías 9:9. También habla de la segunda venida en Zacarías 14:3-4 cuando el Mesías sería aceptado.\\
			Podemos encontrar varios textos mesiánicos como 6:12, 9:9, 11:12 (traición de Judas), 12:10 (herida del soldado cuando ya estaba crucificado).
		\end{subsubsection}
	\end{subsection}
\end{section}
%\end{document}


