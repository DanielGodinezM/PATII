%        File: Daniel.tex
%     Created: Wed Nov 13 08:00 PM 2019 C
% Last Change: Wed Nov 13 08:00 PM 2019 C
%
%\documentclass[12pt]{article}
%\usepackage{enumerate}
%\usepackage[spanish]{babel}
%\usepackage[margin=1.0in]{geometry}
%\begin{document}
\begin{section}{Daniel}
	\begin{itemize}
		\item Título\\
			Le es dado por el personaje principal, uno de los primeros cautivos procedentes de Judá, originario de una familia noble de la triu de Judá. Fue llevado siendo aún muy joven en la primera deportación y posteriormente se conviritó en profeta de Dios.
		\item Autor y fecha\\
			El mismo profeta fue el que scribió el libro durante sus cautiverio en Babilonia del cual aparentemente nunca salió. Se dice que Daniel fue el autor debido a distintos pasajes, entre ello se encuentran Daniel 8:27, 9:2 pues literalmente escribe ``yo, Daniel\ldots''\\
			Fue un personaje distinguido en 2 grandes imperios en los que vivió, el imperio babilónico y el imperio medo-persa. En ambos imperios ocupó cargos políticos de gran relevancia. Debido a la gran sabiduría que demostró fue reconocido por los reyes a los cuales sirvió. Se narra cómo sirvió a Nabucodonosor y al Beltsasar, Darío y Ciro del imperio Medo-persa. Fue contemporáneo de Jeremías y Ezequiel.\\
		También fue contemporáneo de Esdras y Zorobabel pero Daniel nunca regresó a Jerusalén. El libro fue escrito en los años 236-530 a.C. aproximadamente.\\
	\item Tema\\
		La relación entre Israel y los imperios del mundo. Registran grandes imperios de la historia universal, Daniel describe la sucesión que han tenido los grandes poderes mundiales hasta el tiempo del gran reino milenial. Daniel menciona cómo el imperio medo-persa los libró del imperio babilónico dejándolos reconstruir Jerusalén y el templo, después menciona el imperio griego que sometió a Jerusalén y que un personaje griego profanó el templo en Jerusalén y finalmente habla del imperio romano en el 70 d.C. que fue el imperio que destruyó Jerusalén y diseminó a los judíos por todo el mundo. En la vision de Daniel 7 las bestias simbolizan 2 poderes y los imperios mundiales y los cuernos simbolizan los líderes de naciones más pequeñas.
	\item Propósito\\
		Mostrar el imperio de los últimos tiempos. El reino eterno de Dios será gobernado por el cordero de Dios desde lo que se llama la Nueva Jerusalén. El gobierno de Cristo será sobre toda la Tierra, más allá que cualquier imperio anterior. Al establecer el gobierno mundial Dios cumple su propósito de que el hombre gobernara sobre toda la tierra, el segundo Adán es quien cumpliría esta meta. La profecía de Daniel dice que el hijo del hombre recibirá al final de los tiempos gloria y un reino en donde todos habrán de servirle.
	\end{itemize}
	\begin{subsection}{Bosquejo}
		\begin{subsubsection}{Las profecías del dominio gentil (arameo)}
			\newpage
			\begin{enumerate}
				\item Daniel (Daniel 1)\\
					Se ve la historia personal de Daniel, era un muchacho judío de aproximadamente 15-17 años cuando fue deportado en el primer exilio de Judá a Babilona con 3 muchachos de su edad, Ananías, Misael y Azarías. El rey caldeo que reinaba en Babilonia, Nabucodonosor, quería tener un imperio que gobernara toda la tierra por lo que intentó sacar provecho de cada pueblo que conquistaba. Seleccionaba cuidadosamente a cada cautivo que podía presentarle un servicio importante durante su reino, Daniel y sus 3 amigos fueron escogidos para que fueran educados en el palacio del rey y para que después sirvieran al imperio. Estas circunstancias tenían un propósito por parte de Dios, Él utilizó la idea de Nabucodonosor para transmitir un mensaje escatológico a la posteridad. Fueron entregados para su custodia al jefe de los eunucos por lo que probablemente fueron convertidos en eunucos, nunca se menciona que Daniel se haya casado.\\
					En el resto del capítulo se ve que Daniel y sus amigos tienen el deseo de no contaminarse con la dieta de los caldeos siendo mejores que sus magos y astrólogos en el reino. 
				\item Nabucodonosor (Daniel 2-4)\\
					Uno de los grandes reyes de la historia, fue el rey de babilonia durante los años 605-562 a.C. Antes de ser rey fue un líder militar espectacular y derrotó a un faraón. Él mismo conquistó a Jerusalén cuando su padre era todavía rey. Cuando llegó al trono transformó a Babilonia en la ciudad más rica y más hermosa de oriente, construyó grandiosos edificios y un palacio real fabuloso. Célebres fueron en el mundo antiguo sus jardines colgantes, fueron consideradas como unas de las 7 maravillas del mundo. A partir de Daniel 2 se menciona que Nabucodonosor, a los dos 2 años de ser rey, tuvo el sueño de que sería instrumento de Dios para mostrar la sabiduría de Daniel. Ante la ineficacia de los magos para interpretar dichos sueños, Daniel actúa al revelarle al rey el sueño que había tenido y que lo tenía inquieto. El rey había mandado matar a sus adivinos y astrólogos pero Daniel les salva la vida, cuando Daniel dijo que iba a interpretar el sueño del rey ni si quiera sabía cómo lo iba a interpretar. Le pidió a sus compañeros que oraran para que en su misercordia le revelara el sueño del rey.\\
					\\
					En Daniel 2:25 se ve que Daniel es llevado ante el rey, le relata el sueño que no recordaba y le revela la interpretación de dicho sueño. Daniel le dice que hay un Dios en los cielos que revela todos los misterios, Dios le mostró a Daniel la aparición cronológica de los imperios gentiles más grandes hasta llegar al último imperio, el reino de los cielos. A través de la estatua muestra la sucesión de los imperios futuros. El rostro de la estatua era de plata representando al imperio medo-persa. El imperio griego está representado por los muslos de bronce y el imperio romano representado al final.\\
					En Daniel 2:44 dice que ese reino establecido por Dios nunca sería destruido, el reino del Mesías, reino que nunca sería reemplazado.\\
					En Daniel 3 se narra la historia de otra estatua, por edicto de la arrogancia de rey. Manda a hacer una estatua de oro para que todos los habitantes del reino se postraran y lo adoraran en Daniel 3:4-5.
					\newpage
					Ananías, Misael y Azarías que ni adoran ni se postran a la imagen y son acusados ante el rey de esa desobediencia, el castigo era que debían de ser echados a un horno, fueron llevados a la presencia del rey y lo desafiaron contestándole que Dios los podía liberar. Los amigos mostraban confianza en Dios, no le tenían miedo al hombre más poderoso del mundo. En Daniel 3:18 ellos le dicen al rey que en caso de que Dios no los salvara aún así no le servirían a sus dioses.\\
				Estos hombres fieles fueron echados al horno y Dios los protegió, el fuego no les hizo daño. El ejemplo de fidelidad de los hombres hizo que el rey reconociera al Dios verdadero y el rey dignificó a estos hombres.\\
				\\
				Ahora el rey tiene na visión de un árbol de gran tamaño que es mandado cortar, sus adivinos y astrólogos no pueden interpretar el sueño y Daniel queda perturbado por la interpretación que Dios le da del sueño, Dios muestra la disciplina que Dios le iba a mandar al rey. El portentoso árbol era el rey mismo y aparece un santo descendiendo del cielo con la orden de que el árbol fuera cortado dejando una parte pequeña del árbol para que no muriera completamente. Ese trozo  de árbol se relaciona con las bestias del campo y dice que su corazón de hombre sería cambiado por uno de bestia. Un año después de la visión, estaba paseando el rey por su palacio y una voz del cielo llegó a sus oídos, a partir de ese momento, el rey comenzó a actuar como animal comiendo hierba del campo. En Daniel 4:34, le fue dada de vuelta la razón glorificando al altísimo como el rey de todas las edades.\\
				Los capítulos 5 y 6 ocurren después de los capítulos 7 y 8.
				\item Beltsasar (Daniel 5)\\
					Beltsasar, hijo de Naponido, reinó en el año 546 a.C. y en Daniel 5 se narra que organizó una fiesta en la cual parece que habían sobrepasado el consumo de vino, el rey manda a traer los utensilios robados del templo de Dios en Jerusalén y el rey los ocupa en la orgía que estaba teniendo. Este acto de profanación le trajo juicio, en Daniel 5:5 aparece una mano que comenzó a escribir en la pared. Se espantó el rey y mandó a traer a los magos para que le descifraran la interpretación del escrito. Daniel lo interpreta diciéndolde que había ofendido a Dios y le recuerda que la vida de todo hombre está en manos del altísimo. La interpretación de la escritura en la pared era que Dios le ponía fin a su reino. Dios, en ese momento, entregaba el reino de Babilonia al imperio medo-persa, ese mismo día Darío conquistó el reino de Babilonia y fue asesinado Beltsasar.
				\item Darío (Daniel 6)\\
					Rey persa distinto que procedió a Ciro. Daniel debía de ser probado ahora, los sátrapas no estaban de acuerdo que un extranjero como él estuviera en un cargo superior al de ellos. Hicieron que se promulgara un decreto indicando que cualquiera que adorara a alguien distinto del rey sería echado a un foso de leones, Daniel hizo caso omiso y siguió adorando al Dios de Israel como siempre. Daniel fue acusado ante el rey quien se vio obligado a arrojarlo al foso de los leones aunque lo apreciaba mucho y esperando que Dios lo salvara. Dios manda cerrar la boca de los leones, Daniel sale ileso de la prueba por lo que el rey castiga a sus acosadores y los manda al mismo foso. El rey terminó escribiendo otro decreto que anulaba al primero.
					\newpage
				\item Las bestias (Daniel 7)\\
					Empiezan a manifestarse las visiones de carácter escatológico empezando desde el imperio babilónico, el medo-persa, el griego, romano y el imperio del hijo del hombre. En la visión se muestra y representan reinos en formas de bestias como un león con ojos de águila que es el reino de Babilonia. Elleón alado se ha encontrado en las ruinas de Babilonia dado que era el emblema del imperio. Después aparece un oso que representa al reino medo-persa, un leopardo con 4 alas y cabezas del reino griego y la bestia más espantosa, con 10 cuernos representando 10 reyes de los cuales surgía otro cuerno pequeño con ojos y boca que hablaba grandes cosas. Después de esta visión, Daniel ve un anciano sobre su trono vestido de blanco y con pelo como de lana. El ser actuaba como juez y le fueron abiertos unos libros. Al cuerno pequeño se le identifica como el anticristo.\\
De Daniel 7:16-28 un ángel le explica la visón.-
			\end{enumerate}
		\end{subsubsection}
		\begin{subsubsection}{Profecía del destino de Israel (Daniel 8-12)}
			A partir de Daniel 8 el profeta escribe en hebreo ya que esa porción del libro está dirigida exclusivamente a los judíos.
			\begin{enumerate}
				\item Los cuernos (Daniel 8)\\
					En esta visión se muestra el poder del hombre a través de los cuernos de los animales, los cuernos son cada vez más potentes y grandes. Habla de los imperios medo-persa y griego.
				\item Las setenta semanas (Daniel 9)\\
					Daniel muestra su pesar por la condición del pueblo de Dios. Daniel sabía que Dios les iba a levantar el castigo y en Daniel 9:19 le habla a Dios para que los perdonara y no tardara. Aún estaba orando cuando le llegó la respuesta de Dios, 70 semanas determinadas sobre el pueblo para terminar su pecado. La profecía aún no se ha cumplido ya que es para los tiempos escatológicos.
				\item La gloria de Dios (Daniel 10)\\
					Daniel observa en su visión de la gloria del Dios altísimo, menciona la característica del ser revelado. La intención de la revelación era informarle sobre las cosas en un futuro lejano.
				\item La bestia (Daniel 11)\\
					Daniel narra los enfrentamientos entre los distintos imperios por la supremacía del poder y se sigue mostrando el poder del hombre a través de cuernos. Los cuernos pequeños son historias entrelazadas con diferencia significativa, uno de los cuernos sería un hombre poderoso que blasfemaría el nombre de Dios en un futuro cercano, el otro sería el anticristo.
				\item La gran tribulación (Daniel 12)\\
					Se narran aspectos de lo que se llama la Gran Tribulación, Daniel habla del destino final del pueblo de Dios empezando desde Daniel 12:1.
			\end{enumerate}
			Daniel siempre sigue la voluntad de Dios desde su juventud, apegado a las Escrituras y en su vejez consagró todo su ser a Dios. En medio de la facinación de una corte oriental conservó su integridad.
		\end{subsubsection}
	\end{subsection}
\end{section}
%\end{document}


