%        File: Job.tex
%     Created: Mon Oct 21 09:00 PM 2019 C
% Last Change: Mon Oct 21 09:00 PM 2019 C
%
%\documentclass[12pt]{article}
%\usepackage[margin=1.0in]{geometry}
%\usepackage{enumerate}
%\usepackage[spanish]{babel}
%\begin{document}
\begin{section}{Job}
	\begin{itemize}
		\item Título\\
			El libro lleva el nombre del personaje principal de la historia quien fue un personaje histórico aunque se considera que no fue hebreo ya que se indican los nombres de sus hijos e hijas, cosa que no era costumbre dentro del pueblo hebreo.\\
			Su nombre significa ``perseguido'' en hebreo y por lo relatado en el libro se sabe que vivió en un periodo entre el diluvio y los días de Moisés. Como vivió 140 años se sitúa comunmente en la era patriarcal, es decir, en los tiempos de Abraham.
		\item Autor y fecha\\
			No se dice explícitamente quién fue el autor del libro ni la fecha en la que se escribió. Las hipótesis más aceptadas son que fue escrito por Job o inclusive por Moisés pero no se sabe con certeza debido a que la historia llega a narrar acontecimientos fuera de la Tierra.
		\item Tema principal\\
			La soberanía de Dios pues se ve revelada a lo largo del libro. Dios permite, dentro de Su voluntad, que le ocurran sucesos tristes y de dolor a sus hijos.
		\item Propósito\\
			Reconocer la total dependencia y sumisión que tenemos hacia la voluntad de Dios.
	\end{itemize}
	\begin{subsection}{Bosquejo}
		\begin{subsubsection}{El prólogo (Job 1:1-5)}
Se narra la estructura que tenía la familia de Job; tenía una esposa, 7 hijos y 3 hijas. Desde Job 1:1 se dice que Job era un hombre recto y temeroso de Dios, precisamente por dicha rectitud y sumisión a Dios es que Él lo bendice con muchas riquezas.\\
En Job 1:5 se narra además que Job era un hombre piadoso ante su familia pues pensando aún en el pecado de sus hijos era que él ofrecía sacrificios para que sus pecados fueran prometidos.
		\end{subsubsection}
		\begin{subsubsection}{El drama (Job 1:6-42:6)}
			Desde el primer versículo del drama se menciona que Satanás afirma que la fidelidad de Job fue solamente por la prosperidad que Dios le había dado pero que esa fidelidad decaería ante la tribulación. Dios, por el contrario, conocía a la perfección la rectitud de Job por lo que permite que Satanás pruebe a Job con la condición de que no corriera peligro su vida.\\
\\
Satanás comienza su trabajo a partir de Job 1:13 pero a pesar de ello se ve que la fidelidad de Job no cede. Satanás continúa y para el final de capítulo Satanás le quita a Job todas sus pertenencias a pesar de que su rectitud no se quebranta.\\
Posteriormente, en Job 2:7 Dios permite que Satanás le quite su salud por lo que Job contrae una sarna maligna. Después de que contrae la sarna, en Job 2:9 Job tiene discusiones con su esposa y posteriormente con 3 de sus amigos. 
\newpage
Su esposa, su ayuda idónea, le falló cuando más la necesitaba.\\
Por parte de sus amigos, su respuesta fue la de dudar acerca de la rectitud de Job ya que consideraban la situación de Job como una disciplina por parte de Dios.\\
\\
	A partir del capítulo 4 se narra la discusión que inicia con sus amigos, empezando con  Elifaz. Elifaz le repite a Job que todo lo que le había pasado era debido a que Dios estaba disciplinando sus pecados.\\
En el capítulo 8 se narra que su siguiente amigo, Bildad, directamente lo acusa de ser un hipócrita, cuestionando su rectitud.\\
En el capítulo 11, su tercer amigo, Zofar, le dice aún que Dios no lo ha castigado como en realidad lo merece.\\
\\
Vemos que sus amigos aparentaban querer ayudarle a Job pero en realidad solamente lo desanimaron más pues Job se vio en la necesidad de defenderse de las acusaciones de sus propios amigos, pues él sabía que siempre había actuado con el propósito de satisfacer la justicia de Dios, Job va más aún llamándose justo a él mismo, olvidando que en realidad, delante de Dios nadie puede considerarse justo.\\
\\
 En el capítulo 32 aparece Eliú, un joven, hablando de manera muy tajante pues añade una nueva perspectiva sabia al tema a pesar de ser el más joven. De manera arrogante le hace ver a Job que Dios no es injusto pues justamente Dios es misericordioso y siempre tiene un propósito para su acciones.\\
\\
Ésto nos refleja que, en realidad, ningún hombre está capacitado para responder la razón por la que las personas que nosotros pensamos justas, sufren.\\
Dios constantemente nos somete y normalmente la dificultad que encontramos en las pruebas es que nosotros mismos somos los que nos metemos en ellas.\\
\\
Podemos seguir confiando en Dios en medio de las pruebas a pesar de que veamos que Él no responde a todo lo que le pedimos, esto es muy común cuando vemos que alguien está pasando por una prueba a la cual no le conocemos una razón.\\
En este libro Dios mismo dice que Job era una persona recta, hay gente malvada que reniegan de Dios y sin embargo prosperan.\\
\\ 
La falta de respuesta a esta pregunta es lo que hace que la fe del creyente se debilite. Pensamos que siempre debemos de tener salud plena pero cuando pensamos eso simplemente nos engañamos pues en realidad no conocemos a Dios y a sus promesas, Dios nunca prometió inmortalidad ni salud perpetua sino que tenemos la certeza de que podemos morir a causa de una enfermedad o un accidente. Más aún, cuando estamos enfermos es cuando más buscamos a Dios y Él permanece con nosotros en medio de la pruebas. La intimidad que tenemos en nuestra relación con Dios debe de ser invariante antes las circunstancias adversas que suframos durante el trayecto de nuestra vida.\\
Tener fe como hijos de Dios ante este tipo de circunstancias no es algo opcional ya que no nos corresponde el condicionar a Dios.
\newpage
El relato de esta parte de la historia de Job nos enseña que en realidad debemos de permanecer en Dios a pesar de las circunstancias, sea que no nos dé lo que le pedimoss o peor aún, que nos quite lo que ya tenemos. Dios es tan justo que mandó a sufrir a Su hijo y a morir en una cruz. Permitió que Jesús fuera escupido, humillado y muerto en una cruz. El acto de mayor justicia que podemos entender y ver en la Biblia es que Dios entregó a Su propo hijo.\\
\\
Por lo que nos dice $1^{a}$ Pedro, sabemos que es necesario que pasemos por diversas pruebas. Todo en nuestra vida tiene un propósito eterno para hacernos más humildes o separarnos de la confianza que tenemos en las cosas del mundo.\\ 
Dios permite el sufirmiento también como disciplina por nuestras faltas o consecuencias de ellas. Cristo nos mostró cómo podemos atravesar las pruebas en la vida para llegar a la vida de gloria en la eternidad.\\
\\
Dios entra en escena en el capítulo 38 pues le contesta a Job. Dios le da un discurso de la grandeza de la creación en Job 38:1-3. No valía la pena que Dios le explicara toda la grandeza de las cosas a Job pues ni siquiera le iba a entender. \\
Job se dirigó a Dios con lamento sin recibir respuesta satisfactoria pues Dios no tiene obligación de dar cuenta de sus actos. A la luz de la superioridad del intelecto de Dios, Él le responde. La lección que nos deja es que debemos de humillarnos frente a Dios en una prueba, aceptar que Él es soberano y esperar qué es lo que sigue después de esta revelación. Job, finalmente, se da cuenta de lo pequeño que es.\\
\\
Al reconocer la grandeza de Dios, del creador, él simplemente calla. Dios es la fuente de la sabiduría y del orden, Él sabe lo que está haciendo de manera que Job reconoce que puede confiar en Dios. Humildemente Job responde que hablaba de cosas que no entendía en Job 42:3,  acepta así la voluntad de Dios en medio de la prueba. En medio de su sufrimiento entendió quién era Dios, sus preguntas quedaron contestadas sabiendo que Dios es soberano y ello debería de ser suficiente para que seamos confortados en las pruebas.\\
Dios no ejerce su soberanía de manera caprichosa sino de manera amorosa buscando lo mejor para nosotros. No podemos cuestionarlo ni retarle a hacer algo porque Él sabe lo que hace. Dios es soberano, infinitamente sabio y perfecto en Su amor. Dios siempre desea lo mejor para nosotros, sabe qué es lo mejor y tiene el poder para hacerlo.
\end{subsubsection}
\begin{subsubsection}{El Epílogo (Job 42:7-17)}
Job fue reivindicado frente a sus amigos que, sin misericordia, lo habían tachado de pecador. Sus amigos tampoco habían reconocido el testimonio fiel que había mantenido Job pues lo criticaron cuando debieron de haberle apoyado.\\
	Consideraban a Dios como un dios vengativo que buscaba cualquier excusa para castigar. No se dieron cuenta de Su amor ni de Su perdón para cualquiera que sea fiel a Él. Al final Job es recompensado por su fidelidad. Él fue recompensado con una doble porción de lo que tenía anteriormente, Dios le restauró sus animales y le concedió otra familia.\\
	\\
	A través del relato de Job aprendemos de lo que Dios hace en Su pueblo.
	\newpage
	Puede ser que aún no podamos comprender lo que Dios hace pero debemos de estar dispuestos a confiar en Él. Al terminar el estudio del libro entendemos que la mejor forma por la que Dios lleva a Su pueblo a la madurez es por medio de la aflicción. Estas aflicciones además se resuelven cuando ponemos la vista en Dios.
\end{subsubsection}
\end{subsection}
\end{section}
%\end{document}


