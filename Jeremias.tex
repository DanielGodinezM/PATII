%        File: Jeremias.tex
%     Created: Mon Nov 11 06:00 PM 2019 C
% Last Change: Mon Nov 11 06:00 PM 2019 C
%
%\documentclass[12pt]{article}
%\usepackage{enumerate}
%\usepackage[spanish]{babel}
%\usepackage[margin=1.0in]{geometry}
%\begin{document}
\begin{section}{Jeremías}
	\begin{itemize}
		\item Título\\
			Le fue dado el título según el protagonista principal que es el profeta Jeremías. Su ministerio empezó 112 años después del ministerio de Isaías. Nació en la ciudad de Anatot, ciudad que estaba en el territorio de la tribu de Benjamín que pertenecía al reino del sur.\\
			Anatot era una ciudad a la cual se les había asignado los levitas, ahí vivía la familia de sacerdotes de donde venía Jeremías. Como levita, era sacerdote y posteriormente Dios lo llama para convertirlo en profeta. Se dice que fue descendiente de Abiatar, el sacerdote que se unió con Adonías en contra de Salomón. Salomón destierra a Abiatar por traición y lo manda a Anatot y supuestamente ahí engendró a Jeremías.\\
			Jeremías ministró cuando Asirira estaba en decadencia y Egipto y Babilonia empezaban a crecer, dos pueblos que lucharían por tener el control del mundo. Jeremías puso sobre aviso a su pueblo de que Babilonia sería quien tendría la supremacía sobre las demás naciones. Cuando empezó a profetizar, Babilonia todavía estaba bajo el dominio de Asiria.\\
			\\
			Josías promovió una reforma religiosa y desarrollo una política independiente para tener una vida más cercana a Dios. Dicha reforma es bruscamente interrumpida cuando Josías muere en una batalla, ello marcó que los caldeos, al ser derrotados los asirios, se levantaran como la potencia militar mientras que Judá había perdido a su rey.\\
			Los reyes que le sucedieron a Josías cometieron muchos errores políticos y religiosos. Judá no aceptó las profecías de Jeremías acerca de su próximo juicio, la profecía que decía que si seguían rechazando a Dios serían conquistados por los caldeos.\\
			Jeremías constantemente advertía sobre el juicio próximo de la cautividadd babilónica a menos de que se arrepintieran. Jeremías fue contemporáneo de Habacuc, Sofonías, Ezequiel y Daniel, estando Ezequiel y Daniel en el exilio. \\
			Jeremías se destacó de los demás profetas por su ministerio solitario dado el mensaje que daba de parte de Dios, fue encarcelado en varias ocasiones pero nunca comprometió el mensaje que Dios le había dado.\\
			Jeremías fue muy pobre, contrastando con la vida de Isaías y de Daniel, carente de todo tipo de provisiones para que pudiera cumplir con la tarea que Dios le había dado. Lo rechazaron sus vecinos, su familia, los falsos profetas, los sacerdotes, sus amigos, los reyes y todo aquel que escuchara su mensaje, finalmente fue llevado a Egipto en contra de su voluntad.\\
			Fue considerado como un miserable fracasado, durante 40 años sirvió como mensajero de Dios. Su ministerio fue de los años 627-586 a.C. 
		\item Autor y fecha\\
En Jeremías 1:1-2 se dice que son palabras de Jeremías directamente dadas por Jehová. Jeremías fue asistido por un escriba llamado Baruc, quien fue responsable de escribir todo lo que decía Jeremías. Baruc lo siguió hasta el exilio y por ello se cree que después de la muerte de Jeremías, Baruc se encargó de juntar sus escritos.\\
El libro no presenta un contenido ordenado cronológicamente.
\newpage
El libro fue escrito en los años 627-561 a.C.
Baruc es a quien, además, se le atribuye el escrito apócrifo de Baruc.\\
No se menciona que Jeremías haya tenido otras personas las cuales le hayan sido fieles, su mensaje de fatalidad que anunciaba hacía que las personas se alejaran de él. Judá tenía el mal impregnado, cuando muere Josías, 3 hijos de él lo sustituyen en el trono, Joacaz, Joacim y Sedequías. A diferencia de Josías, ellos fueron pésimos gobernantes.\\
De todos ellos, Joacim fue el que más abiertamente fue hostil en contra de Jeremías, destruyó uno de sus manuscrtos proféticos y lo lanzó al fuego. Sedequías fue un gobernante débil y vacilante que en ocaciones permitió que maltrataran a Jeremías y lo encarcelaran. Jeremías condenó a los sacerdotes y reyes ya que eran los responsables de que el pueblo se estuviera desviando.\\
\\
Jeremías sabía que finalmente los caldeos conquistarían Judá por lo que él faverocía la rendición de su pueblo ante Babilonia y recomendó a los que estaban el exilio que vivieran normalmente, sin rebelarse antes los caldeos. Anticipaba el horror de la destruccción de Jerusalén, juicio que finalmente llegó por su desobediencia.\\
Se destaca el anuncio por primera vez que se hace de un nuevo pacto, iba a ser diferente ya que Dios no grabaría su ley sobre piedra sino que las iba a escribir en los corazones de su pueblo, produciendo asi la capacidad de conocer a Dios y serle fiel. El antiguo pacto había sido invalidado por Su pueblo, por su desobediencia.\\
La fidelidad de Dios era muy clara pero ellos no le fueron fieles a Dios. En Jeremías 18 se hace la ilustración con la relación que hay entre un alfarero y el barro, en Jeremías 13 se lilustra la sobrerbia de Judá con un cinto de lino podrido. Dios hace ilustraciones por medio de Jeremías en estos capítulos y otros más acerca de la condición espiritual del pueblo. Jeremías es cuidadoso diciendo que el arrepentimiento podía evitar dicho juicio.\\
Dios les daría un nuevo pacto y escribiría Su ley en sus corazones, el trono de David sería restablecido.
		\item Tema\\
			La última advertencia de Dios a Judá. Este libro es, primeramente, un mensaje de juicio sobre Judá por su idolatría pues el profeta sabía de la destrucción de la ciudad y del templo. Aconsejó que no se resistieran a sus enemigos y que hicieran alianzas para defenderse, él predicaba que ellos deberían de tener sumisión hacía sus enemigos pues Dios había decidido entregar Jerusalén a los caldeos. Jeremías tenía una actitud realista ya que había sido informado por Dios mismo acerca del castigo que se sobrevenía sobre su pueblo. Jeremías era más sensato, sabía que todo era el cumplimiento de la voluntad divina.\\
			Los contemporáneos de Jeremías no lo entendían, el pueblo pensaba que Dios no iba a permitir que fuera invadida la ciudad santa. 
		\item Propósito\\
			Que el pueblo de Israel entendiera que cuando se separan de Dios quedan totalmente desprotegidos. Después de la muerte de Josías, Judá había abandnado casi completamente a Dios. 
			\newpage
			Jeremías advertía que el juicio de Dios estaba a la puerta, Su misericordia había llegado a su fin. Dios trajo a Nabucodonosor de regreso a Jerusalén para destruir a todo Judá, Jeremías amaba a Judá pero amaba más a Dios, fue obediente a lo que Dios le dijo que hiciera y también sabía que Dios era justo y recto.\\
			Dios no se olvidó de ellos, tenía un propósito para el fiel remanente que quedaría. Dios estaba prometiendo una restauración final para Judá
	\end{itemize}
	\begin{subsection}{Bosquejo}
		\begin{subsubsection}{Profecías sobre Judá (Jeremías 1-45)}
			\begin{enumerate}
				\item Llamado de Jeremías (Jeremías 1)\\
					Se narra el llamamiento de este profeta, Dios le comunica que lo escogió desde antes de su naciemiento y Jeremías le responde a Dios que no era apropiado para la tarea. Dios le asegura que le proveería todo lo necesario para que cumpliera con su ministerio, tocó su boca y le dijo que había puesto las Palabras de Dios para que se preparara y le dijera al pueblo todo lo que ya le había dicho.
				\item Condenación de Judá (Jeremías 2-29)\\
					Se narran 14 mensajes que fueron predicados por Jeremías, le trajeron muchas dificultades pues los exhortaba para que se sometieran voluntariamente a los caldeos. 
					\begin{itemize}
						\item Primer mensaje (Jeremías 2:1-3:5)\\
							Les recuerda el amor que inicialmente el pueblo había mostrado hacia Dios. Dios se enojó en contra de Judá por su infidelidad y dijo que aún los incrédulos eran mejor que ellos. En Jeremías 2:11-13 menciona que los corazones de su pueblo eran como cisternas rotas, que  no retenían la Palabra de Dios. 
						\item Segundo mensaje (Jeremías 3:6-6:30\\
							Jeremías compara al reino del norte con una fornicaria que nunca se arripintió en Jeremías 3:6-8. 
						\item Tercer mensaje (Jeremías 7-10)\\
							Condena que hace Jeremías a Judá por su hipocresía, les advierte que no confíen en los muros del templo ya que a Dios le importaba más preservar a Su pueblo que preservar el templo.\\
							En Jeremías 9 se ve por qué se le conoce como el profeta llorón pues vemos que el profeta estuvo llorando.
						\item Cuarto mensaje (Jeremías 11-12)\\
							Les recuerda que habían violado el pacto que Dios había hecho con ellos en el Sinaí. 
						\item Quinto mensaje (Jeremías 13)\\
							Jeremías usa símbolos y señales para comuncar su mensaje y su cautiverio.
						\item Sexto mensaje (Jeremías 14-15)\\
							Se simboliza su mensaje por medio de una sequía.
							\newpage
						\item Séptimo mensaje (Jeremías 16-17)\\
							Se anuncia juicio contra Judá y se le ordena a Jeremías no tener familia. Dios maldice al hombre que confía en sí mismo y menciona que Él es el único que podría conocer cuanta maldad había en sus corazones.
						\item Octavo mensaje (Jeremías 18-20)\\
							Ilustra la relación entre el pueblo y Dios con una vasija y su relación con el alfarero. En Jeremías 19:6 se habla de Tofet al cual dice que se le llamaría ``Valle de la matanza'', en Jeremías 20 vemos las persecuciones de las cuales fue objeto el profeta y en Jeremías 20:12-18 se lee la desesperación de Jeremías ante todo su sufrimiento. 
						\item Noveno mensaje (Jeremías 21-22)\\
							Predice el cautivero en Babilonia al ver la inminente invasión de Nabucodonosr. En Jeremías 22:30 se dice que de Joaquín, el rey al cual en ocasiones se le llama Jeconías o Conías, siendo de la línea de David, ninguno de sus descedendientes se sentaría en el trono de David. En Mateo se ve la descendencia de Jeconías hasta José. Jeremías había profetizado que nadie más de la línea de Jeconías se sentaría en el trono pero Jesús reinaría en el milenio, el derecho de sangre que tiene Jesús al trono no es de José ya que no era su padre, el derecho de sangre lo obtuvo a través de María quien también fue descendiente de David, de acuerdo a la genealogía en Lucas. En Lucas se habla de José como hijo político de Eli,  quien era padre de María. Lucas detalla la genealogía de María mientras que Mateo detalla la de José.\\
						\item Décimo mensaje (Jeremías 23-24)\\
							Jeremías 23 narra un ay dirigido a los sacerdotes que habían sido infieles.
						\item Décimo primer mensaje (Jeremías 25)\\
							Profecía acerca de los 70 años de cautiverio que tendrían antes de regresar a Jerusalén. El tiempo se cuenta a partir de los años 605-586 a.C. cuando Zorobabel regresa a Jerusalén.
						\item Del décimo segundo al décimo cuarto mensaje (Jeremías 26-29)\\
							Dirigidos a los reyes gentiles y judíos del primer cautiverio.

						\end{itemize}
					\item El nuevo pacto (Jeremías 30-33)\\
						Se ve uno de los principales mensajes de esperanza, un nuevo pacto en el que informa los grandes propósitos que tenía para Su pueblo, trata sobre la salvación basada en el sacrificio expiatorio de Cristo, habla del último sacrificio por el cual iban a ser expiados sus pecados en Jeremías 31:33. \\
						El pacto anterior mencionado es el del Sinaí con Moisés, dice que Dios fue fiel pero que ellos no lo fueron, el nuevo pacto se distinguía ya que no contaba con una ley escrita en una piedra sino que estaría escrita en sus corazones. Les daría una nueva mente y la facilidad para acatar lo que exigía. Este pasaje es citado en Hebreos 8:8-12, 10:16-17. Es una profecía que menciona que ahora nosotros tenemos la capacidad de obedecer los mandamientos que concoemos de la ley de Dios gracias al Espíritu Santo.\\
						En Jeremías 32:2-3 el profeta vuelve a ser apresado.
						\newpage
					\item La caída de Judá (Jeremías 34-45)\\
						Se narra la historia de la caída de Judá en manos de Babilonia y Dios cumple con su advertencia en Jeremías 32:2-3.\\
						En Jeremías 34 se narra la profecía sobre la caída de Jerusalén, empieza con las fuerzas de Nabucodonosor y sus aliados que venían en contra de Jerusalén y las ciudades judías. Dios mismo estaba entregando esta ciudad al rey de Babilonia y sus fuerzas, entraron y destruyeron Jersusalén. En Jeremías 36 Joacim desprecia el mensaje de Dios a través de Jeremías cuando destruyó su escrito. Todo ello fue antes de la destrucción de Jersualén, detalles que nos indica que Baruc no compiló los escritos de manera cronológica.\\
						En Jeremías 37:13-14 se menciona que en esos días la ciudad ya estaba sitiada por Nabucodonosor. En Jeremías 38:6 el profeta es arrojado con sogas a una cisterna con lodo.\\
						Los invasores encontraron a Jeremías en el patio de la cárcel cuando entraron a la ciudad. Los caldeos le permitieron a Jeremías quedarse en la ciudad destruida junto con la gente más probre. La mayoría de sus habitantes fueron llevados cautivos mientras que el remanente fueron los pobres e inútiles.\\
						En Jeremías 43 el pueblo que quedó se rebela y huye a Egipto llevándose a Jeremías.
			\end{enumerate}
		\end{subsubsection}
		\begin{subsubsection}{Profecías sobre los gentiles (Jeremías 46-51)}
			Habla de cuando Jeremías profetizó que también sus vecinos serían conquitsados. Dichas profecías fueron para Egipto, Filistea, Moab, Amón , Edom, Damasco, Arabia, Elam, y Babilonia.
			En Jeremías 50 se menciona una profecía en contra de Babilonia.
		\end{subsubsection}
		\begin{subsubsection}{La caída de Jerusalén (Jeremías 52)}
			Es un capítulo añadido no escrito por Jeremías donde se habla de un resumen sobre la destrucción y deportación de Judá.
		\end{subsubsection}
	\end{subsection}
\end{section}
%\end{document}


