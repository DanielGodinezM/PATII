%        File: Habacuc.tex
%     Created: Wed Nov 20 05:00 PM 2019 C
% Last Change: Wed Nov 20 05:00 PM 2019 C
%
%\documentclass[12pt]{article}
%\usepackage{enumerate}
%\usepackage[spanish]{babel}
%\usepackage[margin=1.0in]{geometry}
%\begin{document}
\begin{section}{Habacuc}
	\begin{enumerate}
		\item Título\\
			Habacuc, que probablemente significa ``uno que abraza'', fue un profeta del reino del sur. EL título toma el nombre del profeta que se describe en el libro, no se sabe de donde era ni su linaje. Se considera que fue contemporáneo de Jeremías, Ezequiel, Daniel y Sofonías.
		\item Autor y fecha\\
			Se le da la paternidad del libro al profeta al que al leer nosotros el texto se le nombra de manera familiar, por lo que probablemente era muy conocido por el pueblo. Su ministerio fue probablemente antes de la invasión babilónica en los años 612-605 a.C.\\
			En este libro veremos un profeta que cuestiona a Dios por considerar que Dios no había castigado a Judá por su pecado, él pensaba que había una contradicción en que Dios permitiera que el pueblo seguiriera pecando.
		\item Tema\\
			El justo por la fe vivirá. El profeta señala en el libro el contraste entre una persona justa y una orgullosa. El orgulloso hace referencia al pueblo caldeo y lo contrastará con aquellos justos del pueblo de Judá que viven por fe. A diferencia del orgulloso, el justo prosperará por su fe en Dios.\\
			Jehová declara que prevalecerán las circunstancias hasta que se cumplan Sus palabras que fueron dadas por el profeta. El justo vivirá día tras día haciendo la voluntad de Dios, no vacilará en tomar el camino correcto ni se verá detenido por las circunstancias de la vida sino que en realidad vivirá por fe. En Romanos Pablo cita Habacuc 2:4 y más adelante dice que la relación con Dios siempre debe de ir acompañada de fidelidad, no se puede separar de la santificación. La prueba de que una persoa es salva se ve manifestada en una vida nueva, ha cambiado su estilo de vida y ha dejado atrás el viejo hombre.\\
		\item Propósito\\
			Confiar en la soberanía de Dios y no juzgar sus caminos. Habacuc afirma que a pesar de la condenación del ejército caldeo él se regocijaría en Jehová cuando Jerusalén fuera destruida pues Judá debe de sufrir las condenaciones del pacto mosaico por su desobediencia. La confianza de Habacuc se ve en el final del libro ya que dice que Jehová es Dios de su salvación y a pesar de la debilidad de Judá, Dios es fortaleza de él. Jehová es el Señor que permite que las cosas sucedan y no se puede dudar de la voluntad de Dios.
	\end{enumerate}
	\begin{subsection}{Bosquejo}
		\begin{subsubsection}{Fe probada (Habacuc 1)}
			La fe del profeta es probada a través de un dialogo, Habacuc se queja de que Dios no le escucha ni le hace caso a la preocupación que tenía de que Judá estaba pecando. En Habacuc 1:2-4 hace referencia a la injusticia que hacía Judá, impíos sobre sus hermanos justos.
			\newpage
			Habla de la injusticia que hacían los nobles de Judá hacia los pobres, si el pueblo no veía el castigo proximo de Dios se alentarían a seguir con su misma conducta opresora y pecaminosa.\\
			El profeta cuestiona a Dios en el mismo pasaje, le cuestion que hasta cuando le clamaría. Vemos al hombre cuestionando a un Dios soberano. La historia nos cuenta como Dios permitió que esos nobles de los que se quejaba Habacuc que eran injustos fueron los primeros que fueron llevados cautivos.\\
			Dios le responde al profeta con una visión y le dice que pronto iba a actuar y confrontar a Judá, la disciplina ya estaba decretada y Habacuc sería testigo de cómo Dios usaría a los caldeos para disciplinar a Su pueblo. Dios tenía lista la herramienta que iba a utilizar para castigar la maldad de Su pueblo.\\
			Le resulta difícil a Habacuc comprender la brutalidad de los caldeos y conciliarlo con un Dios amoroso y santo por lo que nuevamente le protesta a Dios ahora por la brutalidad en la que habría de disciplinar a Su pueblo. 
		\end{subsubsection}
		\begin{subsubsection}{Fe enseñada (Habacuc 2)}
			Aparece un segundo diálogo entre el Señor y el profeta, Dios le contesta al profeta las palabras que más adelante citaría Pablo para estudiar la justificación a través de la fe. Los orgullosos viven confiados en ellos mismos pero los justos viven por su fe en el Dios verdadero, este pasaje lo utilizó Lutero como base para explicar que la savlación es a través de la fe. Dios estaba al tanto de todo, a cada pecador le llegaría su juicio y solamente ocupó a un pueblo incrédulo que también sería disciplinado posteriormente.\\
			Se ven 5 ayes contra Babilonia en Habacuc 2:6, 9, 12, 15, 19. Todos ellos son ayes en contra de los vicios de los caldeos.
		\end{subsubsection}
		\begin{subsubsection}{Fe triunfal (Habacuc 3)}
			En Habacuc 3:3 se narra una oración de Habacuc, todo el capítulo es un salmo. \\
			%La palabra Sigionot es el plural de Sig (término musical).
			Habacuc se sentía totalmente abrumado por las respuestas de Dios de la misma manera que Job se abrumó ante las respuestas de Dios. Primero pensaba que Dios estaba haciendo muy poco pero después le parecía que Dios estaba exagerando y se le revela mostrando sus atributos. Primero Habacuc clamaba por castigo y ahor a pide misericordia para su pueblo, hace una oración de himno de alabanza a la gracia y suficiencia de Dios. La confianza que tenía para con Dios había sido renovada aunque no lo entedía. Habacuc 3:18 es un canto triunfal del profeta al reconocer la grandeza del carácter de Dios.
		\end{subsubsection}
	\end{subsection}
\end{section}
%\end{document}


