%        File: Habacuc.tex
%     Created: Wed Nov 20 05:00 PM 2019 C
% Last Change: Wed Nov 20 05:00 PM 2019 C
%
\documentclass[12pt]{article}
\usepackage{enumerate}
\usepackage[spanish]{babel}
\usepackage[margin=1.0in]{geometry}
\begin{document}
\begin{section}{Habacuc}
	\begin{enumerate}
		\item Título\\
			Fue un profeta del reino del sur. EL título toma el nombre del profeta que se describe en el libro,no se sabe de donde era ni su linaje.
		\item Autor y fecha\\
			Se le da la paternidad del libro al profeta al que al leer nosotros el texto se le nombra de manera familiar, como algien muy conocidopor lo que probablemente era muy concoido por el pueblo. El ministerio de su ministreoo fue probalbente ants de la invasion babilonica en 612-605ac.\\
			En este libro veremos un profeta que cuestiona a Dios por considerar que Dios no había castigado a Judá por su pecado. Pensaba que había una contradicción en que Dios ermitiera que el peublo seguiriera pecando.
		\item Tema\\
			Le justo por la fe vivirá.El profeta señala en el libro el contraste entre una perosna justa y una orgullosa. El orgulloso hace referencia el peblo caldeo y lo contrastará con aquellos justos del pueblo demJdá que vive por fe. a diferencia del oruglloso, el justo prosperará por su fe en Dios.\\
			Jehová delcara que prevalecerán las circunstancias hasta que se cumplasn Sus palabras que dicen por elprofeta. El justo viviŕa d+ías tras días haciendo la voluntad de Dios, no vacilará hcer del camino que debe d tomar i que se verá detenido por a circunstanicas de la vida sino queen realidad vivirá por fe. En Romanos Pablo cita a Habacuc 2:4 y más adelante dice que esta relación con Dios siempre debe de ir acompañada de fidelidad, no se puede separar de la santificación. la prueba de que una persoa es salva se ve manifestada en una vida nueva, cambiada el estilo de vida y que ha dejado el viejo hombre. \\
			El justo por la fe vivirá.
		\item Propósito\\
			Confiar en la soberanía de Dios y no juzgar sus caminos. habacuc afirma que a pesar de la condenación del ejérctio caldeo cuando se destruya Jerusalén dice que Habacuc se regocijará en Jehova pues Judá debe de sufir as condenaciones del pacto mosaico por  su desobediencia. la confianza de Habacuc se ve en el final del libro ya que dice que Jehová es dios de su slavación y a pesar de la debilidad de Judá, Dios es su fortaleza de él. Jehoǘa es el Señor que permite que las cosas sucedan y no se pueden dudar de la voluntad de Dios.
	\end{enumerate}
	\begin{subsection}{Bosquejo}
		\begin{subsubsection}{Fe probada (Habacuc 1)}
			La fe del profeta es probada a través de un dialogo, Habacuc se queja de que Dios no le escucha ni le hace caso a la preocupación que tenía de que Israel,Judá estaba pecando. En vs 2-4 hace referencia a la injusticia que hacía Judá, impíos sobre sus hermanos justos. Hace referenecia acerca de la injusticia que hacían los nobles de Udá hacia los pobres, si el pueblo no veía el castigo proximo de Diose alentarían a eguir con su misma conducta opresora y seguir pecando.\\
			El profeta cuestina a dios en vs 2-4 y le dice a Dios que hasta cuando le clamaría, un reclamo hacia Dios. Vemos al hombre cuestiionando a un Dios soberano. La historia nos cuensta como Dios permitió que esos nobles de los que se quejaba Habacuc que eran injustos fueron los primeros que fueron llevados cautivos.\\
			Dios le responde al porfeta con una visióny le dice que proto iba a ctuar y conforntar a Judá la disciplina ya estaba decretada y habacuc sería testigo de cómo Dios usaría a los caldeos para dsiciplinar a Su ueblo, 1:17. Dios tenía lista a herramienta que iba a utilizar para castigar a maldad de su pueblo.\\
			Le resutlta difícil a Habacuc la brutalidad de los caldeos , conciliarlo con un Dios amoros y santo y nuevamente le protesa a Dios ahora por la forma en la que habría de disciplinar a Su pueblo. 
		\end{subsubsection}
		\begin{subsubsection}{Fe enseñada (Habacuc 2)}
			Aparece unsegundo diálogo entre el Señor y el profeta, Dios le contesta al profeta las palabras que más adelante citaría Pablo para estudiar la justificación a traves de la fe. Los orgulosos viven confiados en ellos mismo pero los ustos viven por su fe en el Dios verdadero, este pasaje lo utilizó Lutero como base para explicar que la savlación es a traves de la fe. Estaba Dios al tanto de todo, a cada pecador le llegaría su juicio. Dios solamente ocupó a un pueblo incrédulo que también sería dsiciplinado psteriormente.\\
			Se ve 5 ayes contra Babilonia (vs 6,9,12,15,19). Todos son ayes en contra de los vicios de los caldeos.
		\end{subsubsection}
		\begin{subsubsection}{Fe triunfal (Habacuc 3)}
			En 3:3 se narra una oración de habacuc, el cap es un salmo. Seillonot es un término plural de Sigallón (término musical). Habacuc se sentía totlamente abrumado por las respuestas de Dios, ede la mismsa manera que Job se sbrumó ante las respuestas de Dios. Primero pensaba que Dios estaba haciendo muy poco pero después le parecía que Dios estaba exagerando y se le revela mostrando sus atributos. Primero clamaba por castigo y ahor apidemisericodira para su pueblo, hace una oración de hino de alabanza a la gracia y suficiencia de Dios. La confiazna que tenía para con Dios había sido renovada aunque no lo entedía. vs 18, aunque falte el producto del olivo \ldots con todo me alegraré``. E sun canto triunfal del profeta al reconocer la grandeza del carácter de Dios.
		\end{subsubsection}
	\end{subsection}
\end{section}
\end{document}


