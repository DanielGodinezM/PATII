%        File: Oseas.tex
%     Created: Wed Nov 20 02:00 PM 2019 C
% Last Change: Wed Nov 20 02:00 PM 2019 C
%
%\documentclass[12pt]{article}
%\usepackage[spanish]{babel}
%\usepackage[margin=1.0in]{geometry}
%\usepackage{enumerate}
%\begin{document}
\begin{section}{Oseas}
	\begin{enumerate}
		\item Título\\
			Oseas fue el profeta del reino del norte por el cual se le da el título al libro. Su nombre tiene el mismo significado que el de Josué y el de Jesús, ``salvación''. Este personaje no es mencionado en algún otro libro pero se sabe que era una persona muy culta. Llama al reino del norte con varios nombres distintos tales como: Israel, Efraín (tribu más grande del reino del norte), Samaria (capital), entre otros. 
		\item Autor y fecha\\
			Se considera que lo escribió Oseas en los años 753-710 a.C. aproximadamente. Fue contemporáneo de Amós que profetizó en el reino del norte y de Isaías y Miqueas que profetizaron en el reino del sur. Jeroboam II gobernó en Israel y Jotan y Ezequías reinaron en el reino del sur en el tiempo en el que Oseas estaba profetizando. Oseas fue la última voz que escuchó el reino del norte por parte de Dios para que se arrepintieran, no escucharon a Oseas y el juicio de Dios fue llevado a cabo.
		\item Tema\\
			El amor de Dios para Israel. Dios muestra la lealtad de su amor hacia el pueblo con el que había hecho sus pactos. El libro contiene mucha condenación pero se trata profundamente del amor que tenía Dios hacia su pueblo.
		\item Propósito\\
			Mostrarnos que Dios es fiel. Es firmemente constante la fidelidad de Dios que se revela para Israel en el cumplimiento de los pactos que había hecho con Su pueblo. 
	\end{enumerate}
	\begin{subsection}{Bosquejo}
		\begin{subsubsection}{Esposa adúltera - Esposo fiel (Oseas 1-3)}
			Durante el reinado de Jeroboam II Israel estaba disfrutando de prosperidad material y paz política, Jeroboam II fue el rey más importante porque hizo prosperar al reino del norte a pesar de su gran idolatría un poco antes de que la capital, Samaria, cayera en manos de invasores. Oseas profetiza de manera singular a través de su propia experiencia, es instruido por Dios a casarse con una mujer fornicaria llamada Gomer, hija de Diblaim.\\
			Gomer no era una prostituta sino que se iba a volver adúltera con el paso del tiempo. Oseas fue y se casó con ella y muestra el reflejo de la infidelidad del pueblo de Dios con su vida. Gomer le es infiel y se va detrás de otros amantes, mostrando de manera gráfica la infidelidad que tenía Israel para con Dios que los seguía amando. El primer hijo de Oseas fue Jezreel, el nombre de un sitio donde habían ocurrido hechos sanguinarios. Jezreel era una ciudad que vio el crimen de Acab y Jezabel en contra de Nabot y también allí fue profetizado por Elías que Jezabel moriría y sería comida por perros.\\
			Ponerle el nombre de Jezreel a su hijo es algo negativo ya que no indicaba bendición por parte de Dios sino destrucción. Después Oseas tiene una hija a la cuall llama Lo-ruhama que quiere decir ``no compadecida''. A pesar de estas manifestaciones visibles acerca de lo que le esperaba a Israel, seguían sin ponerle atención. \\
			Después se narra el nacimiento de su tercer hijo al cual llama Lo-ammi que significa ``no pueblo mío'', otro mensaje para el pueblo de Israel. Dios les habla directamente diciéndoles que no son Su pueblo y que Él no es su Dios. En Oseas 3 vemos que Gomer es vendida por su esposo como esclava pero Dios le ordena aOseas que la vuelva a comprar y la restaure, esto es mencionado en Oseas 3:2.
		\end{subsubsection}
		\begin{subsubsection}{Israel adúltera - El Señor fiel (Oseas 4-14)}
			Oseas describe con los labios lo que había mostrado su vida conyugal. Israel era un pueblo blasfemo, homicida, reblede, idólatra y el peor testimonio venía de parte de los sacerdotes, quienes fuera de ser una guía para el pueblo, lo corrompieron. Oseas 4-5 están escritos en imperativo, ``oíd'' les dice Dios pero ellos nunca le oyeron ni le hicieron caso a sus advertencias.\\
			Por causa de ese pecado Israel fue removida de la tierra que Dios les había dado. Eran actos externos de religiosidad lo que tenían pero no tenían verdadera comunión con Dios.\\
			En Oseas 7 se dice que todo el pueblo ya se había olvidado de Dios pero Él no se había olvidado de ellos y los compara con un horno por sus pasiones políticas. Depués cambia el símil y lo representa con una paloma inestable que vaga en su política pero que Dios la casaría. \\
			En el juicio serán castigados, Dios expresa el anhelo de rescatarlos pero ellos quieren seguir por su camnio de destrucción. Dios disciplinó a Israel pero en Su amor y misericordia los va a restaurar al remanente que muestre arrepentimiento de todas la tribus de Israel según Oseas 14:4 \\
		\end{subsubsection}
	\end{subsection}
\end{section}
%\end{document}


