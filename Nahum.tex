%        File: Nahum.tex
%     Created: Wed Nov 20 05:00 PM 2019 C
% Last Change: Wed Nov 20 05:00 PM 2019 C
%
%\documentclass[12pt]{article}
%\usepackage{enumerate}
%\usepackage[spanish]{babel}
%\usepackage[margin=1.0in]{geometry}
%\begin{document}
\begin{section}{Nahúm}
	\begin{enumerate}
		\item Título\\
			Nahúm, que significa ``consuelo'', era un profeta del reino del norte por el cual se le da el título al libro. En Nahúm 1:1 se identifica al profeta como Nahúm de Elcos donde Elcos era probablemente era Capernaum que significa ``el pueblo de Nahúm''.
		\item Autor y fecha\\
			La fecha se tiene situada ya que se habla del juicio sobre Nínive, capital de Asiria. En Nínive, un siglo antes se habían arrepentido a través del mensaje de Jonás pero ahora el mensaje de Nahúm es de juicio. Aparentemente la nueva generación de Nínive volvió a caer en idolatría cuando Asiria estaba en la cima de su poder, cuando habían conquistado al mundo que se conocía en ese entonces.\\
			Asiria había destruido a Israel y ahora estaba amenazando a Judá. Dios no volvió a perdonar a Nínive, ya había prolongado Su gracia cuando se arrepintieron por el mensaje de Jonás pero los castigó cuando volvieron a su pecado. De acuerdo a ello es que se ha determinado que el libro fue escrito en el año 642 a.C. aproximadamente.\\
			Se considera, además, que Sofonías y Nahúm fueron contemporáneos.
		\item Tema\\ 
			El juicio es sobre todas las naciones. El profeta clama 3 veces la venganza de Dios sobre sus enemigos, poco consideran las naciones cómo provocan la ira de Dios cuando atacan a Su pueblo, Dios no solamente es lento para la ira y fortaleza de los que confían en Él sino que además no tendrá por inocente al culpable. 
		\item Propósito\\
			Entender que Dios aborrece el pecado independientemente de quien se trate, el pecado no es algo cualquiera ya que nos trae consecuencias y además impacta a nuestro Dios. Nunca vemos el pecado correctamente hasta que vemos que es una ofensa directa en contra de Dios, quebrantamos la ley que nos dio.\\
			El poder que tiene el pecado en la naturaleza pecaminosa del hombre es disminuido al ver el peacdo como lo ve Dios. Aunque en esta vida no seremos perfectos, si verdaderamente le tememos seremos más santos y caminaremos en santidad hacia la meta.
	\end{enumerate}
	\begin{subsection}{Bosquejo}
		\begin{subsubsection}{La destrucción de Nínive declarada (Nahum 1)}
			En la primera parte del mensaje Nahúm muestra a Dios como juez. Dios es santo, aborrece el pecado y enjuicia a quienes pecan y no se arrepienten. Nínive ya había conocido un mensaje previo por parte de Dios pero después de un arrepentimiento momentáneo le siguieron ofendiendo y por lo tanto merecían el juicio que ciertamente no iba a tradar. En Nahúm 1:3 dice que no tendrá por inocente al culpable.\\
			Dios, después de que Nínive se había arrepentido pero regresó a su vida pecaminosa, ve su pecado insistente por lo que desató su ira y decretó la destrucción de la ciudad.
		\end{subsubsection}
		\newpage
		\begin{subsubsection}{La destrucción de Nínive detallada (Nahum 2)}
			El profeta detalla cómo sería la destrucción de la ciudad. En el hebreo original Nahúm 1:15 forma parte de Nahúm 2 ya que es conclusión de Nahúm 1 pero a la vez es la introducción de Nahúm 2. En Nahúm 1:15  la expresion ``he aquí\ldots'' es una forma típica de introducir una nueva idea.\\
			La destrucción de Nínive sería un ataque furioso y habría una gran matanaza, un pueblo poderoso se enfrentaría en su contra, destruirían a la ciudad y las calles serían llenas de soldados que destruirían todo. Nahúm describe esos horribles días para la ciudad de Nínive a partir de Nahúm 2:3.\\
			Nínive será retribuida con mayor violencia que la que hacía con los pueblos que había destruido. Nínive se caracterizaba por ser cruel con las naciones que conquistaba.
		\end{subsubsection}
		\begin{subsubsection}{La destrucción de Nínive demandada (Nahum 3)}
			Se dan las razones del juicio de Nínive, una ciudad sanguinaria. Sería destruida brutalmente ya que ella así lo había hecho con sus enemigos además de haber ofendido a Dios con su idolatría, ya no había forma de sanar al pueblo enfermo. Su pecado era conocido por todos de acuerdo al final del libro. Nínive estaba recibiendo lo justo de acuerdo a su maldad. \\
			Nahúm termina siendo enfático en el carácter irreversible de Nínve y del reino Asirio, ya no había posibilidad de arrepentimiento así como había habido en el tiempo de Jonás. Nahúm termina con la desaparaicón de Nínive que era un mensaje de esperanza para aquellos que vivían bajo dicho imperio.\\
			La conclusión de esta profecía se cumplió 50 años después a manos de los caldeos en el año 612 a.C.
		\end{subsubsection}
	\end{subsection}
\end{section}
%\end{document}


