%        File: Nahum.tex
%     Created: Wed Nov 20 05:00 PM 2019 C
% Last Change: Wed Nov 20 05:00 PM 2019 C
%
\documentclass[12pt]{article}
\usepackage{enumerate}
\usepackage[spanish]{babel}
\usepackage[margin=1.0in]{geometry}
\begin{document}
\begin{section}{Nahúm}
	\begin{enumerate}
		\item Título\\
			Nahúm era un profeta del reino dle norte, el título del libro en 1:1 se identifica al profeta como Nahúm de Elcos. Elcos era un lugar probablemtne era Capernaum que significa ``el pueblo de Nahúm''.
		\item Autor y fecha\\
			Fecha situada ya que se abla del juicioi sobre Nínive, capital de Asiria. En Nínive un siglo antes se habían arrepentido a través del mensaje de Jonás. El mensaje de Nhaaúm es de juicio. Aparentemente la nueva generación de Nínive volvió a caer en idolatría cuando Asiria estaba en la cima de su poder, había conquistado al mundo que se conocía en ese entonces.\\
			Asiria había destruido a Israel y ahora estaba amenazando a Judá. Dios no volvió a perdonar a nínive, ya había rolongado su gracia cuando se arrepintieron por e  mensaje de Jonás pero los castigó cuando volvierona  su pecado. 642 ac escrito aprox.\\
			Se considera que Sofon+ías y Nahú fueron contemporáneos.
		\item Tema\\ 
			El juicio es sobre todas las naciones. El profeta clama 3 veces la venganza de Dios sobre sus enemigos, s¿poco considera las naciones cómo provocan la ira de Dios cuando atacan a Su pueblo, Dios debido a su carácter, no solamente es lento para la ira y es fortaleza de los que confían en Él ero tambíén no tendrá por inocente al culpable. 
		\item Propósito\\
			Enteder que Diosaborrece el pecado independiemten de quien se trate, el pecado no es algo cualquiera ya que nos trae consecuencias y además impacta a nuestro Dios, nunca vemos el pecado correctamente hasta que vemos que es una ofensa directa en contra de Dios, esa ees la forma que al quebrantar la ley que nos dio \ldots \\
			El poder que tiene el pecado en la natrualeza pecaminosa del hombre es dimsi uido al ver el peacdo como lo ve Dios, aunque en esta vida no seremos erfectos si verdaderamente le tememos seremos más santos y caminaremos en santidad hacia la meta.
	\end{enumerate}
	\begin{subsection}{Bosquejo}
		\begin{subsubsection}{La destrucción de Nínive declarada (Nahum 1)}
			EN la primera parte del mensaje Nahúm muestra a Dios como juez, dios es santo aborreceel pecado y enjuicia a quienes pecan y no se arrepienten. Nínive ya había conocido u mensaje previo por parte de Dios pero después de un arrepentimiento momentáneo le siguieron ofendiendo y por lo tanto merecía el juicio y ciertamente no iba a tradar en 1:3 dice que no tendrá por inocente al culpable. \\
			Dios  , despué sde que Nínive se había arrepentido pero que regresóa  su vida pecaminosa, al ver Dios el pecado insistente desató su ira y decretó destruirla
		\end{subsubsection}
		\begin{subsubsection}{La destrucción de Nínive detallada (Nahum 2)}
			El profeta detalla como será la destrucción de la ciudad. En el hebreo original 1:15 forma parte del cap 2, es conclusión del primer cap e intro del cap 2. En el vs 13 ``he aquí\ldots'' es una forma típica de introducir una nueva idea.\\
			Sería un ataque furioso y habría una gran matanaza, un puelbo poderoso se enfrentaría en su contra, destruirían a la ciudad, las calles erían llenas de soldados que destruirían todo. Nahúm describe esos horribles días para la ciudad de Nínive, 2:3.\\
			Nínive será retribuida exactamente igual a la violencia que hacía con los pueblos que había destruido pero sería mayor, Nínive se caracterizaba por ser curelcon las naciones que conquistaba.
		\end{subsubsection}
		\begin{subsubsection}{La destrucción de Nínive demandada (Nahum 3)}
			Se dan las razones del juicio de Nínive, era una ciudad sanguinaria. Sería destruida brutalmente ya que ella así lo había hecho con sus enemigos además de haber ofendido a Dios con sus idolatrías, no había forma de sanar al pueblo enfermo. Su pecado era conocido por todos de acuerod al final del libro. Nínive estaba recibiendo lo justo de acuerdo a su maldad. nahúm termina siendo enfático en el carácter irreversible de Nínve y del reino Asirio, ya no había posibilidad de arrepentimiento así como había habido en el tiempo de Jon+as. Nahúm termina con la desaparaicón de Nínive que era un mensaje de eesperanza para aquellos que vivían bajo dicho imperio.\\
			El cumplimiento de esta profecía se cumplio 50 años después a manos de los caldos en el 612a.c.
		\end{subsubsection}
	\end{subsection}
\end{section}
\end{document}


