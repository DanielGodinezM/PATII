%        File: Jonas.tex
%     Created: Wed Nov 20 02:00 PM 2019 C
% Last Change: Wed Nov 20 02:00 PM 2019 C
%
%\documentclass[12pt]{article}
%\usepackage{enumerate}
%\usepackage[spanish]{babel}
%\usepackage[margin=1.0in]{geometry}
%\begin{document}
\begin{section}{Jonás}
	\begin{enumerate}
		\item Título\\
			Jonás, hijo de Amitai, es el nombre del profeta de la historia de este libro. Jonás ,quien era del reino del norte, ministró durante el reinado de Jeroboam II. A diferencia de los demás libros éste no narra visiones sino que narra la misión que le había dado Dios al profeta, ir a profetizar a una ciudad que era la capital del imperio Asirio, no al pueblo de Dios. El nombre Jonás significa ``tórtola ''o ``paloma''. 
		\item Autor y fecha\\
			Tradicionalmente se considera que el autor del libro fue Jonás y se sitúa la fecha de escritura del libro en los años 793-753 a.C. aproximadamente de acuerdo al reinado de Jeroboam II.\\
			Es muy conocido el libro ya que se narra que el profeta vivió dentro de un pez por 3 días, hecho que causa que muchos lo consideren como una parábola. Vemos dentro de las Escrituras que Jesús mismo considera dicha narración como histórica en Mateo 12:38-41.\\
			Jesús atestigua que fue un hecho real, se refirió a Jonás como alguien que verdaderamente exisitió. En el libro se aprecia la gracia de Dios que se extiende a toda nación. Dios manda a Jonás a predicar a una nación gentil para mostrar la gracia de Dios para los pueblos gentiles. La nación a la que le predicó Jonás fue Asiria, este destino lo escogió Dios para Jonás, que fuera a predicar Su mensaje. Un mensaje para un pueblo brutal y odiado que destruía con saña a sus enemigos.\\
			Jonás no quería el arrepentimiento de sus enemigos sino su juicio, no quería que fueran perdonados por sus faltas.
		\item Tema\\
			La salvación es para todas las naciones. Mediante la gracia divina sabemos que Dios es Dios de todas las naciones de acuerdo a la promesa que le había dicho a Abraham. Dios no puede tener un plan de salvación exclusivo para un solo pueblo ya que no hay dstinción entre judíos y gentiles, la pecaminosidad es de todos los seres humanos. Dios manda al profeta para llevarle el mensaje de arrepentimiento y salvación a esa nación mostrando un caso único ya que siendo profeta del reino del norte, Dios lo utlizó para mostrar que Su misericordia era para todos los pueblos de la Tierra.
		\item Propósito\\
			Mostrar que debemos de ser obedientes a los mandatos de Dios, la obediencia está por encima de cualquier otro deber. Su trabajo consistía en ir a las personas y los lugares a los que Dios le mandaba. Jonás era siervo del Señor, no desconocía el carácter de Dios pero su desobediencia lo llevó a olvidarse de su posición como profeta.\\
			Dios, aunque seamos desobedientes, nos protege de los peligros mayores y nos da una oportunidad más para que rectifiquemos nuestra conducta frente a Él. Las consecuencias de la desobediencia son terribles pero Él siempre tiene la solución y puede restaurarnos.
	\end{enumerate}
	\newpage
	\begin{subsection}{Bosquejo}
		\begin{subsubsection}{Avivamiento para un profeta (Jonás 1-2)}
			Dios provoca el avivamiento en un pueblo pero primero lo provoca en un hombre, su profeta. Dios envía a Jonás a Nínive a anunciar la inminente destrucción de la ciudad si no se arrepentían. Envió a un profeta a una ciudad que él odiaba porque era la enemiga de su pueblo. La instrucción de Dios es un caso único, Jonás era un profeta de Israel y ese reino estaba apunto de ser destruido por los asirios, por ello es que tenía tanta resistencia de ir a profetizarles pero Dios le quería mostrar que tenía misericordia de todo hombre.\\
			En Jonás 1:2 Jonás seguramente sabía que Dios tenía una intención real de que se arrepintieran los asirios para perdonarles sus falta por lo cual Jonás no quería ser parte del perdón de Dios para sus enemgios, se levantó temprano y salió huyendo pero al estar Jonás en una barca Dios utiliza Su poder para hacerlo regresar. Los marineros oraron y echaron suertes para ver quien había enfurecido a alguna divinidad para tener semjante agitamiento de los mares. Jonás fue arrojado al mar por sugerencia del profeta mismo y Dios tenía un pez grande ya preparado para que tragase a Jonás en Jonás 1:17.\\
			En Jonás 2 ya está dentro del pez y entonces Jonás entiende que de Dios nadie puede esconderse por lo que se arrepiente, todo ello lo había causado Dios para darle un avivamiento al profeta y Dios le indica al pez que lo vomite.
		\end{subsubsection}
		\begin{subsubsection}{Avivamiento para una ciudad (Jonás 3-4)}
			Dios vuelve a repetir la orden a Jonás, había aprendido la lección y a los 3 días ya estaba predicando en Nínive. El pueblo creyó en sus profecías de acuerdo a Jonás 3:4-5. El pueblo agobiado proclamó ayuno y se arrepintieron de sus pecados esperando que Dios no llevara a cabo Su juicio. El avivamiento para la ciudad había evitado el juicio para los habitantes de Nínive. Cuando Jonás vio que el pueblo se arrepentía no sentía alegría por el odio que ya tenía en contra del pueblo, Jonás se enojó porque Dios estaba mostrando Su misericordia ante el pueblo que odiaba.\\
			Jonás de forma encaprichada fue a sentarse en una enramada y Dios le vuelve a mostrar su paciencia. Surge una calabacera para darle sombra al profeta pero vemos que Jonás le da más valor a las cosas creadas, en este caso la calabacera, que a la gente. Jonás no le da el crédito a Dios pero Él le muestra que debe estar agradecido con Él y no con la calabacera. Jonás se quiere morir nuevamente, vemos que todos habían obedecido a Dios menos el profeta. La tormenta apareció a la orde de Dios, los marineros obedecieron la voz de Dios, el pez se comió a Jonás por orden de Dios y el pueblo de Nínive se arrepintió por orden de Dios así como la calabacera salió por orden de Dios. Todos obedecieron a Dios excepto el profeta.\\
			Dios le hace ver que así como él tuvo lástima de la calabacera así Él tuvo lástima de más de 120,000 personas. Fue un mensaje divino también para Su pueblo para que se avergonzara de que un pueblo pagano le hiciera más caso a Su mensaje.
		\end{subsubsection}
	\end{subsection}
\end{section}
%\end{document}


