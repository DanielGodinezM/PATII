%        File: Jonas.tex
%     Created: Wed Nov 20 02:00 PM 2019 C
% Last Change: Wed Nov 20 02:00 PM 2019 C
%
\documentclass[12pt]{article}
\usepackage{enumerate}
\usepackage[spanish]{babel}
\usepackage[margin=1.0in]{geometry}
\begin{document}
\begin{section}{Jonás}
	\begin{enumerate}
		\item Título\\
			Es el nombre del porfeta de la historia de este libro. Ministró durante el reindao de Keroboam II, era del reino del norete. A diferencia de los demás libros, no narra porfecías ni visiones sino que narra la misión que le hbaía dado Dios al profeta a una ciudad que era la cpaital del imperio Asirio, no parte del pueblo de Dios. El nombre significa ``tótolra ''o ``paloma''.
		\item Autor y fecha\\
			radicionalmente se considera que fue Jonás. Sitúa la fecha del libro en los años 793-753ac aprox de acuerdo al reinaod de Jeroboam II.\\
			Es muy conocido ellibro ya que se narra que el orfeta vivió dentro de un pez por 3 días por lo que muchos lo consideran como una parábola. Vemos dentro de las Escrituras que jesús considera dicha narración como histórica en Mateo 12:38-41.\\
			Jesúss atestigua que fue un hecho real, se refirió a Jonas como algine que verdaderamente exisitió. En el libro se aprecia la gracia de Dios que s extiende a toda nación. Dios manda a Joná a predicar a una nación gentil. La gracia de Dios para los pueblos gentiles. La nación a la que e predicó Jonás fue a Asirir, este destino lo escogió Dios para Jonás, que fuera a predicar su mensjae. Un mensaje para un pueblo brutal , que deruía con zaña a sus enemigos, razón por la cual los asirios eran odiados.\\
			Jonás no quería el arrepentimiento de sus enemigos sino su juicio,nonquería que fuera perdonado por sus faltas.
		\item Tema\\
			La salvación es para todas las naciones. Mediante a gracia divina sabemos que Dios es dios de todas las naciones de acuerod a la promesa que le había dichoc a Abraham. No puede tener un plan de salvaciónexlcusivo para un solo pueblo, no hay dstinción entre judíos y gentilos ya que la pecaminosidad es para todos los seres humnaos. Dios manda al prfeta para llevarle a la nacipon el mensaje ede arrapentimiento y salvación, es unc aso único ya que siendo profeta del reino del nrote Dios loútlizó para mostrada que su misericiodria era para todos los pueblos de la tierra.
		\item Propósito\\
			Mostrar que debemos de ser oedientes a los madnatos de Dios, La obeiencia está por encima de cualquier otro deber. Su trabajo cosnitía en ir a las peronas y los ligar es a los que Dios le mandaba. Jonás era siervo del Señor, no desconocía el carácter de Dios pero su deosbediencia lo levó a olbvidarse de su posición como profeta.\\
			Dios, aunque seamos desobedientes, ns protege de todos los peligors mayores y nos da una oportunidad más para que rectifiquemos nuestra conducta frente a Él. las cosnecuencias de las desocbendiecia son terribles pero Él siempre tiene la solución y puede restaurarnos.
	\end{enumerate}
	\begin{subsection}{Bosquejo}
		\begin{subsubsection}{Avivamiento para un profeta (Jonás 1-2)}
			Dios provoca el avisamtento en un pueblo pero primero lo provoca en un hombre, su profeta. Dios envía a Jonás a Nínive a a nunciar la inmimnete destrucción de la ciudad si no se arrepentían. Envió a un profeta a una ciudad que él odiaba porque era la enemiga de su pueblo. La isntrucción de Dios es un caso único , Jonás era un rpfeta de Israel y ese reino estaba apunto de ser dedstruidopor los asirios, por ello es que tenía tanta resistencia de ir a profetizarles pero Dios le quería mstrar que tenía misericodiria de todo hombre.\\
			en 1:2 Jonás seguramente sabía que Dios tenía una intención real de que se arrepintieran los asirios para perdonarles sus falta pero oJonás no uqería ser parte del perdón de Dios a sus enemgios. Se levantó temprani y salió huyendo pero Dios puede controlar los movimiento de cualquier personapero al estr estar Jonás en una barca Dios utiliza su poder para hacerlo regresar. Los marinerps oraron y echaron suertes para ver quien había enfurecido a una divinidad para tener semjanete agitamiento de los mares. Jonás fue arrojado al mar por sugerencia del profeta y Dios tenía preparado un pez grande ya preparado para que tragase a Jonás, 1:17.\\
			En el cap 2 ya está dentro del pez y entonces Dios entiende que de Dios nadie puede desconderse y se arrepiente, todo ello lo había causado Dios para darle un areavivamiento al profeta y Dios le indica al pe que lo vomite.
		\end{subsubsection}
		\begin{subsubsection}{Avivamiento para una ciudad (Jonás 3-4)}
			Dios vuelve a repetir la orden a Jonás, había aprendidio la lección y a los 3 días ya estaba predicando. El pueblo creyó en sus profecías, Jonás 3:4-5. El pueblo agoviado proclamó ayuno, se arrepintieron de sus pecados esperando que Dios no llevara a cabo su juicio. El avivameiteo para la cudad había evitado el juicio para los habitantes de Nínive. Cuando Jonás vio que el pueblos e arrepentía no sentía alegría por su odio en contra del pueblo, Jonás se nojó porque Dios estaba  mostrando su misericordia antes el pueblo que odiaba.\\
			Jonás de forma encapricahada fue a sentarse en una enramada y vemos la paciencia de Dios. Surge una clabacera para darle sombra al porfeta pero le da más valor a las cosas creadas que a la gente. No le da el crédito a Dios y le muestra que debe estra agradecido con Él y no con la calabecar.Jonás se quiere morir nuevamente, todos obedeierona Dios menos el profeta, la tormenta apareció a la orde de Dios, los marineros obedecieron la voz de Dios, el pez se comió a Jonás por odren de Dios y el pueblo de Nínive por orden de Dis, etc. Todos obedecieron a Dios excepto el profeta.\\
			Dios le hace ver que así como él tuvo lastimas de la calabacer así el tuvo lástima de más de 120 000 personas. Fue un mesnaje para que su pueblo se avergonzara de que un pueblo pagano le hciera más caso a su mensaje.
		\end{subsubsection}
	\end{subsection}
\end{section}
\end{document}


